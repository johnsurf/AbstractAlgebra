%   11/11/92 211121724  MEMBER NAME  REPORT1  (TEXFILES) M  TEX
%format: latex    %hyphen: English
%\documentstyle[11pt,leqno,epsfig]{note}
%\documentstyle[11pt,epsfig]{note}
%\documentstyle[11pt,epsfig,pstimes,psmath]{note}
%\documentstyle[epsf]{article}
 
%\documentclass{article}
\documentclass[epsf]{article}

\font\twelverm=cmr12
\font\twelvebf=cmbx12
\font\tenbf=cmbx10
\font\tenrm=cmr10
\font\tenit=cmti10
\font\twelveit=cmti12
\font\elevenbf=cmbx10 scaled\magstep 1
\font\elevenrm=cmr10 scaled\magstep 1
\font\elevenit=cmti10 scaled\magstep 1
\font\ninebf=cmbx9
\font\ninerm=cmr9
\font\nineit=cmti9
\font\eightbf=cmbx8
\font\eightrm=cmr8
\font\eightit=cmti8
\font\sevenrm=cmr7

%\usepackage{amsmath,amssymb,latexsym} 
\usepackage{amsmath,amssymb,latexsym,hyperref} 
\usepackage{easybmat,graphicx}
\usepackage{multirow,bigdelim}
\usepackage{caption}
\usepackage{subcaption}
%\input ../inputs/epsf.tex

\newcommand*\hexbrace[2]{%
  \underset{#2}{\underbrace{\rule{#1}{0pt}}}}
\hypersetup{
    colorlinks=true,
    linkcolor=blue,
    filecolor=magenta,      
    urlcolor=cyan,
}

%\hypersetup{
%    bookmarks=true,         % show bookmarks bar?
%    unicode=false,          % non-Latin characters in Acrobat�s bookmarks
%    pdftoolbar=true,        % show Acrobat�s toolbar?
%    pdfmenubar=true,        % show Acrobat�s menu?
%    pdffitwindow=false,     % window fit to page when opened
%    pdfstartview={FitH},    % fits the width of the page to the window
%    pdftitle={My title},    % title
%    pdfauthor={Author},     % author
%    pdfsubject={Subject},   % subject of the document
%    pdfcreator={Creator},   % creator of the document
%    pdfproducer={Producer}, % producer of the document
%    pdfkeywords={keyword1, key2, key3}, % list of keywords
%    pdfnewwindow=true,      % links in new PDF window
%    colorlinks=true,       % false: boxed links; true: colored links
%    linkcolor=blue,          % color of internal links (change box color with linkbordercolor)
%    citecolor=green,        % color of links to bibliography
%    filecolor=magenta,      % color of file links
%    urlcolor=cyan           % color of external links
%}

%
%  `pstimes.sty' sets \rm to be Times, and `psmath.sty' sets the math
%  in Times as well.
%
\textwidth = 7.0truein
\textheight = 8.5truein
\oddsidemargin=-0.1truein
\evensidemargin=-0.1truein
\topmargin=0.1truein
%\notename{UCD/IIRPA} \notenumber{92-24}

\begin{document}
%\null
%\vspace{-2cm}
%\hbox to \hsize{\hfill UCD/IIRPA~92-24}

\begin{center}{{\twelvebf Notes on Abstract Algebra and Geometry}}
		\vglue 1.0cm
 	       {\tenrm JOHN R.\,\,SMITH\\}
	       {email: jrsmith@ucdavis.edu\\}
		\vglue 0.25cm
	        \baselineskip=13pt
	       {\tenit Physics Department\\
	        University of California, Davis\\}
                 \baselineskip=12pt
                 {\tenit Davis, CA 95616-8677, USA\\}
		\vglue 0.5cm
                 {\rm April 14, 2019 \\}
\end{center}

\input mytexdefs.tex
%
%\maketitle

%some definitions
%\setlength{\oddsidemargin}{0cm}
%\setlength{\textwidth}{17cm}
%\setlength{\topmargin}{-3cm}
%\setlength{\textheight}{25cm}
%\setlength{\unitlength}{1mm}
%\hsize=6.0 true in
%\vsize=8.40 true in
%\pagestyle{plain}

\parindent = 0.cm
%\vglue 0.3cm
%\rightskip=3pc
%\leftskip=3pc
\twelverm
\baselineskip=14pt
\noindent

%\input /Users/johnrsmith/tex/inputs/phyzzx.tex
%\input /Users/johnrsmith/tex/inputs/myphyx.tex
%$$p(x,D)\psi(x) = {1\over(2\pi)^n}\int_{R^n}\int_{K^n} e^{i k\cdot x}p(x,k)e^{-i k\cdot y}\,d^nk\psi(y)\,d^ny,\eqno(8)$$
%\begin{equation}p(x,D)\psi(x) = {1\over(2\pi)^n}\int_{R^n}\int_{K^n} e^{i k\cdot x}p(x,k)e^{-i k\cdot y}\,d^nk\psi(y)\,d^ny,\label{eqn:symbol}\end{equation}
%Eq.[\ref{eqn:symbol}]
%\begin{equation*}p(x,D)\psi(x) = {1\over(2\pi)^n}\int_{R^n}\int_{K^n} e^{i k\cdot x}p(x,k)e^{-i k\cdot y}\,d^nk\psi(y)\,d^ny,\end{equation*}

%\input Introduction.tex

\section{Logic}

\begin{definition}
If a statement $A$ leads logically to another statement $B$ we say that `if $A$, then $B$' and write $A \rightarrow B$. Another way to think about implication is that $A\rightarrow B$ can be thought of as `if $A$ is true, then $B$ must be true' (`$B$, if $A$') or that $B$ is {\twelveit necessary} for $A$. This can also be expressed as `$A$ {\twelveit only if} $B$'.
\end{definition}

$A\rightarrow B$ implies that $A$ being true is sufficient for $B$ to be true. If the we have both $A\rightarrow B$ and $B\rightarrow A$, then we can say that $A$ is {\twelveit necessary and sufficient} for $B$ -- which can also be stated as $A \iff B$.\\

\begin{definition}
$n$-dimensional Affine Space $\R_n$. An ordered $n$-tuple of real numbers, i.e., a system of $(x_1, x_2, ..., x_n)$ of $n$ real numbers in a definite order, is called a point ($n$ a positive integer). The numbers
$x_1,x_2, ...,x_n$ are called the coordinates of the point; in particular $x_1$ is called its first, $x_2$ its second, ..., $x_n$ its $n$-th coordinate. Two points $P=(x_1,x_2,...,x_n)$ and $Q = (y_1,y_2,...,y_n)$ are said to 
be the {\elevenit same} or to coincide if and only if $x_1 = y_1, x_2 = y_2, ..., x_n = y_n$. The totality of all $n$-tuples of real numbers is called $n$-dimensional Affine Space and is denoted by $\R_n$.
\end{definition}

\begin{definition}{\bf The laws of addition}:\\
\begin{itemize}
\item[A1:] The Commutative Law of addition: $x+ y = y + x$ for all pairs of numbers $x$ and $y$.
\item[A2:] The Associative Law of addition: $(x+y) + z = x + (y+z)$ for any three numbers $x$, $y$ and $z$.
\item[A3:] The existence of a zero. There is a number, called the zero and denoted by $0$, which has the property that
$$ x + 0 = x \hbox{~and~} 0 + x = x \hbox{~for all numbers~} x.$$
\item[A4:] The existence of negatives: Corresponding to each number $x$ there is a number called the negative of $x$ and written $-x$ which satisfies
$$x + -x = 0\hbox{~and~} -x + x = 0.$$
\end{itemize}
\end{definition}

\begin{definition}{\bf Distributive Law}
\begin{itemize}
\item[D1:] The Distributive Law:
\begin{eqnarray*}
(x+y)z &=& xz + yz\\
x(y+z) &=& xy + xz
\end{eqnarray*}
\end{itemize}
\end{definition}

%\clearpage
\begin{definition}{\bf The laws of multiplication}:\\
\begin{itemize}
\item[M1:] The Commutative Law of multiplication : $x y = y x$ for all pairs of numbers $x$ and $y$.
\item[M2:] The Associative Law of multiplication: $(xy)z = x (yz)$ for any three numbers $x$, $y$ and $z$.
\item[M3:] The existence of a unity. There is a number, called the unity and denoted by $1$, which has the property that
$$ x1= x \hbox{~and~} 1x = x \hbox{~for all numbers~} x.$$
\item[M4:] The existence of inverses: Corresponding to each number $x$ there is a number called the inverse of $x$ and written $x^{-1}$ which satisfies
$$xx^{-1} = 1\hbox{~and~} x^{-1}x = 1.$$
\end{itemize}
\end{definition}

\section{Mappings}
\begin{definition}
An injective function $f: X\rightarrow Y$ (also known as injection, or one-to-one function) is a function $f$ that maps distinct elements of its domain to distinct elements; that is, $f(x_1) = f(x_2$) implies $x_1 = x_2$. 
Equivalently, $x_1 \ne x_2$ implies $f(x_1) \ne f(x_2)$ in the equivalent contrapositive statement.
\end{definition}

\begin{definition}
A surjective function  $f: X\rightarrow Y$ (also known as surjection, or onto function) is a function $f$ that every element $y$ can be mapped from some element $x$ so that $f(x) = y$. 
In other words, every element of the function's codomain is the image of at least one element of its domain. It is not required that $x$ be unique; the function $f$ 
may map one or more elements of the set $X$ to the same element of the set $Y$.
\end{definition}

\begin{definition}{\bf Homorphisms and Mappings}:\\
Monomorphism: 1-1, Epimorphism: onto, Isomorphism: 1-1 and onto.\\ Homomorphisms preserve structure (i.e. multiplication and addition in Groups and Rings). \\
Homomorphisms that have the same object and image space are called Endomorphisms.\\ And Endomorphism that is also an Isomorphism is called and Automorphism.\\
In a Group G, the mappings of G into itself, given by $g\rightarrow x^{-1} g x$, where $x$ is any fixed element of G, is an Automorphism, known as an Inner Automorphism.\\ 
\end{definition}

Suppose $f: A \rightarrow B$ and $g: B\rightarrow C$. Hence $(g\circ f) : A \rightarrow C$ exists. Show\\
(a) If $f$ and $g$ are 1-1 (monomorphisms), then $g\circ f$ is 1-1:\\
~~~Suppose $(g\circ f)(x) = (g\circ f)(y)$. Then $g(f(x)) = g(f(y))$. Because $g$ is 1-1, $f(x) = f(y)$ and because $f$ is 1-1, $x = y$. Therefore $(g\circ f)(x) = (g\circ f)(y)$, implies $x = y$; hence $g\circ f$ is 1-1. \\
(b) If $f$ and $g$ are onto (epimorphisms), then $g\circ f$ is onto.\\
~~~Suppose $c \in C$. Because $g$ is onto, there exists $b \in B$ for which $g(b) = c$. Because $f$ is onto, there exists $a\in A$ for which $f(a) = b$. Thus  $(g\circ f)(a) = g(f(a)) = g(b) = c$, Hence $g\circ f$ is onto. \\ \\
(c) If $g\circ f$ is 1-1, then $f$ is 1-1.\\ 
~~~Prove contrapositive, Suppose $f$ is  not 1-1., Then there exists distinct elements $x,y \in A$ for which $f(x) = f(y)$. Then $(g\circ f)(x) = g(f(x)) = g(f(y)) = (g\circ f)(y)$. Hence $g\circ f$ is not 1-1.Therefore if $g\circ f$ is 1-1, then $f$ must be 1-1.\\
(d) If $g\circ f$ is onto, then $g$ is onto.\\
~~~Prove contrapositive, If $a \in A$, then $(g\circ f)(a) = g(f(a)) \in g(B)$. Hence $(g\circ f)(A) \subseteq g(B)$. 
Suppose $g$ is not onto. Then $g(B)$ is properly contained in $C$ and so $(g\circ f)(A)$ is properly contained in C; thus $g\circ f$ is not onto. Therefore if $g\circ f$ is onto, then $g$ must be onto.\\

\section{Linear Dependence}
\begin{definition}
Vectors are directed segments. The vector in $\R_n$ whose components are $a_1, a_2,...,a_n$ is denoted by $$\a = \{a_1, a_2, ..., a_n\}\hbox{~(braces)}.$$
The zero vector in $\R_n$ is denoted as $$\o = \{0,0,...,0\},$$ with $n$ components each 0. 
\end{definition}

\begin{definition}We call $p$ vectors $\a_1, \a_2, ..., \a_p$ {\bf linearly dependent} if there are $p$ numbers $\lambda_1, \lambda_2,...,\lambda_p$ not all $0$, such that 
$$\lambda_1 \a_1 + \lambda_2 \a_2 +...+ \lambda_p \a_p = \o\hbox{~ (the zero vector)}.$$
\end{definition}

If no such $p$ numbers exist, then the $p$ vectors $\a_1, \a_2, ...,\a_p$ are called {\bf linearly independent}.

\begin{theorem} 
It a subset of the $p$ vectors $\a_1, \a_2,...,\a_p$ is linearly dependent, then the set of $p$ vectors is itself linearly dependent.\label{Th1_1}
\end{theorem}

Suppose the first $r$ vectors $\a_1, \a_2, ..., \a_r, (r < p) $are linearly dependent. Then there are $r$ numbers $\lambda_1, \lambda_2, ..., \lambda_r$ not all zero such that 
$$\lambda_1 \a_1 + \lambda_2 \a_2 + ... + \lambda_r \a_r = \o.$$ Setting $\lambda_{r+1}, \lambda_{r+2}, ..., \lambda_p$ to zero we have
$$\lambda_1 \a_1 + \lambda_2 \a_2 + ... + \lambda_r \a_r + \lambda_{r+1}\a_{r+1} +... + \lambda_p \a_p = \o.$$ where the numbers $\lambda_1, \lambda_2, ..., \lambda_p$ are not all zero. Therefore the 
$p$ vectors $\a_1, \a_2, ..., \a_p$ are linearly dependent. 

\begin{theorem}
If the $p$ vectors $\a_1, \a_2, ..., \a_p$ are linearly independent, then so is every subset of these $p$ vectors. \label{Th1_2}
\end{theorem}

For otherwise, by Theorem \ref{Th1_1}, the $p$ vectors would be linearly dependent.

\begin{theorem}
If the $p$ vectors $\a_1, \a_2, ..., \a_p$ are linearly dependent and if $p>1$, then at least one of these vectors
is a linear combination of the others.\label{Th1_3}
\end{theorem}

For, there are numbers $\lambda_i$, which do not all vanish such that $$\lambda_1 \a_1 + \lambda_2 \a_2 +...+ \lambda_p \a_p = \o.$$ Suppose $\lambda_p \ne 0$. The we can solve for $\a_p$,
$$\a_p = -{\lambda_1\over \lambda_p}\a_1 -{\lambda_2\over \lambda_p}\a_2 - ... - {\lambda_{p-1}\over \lambda_p} \a_{p-1},$$ i.e. $\a_p$ is a linear combination of the vectors $\a_1, \a_2, ..., \a_{p-1}$.  

\begin{theorem}
If one of the vectors $\a_1, \a_2, ..., \a_p$ is a linear combination of the others, then vectors $\a_1,\a_2,...,\a_p$ are linearly dependent.\label{Th1_4}
\end{theorem}

Suppose $$\a_1 = \lambda_2 \a_2 + \lambda_3 \a_3 +... + \lambda_p \a_p.$$ Then we have that 
$$\a_1 - \lambda_2 \a_2 - \lambda_3 \a_3 -... - \lambda_p \a_p = \o,$$ and since the coefficient of $\a_1$ is not zero, the vectors $\a_1, \a_2, ... , \a_p$ are linearly dependent. 

\begin{theorem}
If the vectors $\a_1, \a_2, ..., \a_p$ are linearly independent, and if the vectors $\a_1, \a_2,...\a_p,\b$ are linearly dependent, then $\b$ is a linear
combination of $\a_1, \a_2, ..., \a_p$.\label{Th1_5}
\end{theorem}

We have a relation of the form $$\lambda_1 \a_1 + \lambda_2 \a_2 +... + \lambda_p \a_p + \lambda_{p+1} \b =\o,$$ where not all the $\lambda_i$ vanish. Now
$\lambda_{p+1}$ cannot be 0 because the last term would drop out and then all the other $\lambda_i$ would vanish because of the linear independence of the $\a_i$ vectors. Therefore
$\lambda_{p+1} \ne 0$ and so we have 
$$\b = -{\lambda_1\over \lambda_{p+1}}\a_1 -{\lambda_2\over \lambda_{p+1}}\a_2 - ... - {\lambda_p\over \lambda_{p+1}} \a_p.$$


\begin{theorem}
If the vectors $\a_1, \a_2, ..., \a_p$ are linearly independent, and if $\b$ is a not a linear combination of $\a_1, \a_2, ..., \a_p$, then the vectors
$\a_1, \a_2,...\a_p,\b$ are linearly independent.\label{Th1_6}
\end{theorem}

Theorem \ref{Th1_6} follows from Theorem \ref{Th1_5}.

\section{Vector Spaces in $\R_n$}

Let $L$ be a {\elevenit non-empty} set of vectors of $\R_n$ with the properties:
\begin{itemize}
\item[1.] If $\a$ is a vector of the set $L$, then $\lambda \a$ belongs to $L$ for every real $\lambda$.
\item[2.] If $\a$ and $\b$ are two (not necessarily distinct) vectors of the set $L$, then the vector $\a +\b$ also belongs to $L$.
\end{itemize}
Such a set of vectors, satisfying 1. and 2., is  called a {\bf vector space} in $\R_n$.

\begin{theorem} {\bf Steinitz replacement theorem}. Let $\a_1, \a_2, ..., \a_p$ be any $p$ vectors, and let $L$ be the vector space spanned by them. Let $\b_1, \b_2, ..., \b_q$ be any $q$ linearly independent vectors of 
$L$. Then we may replace a certain set of $q$ vectors from among the vectors $\a_1, \a_2, ..., \a_p$ by the vectors $\b_1, \b_2, ...,\b_q$ so that the remaining vectors of $\a_1, \a_2, ..., \a_p$ together with the vectors $\b_1, \b_2, ..., \b_q$ span the entire vector space $L$.\label{Th2_1}
\end{theorem}

\begin{proof}
Use mathematical induction.
\end{proof}

\begin{theorem}
Any set of more than $p$ linear combinations of $p$ given vectors is always linearly dependent.\label{Th2_2}
\end{theorem}

\begin{theorem}
The maximal number of linearly independent vectors in $\R_n$ is $n$\label{Th2_3}
\end{theorem}

\begin{definition}{\bf Dimension and Basis}\\
Let $L$ be any vector space in $\R_n$. The maximal number of linearly independent vectors is $\le n$. Let $p$ be this maximal number of independent vectors in $L$. We call $p$ the {\bf dimension} of $L$. Thus,
$$0 \le p \le n.$$

We call any system of $p$ linearly independent vectors of $L$, such as $\a_1, \a_2, ..., \a_p$ a {\bf basis} of $L$.
\end{definition}

\begin{theorem}
Every vector of $L$ can be represented in exactly one way as a linear combination of the basis vectors $\a_1, \a_2, ..., \a_p$.\label{Th2_4}
\end{theorem}

\begin{theorem}
The dimension of the vector space spanned by the vectors $\a_1, \a_2, ...,\a_p$ is equal to the maximal number of linearly independent vectors among $\a_1, \a_2, ..., \a_p$.\label{Th2_5}
\end{theorem}

\begin{theorem}
Any $k ~ (k\le p)$ linearly independent vectors $$\b_1, \b_2, ..., \b_k$$
 of $L$ can be extended to form a basis of $L$ by suitably adjoining to them $p-k$ other vectors $\a_{k+1}, \a_{k+2}, ..., \a_p$.\label{Th2_6}
\end{theorem}

Let $L'$ and $L''$ be any two vector spaces in $\R_n$. By the {\bf intersection} $D$ of $L'$ and $L''$ we mean the set of all those vectors belonging to both $L'$ and $L''$.\\

If we form the totality $S$ of all vectors of the form $\a' + \a''$, where $\a'$ belongs to $L'$ and $\a''$ belongs to $L''$. $S$ is called the {\bf sum} of $L'$ and $L''$.

\begin{theorem}
If the dimensions of the vector spaces $L', L'', D,$ and $S$ are respectively $p', p'', d$ and $s$, then $$p' + p'' = d + s.$$\label{Th2_7}
\end{theorem}

\section{Linear Spaces}

A linear space $\L_p$ of dimension $p$ is the totality of points of $\R_n$ into which a fixed point $P$ is carried by the vectors of a $p$-dimensional vector space $L$ of $\R_n$. Suppose there are given $p+1$ points 
$$P_k = (x^1_k, x^2_k,...,x^n_k), ~k = 0, 1, ..., p$$ which do not lie an any linear space of dimension $<p$. Let us consider the vectors $\a_k = \overrightarrow{P_0 P_k}, k = 1, 2, ...,p$. These $p$ vectors are linearly independent.
For if they were linearly dependent, then the vector space spanned by them would have dimension $<p$. By applying these vectors to this vector space to $P_0$, we obtain a linear space which on the one hand has dimension $<p$, and on
the other contains the points $P_0, P_1, ..., P_p$, which contradicts our assumption.\\

All points of a linear space are equivalent to one another because any point of it can be used to obtain the entire linear space. 

\section{Linear Equations}

Let a system of $m$ linear equations in $n$ unknowns $x^1, x^2, ..., x^n$ be given in the following form:

%\begin{array*}
%  x+y+z = 1\\,
%  x+y+z = \frac{5}{2},\\
%  x+y+z = 5\\
%\end{array*}

\begin{equation}
\begin{array}{cccc}
a_{11}x^1  &+ ~a_{12}x^2   & + \hdots  &+  a_{1n}x^n   = b_1 \\
a_{21}x^1  &+ ~ a_{22}x^2   &+ \hdots  &+  a_{2n}x^n   = b_2 \\
 \cdots       &\cdots            &\cdots      &\cdots                \cdots \\
a_{i1}x^1   &+  ~a_{i2}x^2    & + \hdots  &+  a_{in}x^n    = b_i \\
 \cdots       &\cdots            &\cdots      &\cdots                 \cdots\\
a_{m1}x^1 &+  ~a_{m2}x^2 & + \hdots  & +  a_{mn}x^n = b_m \\
\end{array}
\label{LinearSystem}
\end{equation}

which can be written in the following form based on the rules of matrix multiplication

\begin{equation}
\begin{bmatrix}
a_{11} & a_{12}  & \hdots & a_{1n}  \\
a_{21} & a_{22}  & \hdots & a_{2n}  \\
\hdotsfor[2]{4}\\
a_{i1} & a_{i2}  & \hdots & a_{in}  \\
\hdotsfor[2]{4}\\
a_{m1} & a_{m2}  & \hdots & a_{mn}  \\
\end{bmatrix}
\begin{bmatrix}
x^1\\
x^2\\
.\\
x^i\\
.\\
x^n\\
\end{bmatrix}
=
\begin{bmatrix}
b_1\\
b_2\\
.\\
b_i\\
.\\
b_m\\
\end{bmatrix}
\label{MatrixSystem}
\end{equation}

To every column of the matrix of coefficients we can associate a vector
$$\a_k = \{a_{1k}, a_{2k},...,a_{mk}\}, ~k = 1,2,...,n.$$
The original system of equations can be expressed as a single vector equation
\begin{equation}\a_1 x^2 + \a_2 x^2 + ... + \a_n x^n = \b,\label{VectorSystem}\end{equation} where $\b = \{b_1, b_2, ..., b_m\}$. \\

Consider the vector space $\L$ spanned by the vectors $\a_k, k = 1,...,n$ in $\R_m$. If Eq.(\ref{VectorSystem}) is solvable for $x^1, x^2, ..., x^n$, then $\b$ must be a linear combination of the $\a_k$ and therefore belongs to $\L$. Conversely if $\b$ belongs to $\L$ then Eq.(\ref{VectorSystem}) is solvable. \\

Consider the vector space $\L'$ spanned by the vectors $\a_1, \a_1, ..., \a_n, \b$. A vector $\c$ in $\L'$ is of the form 
$$\c = \lambda_1 \a_1 + \lambda_2 \a_2 + ... + \lambda_n \a_n + \lambda \b.$$ If $\b$ belongs to $\L$, then Eq. (\ref{VectorSystem}) is satisfied by certain numbers $x^1, x^2, ..., x^n$ which can be substituted for 
$\b$ and so $\c$ is a linear combination of the $\a_k, k = 1,..., n$ and therefore belongs to $\L$. In this case $\L' \subseteq \L$ and also $\L \subseteq \L'$ and therefore $\L'$ and $\L$ are identical. 

\begin{theorem}
The equation (\ref{VectorSystem}) is solvable for the $x^i$, if and only if, the maximal number of linearly independent vectors among $\a_1, \a_2, ..., \a_n$ is the same as the maximal number of linearly independent 
vectors among $\a_1, \a_2, ..., \a_n, \b$.
\end{theorem}

\begin{definition}
The {\bf rank} of a matrix is equal to the maximal number of independent column vectors in the matrix. \\ 
\end{definition}

Consider the following matrix

\[
\begin{bmatrix}
a_{11} & a_{12}  & \hdots & a_{1n}  & b_1\\
a_{21} & a_{22}  & \hdots & a_{2n}  & b_2\\
\hdotsfor[2]{5}\\
a_{i1} & a_{i2}  & \hdots & a_{in}  & b_i \\
\hdotsfor[2]{5}\\
a_{m1} & a_{m2}  & \hdots & a_{mn}  & b_m\\
\end{bmatrix}
\]

obtained by appending the vector $\{b_1, b_2, ..., b_m\}$ to the final column. This matrix is called the {\bf augmented} matrix of the system of equations (\ref{LinearSystem}).

\begin{theorem}
Eqs. (\ref{LinearSystem}) are solvable for the $x^i$ if and only if the rank of the coefficient matrix of the system (\ref{LinearSystem}) is equal to the rank of the augmented matrix
\end{theorem}

The following system of equations is called a {\bf homogenous} {\elevenit in the $x^i$}
\begin{equation}
\begin{array}{cccc}
a_{11}x^1  &+ ~a_{12}x^2   & + \hdots  &+  a_{1n}x^n   = 0 \\
a_{21}x^1  &+ ~ a_{22}x^2   &+ \hdots  &+  a_{2n}x^n   = 0 \\
 \cdots       &\cdots            &\cdots      &\cdots                \cdots \\
a_{i1}x^1   &+  ~a_{i2}x^2    & + \hdots  &+  a_{in}x^n    = 0 \\
 \cdots       &\cdots            &\cdots      &\cdots                 \cdots\\
a_{m1}x^1 &+  ~a_{m2}x^2 & + \hdots  & +  a_{mn}x^n = 0 \\
\end{array}
\label{HomogenousSystem}
\end{equation}

Any set of numbers $x^1, x^2, ..., x^n$ which satisfy Eq. (\ref{HomogenousSystem}) are now taken to be the components of an $n$-dimensional vector $\mathfrak{r} = \{x^1, x^2, ..., x^n\}$. 
We call such a vector a {\elevenit vector solution} of the system (\ref{HomogenousSystem}). The totality of vector solutions of the system of equations (\ref{HomogenousSystem}) forms a
vector space.

\begin{theorem}
If the coefficient matrix of the system of equations (\ref{HomogenousSystem}) has rank $r$, then the set of vector solutions of this system is an $(n-r)$-dimensional vector space. 
\end{theorem}

\begin{theorem}
The system of equations (\ref{HomogenousSystem}) has a non-trivial solution, i.e. a solution $x^1, x^2, ..., x^n$ such that not all the $x^i$ vanish, if and only if, the rank of the matrix of 
(\ref{HomogenousSystem}) is less than $n$.
\end{theorem}

\begin{theorem}
If the number of equations in the system of homogenous equations (\ref{HomogenousSystem}) is less than the number of unknowns, then the system must have non-trivial solutions. 
\end{theorem}

\begin{theorem}
All the solutions of a non-homogenous system of equations (\ref{LinearSystem}) are of the form $z = \mathfrak{r} + \eta$, where $\mathfrak{r}$ is a fixed solution of the non-homogenous system (\ref{LinearSystem})
and $\eta$ runs though all solutions of the corresponding homogenous system (\ref{HomogenousSystem}). 
\end{theorem}

Let $s$ be the maximal number of linearly independent rows of the matrix row vectors. Let us assume that the rows of the homogenous system are ordered so that the first $s$ rows are independent to 
start with. This involves no loss of generality since it does not affect that rank. The $i$-th equation of this system is, by definition of the scalar product of two vectors, equivalent to the vector equation
$$ \a_i \cdot \mathfrak{r} = 0,$$ where we set $\mathfrak{r} = \{x^1, x^2, ..., x^n\}$.

Every system of $n$ numbers $x^1, x^2, ..., x^n$ satisfying the first $s$ equation of (the re-ordered) (\ref{HomogenousSystem}), satisfies all of the equations of (\ref{HomogenousSystem}). Because
the first $s$-rows are linearly independent, all the rows satisfy a relation of the form
$$\a_k = \lambda^k_1 \a_1 + \lambda^k_2 \a_2 + ...  + \lambda^k_s \a_s, \hbox{~for~} 1\le k \le m.$$ If we apply the distributive law on the right, we obtain
$$\a_k \cdot \mathfrak{r} =   \lambda^k_1 (\a_1\cdot \mathfrak{r}) + \lambda^k_2 (\a_2 \cdot \mathfrak{r})+ ...  + \lambda^k_s (\a_s\cdot \mathfrak{r}).$$ Since $\a_i \cdot \mathfrak{r} = 0$ for $i = 1,2, ..., s$ the right hand side 
of this last equation becomes $0$, so that $$\a_k \cdot \mathfrak{r} = 0, \quad k = 1,2, ..., m,$$ and in particular for $k = s+1, s+2, ..., m$ as was to be proved.\\

The vector space of all the solutions of the first $s$ equations of (\ref{HomogenousSystem}) is thus identical with the space of solutions of the entire system (\ref{HomogenousSystem}). It dimension is this $n-r$
since $r$ is the rank of the matrix of coefficients in (\ref{MatrixSystem}). It thus follows that the rank of the matrix of the first $s$ equations, i.e of the matrix which consists only of the first $s$ rows of (\ref{HomogenousSystem})
is equal to $n - (n-r) = r$. But the column vectors of this matrix are vectors with $s$ components. Thus the maximal number of linearly independent column vectors of this matrix is $\le s$ since by previous theorem
any $s+1$ vectors of $s$-dimensional vector space are linearly dependent. \\

Therefore the maximal number of linearly independent column vectors of a matrix is at most as large as the maximal number of linearly independent row vectors. \\

Now consider the transpose of the matrix of coefficients of (\ref{HomogenousSystem}). The maximal number of independent column vectors of this matrix is $s$, and the maximal number of row vectors is $r$.
Applying the previous result we also obtain $s\le r$. Hence we have $r \le s$ and $s\le r$ and therefore s = r. Thus we have proved

\begin{theorem}
The maximal number of linearly independent column vectors of a matrix is equal to the maximal number of linearly independent row vectors of that matrix. 
\end{theorem}

\begin{theorem}
With any given vector space $L$ of $\R_n$, we can always associate a system of homogenous linear equations in $n$ unknowns such that all the vectors of $L$, and no others, are vector solutions of this system.
\end{theorem}\

Let $\a_1, \a_2, ..., \a_p$ be a basis of $L$, where we set $\a_i = \{\a_{i1}, \a_{i2}, ..., \a_{in}\}$. If in addition we set 
$$\mathfrak{r} = \{ x^1, x^2, ..., x^n\}$$ then the equations 

\begin{equation}\a_i \cdot \mathfrak{r} = 0, \quad i = 1, 2, ..., p,\label{10}\end{equation}

form a system of homogenous linear equations in the unknowns $x^1, x^2, ..., x^n$ whose matrix is

\[
\begin{bmatrix}
a_{11} & a_{12}  & \hdots & a_{1n} \\
a_{21} & a_{22}  & \hdots & a_{2n} \\
\hdotsfor[2]{4}\\
a_{i1} & a_{i2}  & \hdots & a_{in}  \\
\hdotsfor[2]{4}\\
a_{p1} & a_{p2}  & \hdots & a_{pn}\\
\end{bmatrix}
\]

Let $p$ be the dimension of $L$.The rank of this system must be $n-p$. The totality of vectors which are orthogonal to $L$ is a vector space $L'$ of dimension $n-p$. It consists precisely of all the vector solutions of (\ref{10}).\\

We now seek a those vectors which are othogonal to the vector space $L'$ just found, We chose a basis $\b_1, \b_2, ..., \b_{n-p}$ of $L'$.The $\b_i$ as vectors of $L'$ are orthogonal to $L$, and this in particular to 
$\a_1, \a_2, ... \a_p$. Thus, for $i=1, 2, ..., p$, we have

$$\b_1 \cdot \a_i = 0, ~~\b_2 \cdot \a_2 = 0, ~~..., ~~\b_{n-p} \cdot \a_i = 0.$$

Now just as for $L$, the totality of all vectors orthogonal to $L'$ consists of all the vector solutions of the equations
\begin{equation}\b_1 \cdot \mathfrak{r} = 0, ~~\b_2 \cdot \mathfrak{r} = 0, ~~..., ~~\b_{n-p} \cdot \mathfrak{r} = 0.\label{11}\end{equation}
 
 Thus the $\a_i, i =1,2, ..., p$ are orthogonal to $L'$. But the vectors which are orthogonal to $L'$ form a vector space $L''$ of dimension $n - (n-p) = p$.Since the vectors $\a_1, \a_2, ..., \a_p$ are in $L''$ and are 
 linearly independent, then form a basis of $L''$. Therefore $L''$ is identical with $L$. Thus the vectors of $L$ are precisely the vector solutions of equations (\ref{11}), i.e, of a certain system of $n-p$ linear homogenous equations.
 
 \begin{theorem}
 The point solutions of a solvable system of equations of the type (\ref{LinearSystem}) form a linear space. This space has dimensions $n-r$m where $r$ is the rank of the coefficient matrix of the system. Conversely, every linear
 space may be represented as the totality of all point solutions of some suitable system of linear equations. 
 \end{theorem} 
 

\section{Vector Spaces in $\R_n$}

Let $L$ be a {\elevenit non-empty} set of vectors of $\R_n$ with the properties:
\begin{itemize}
\item[1.] If $\a$ is a vector of the set $L$, then $\lambda \a$ belongs to $L$ for every real $\lambda$.
\item[2.] If $\a$ and $\b$ are two (not necessarily distinct) vectors of the set $L$, then the vector $\a +\b$ also belongs to $L$.
\end{itemize}
Such a set of vectors, satisfying 1. and 2., is  called a {\bf vector space} in $\R_n$.

\begin{theorem} {\bf Steinitz replacement theorem}. Let $\a_1, \a_2, ..., \a_p$ be any $p$ vectors, and let $L$ be the vector space spanned by them. Let $\b_1, \b_2, ..., \b_q$ be any $q$ linearly independent vectors of 
$L$. Then we may replace a certain set of $q$ vectors from among the vectors $\a_1, \a_2, ..., \a_p$ by the vectors $\b_1, \b_2, ...,\b_q$ so that the remaining vectors of $\a_1, \a_2, ..., \a_p$ together with the vectors $\b_1, \b_2, ..., \b_q$ span the entire vector space $L$.\label{Th2_1}
\end{theorem}

\begin{proof}
Use mathematical induction.
\end{proof}

\begin{theorem}
Any set of more than $p$ linear combinations of $p$ given vectors is always linearly dependent.\label{Th2_2}
\end{theorem}

\begin{theorem}
The maximal number of linearly independent vectors in $\R_n$ is $n$\label{Th2_3}
\end{theorem}

\begin{definition}{\bf Dimension and Basis}\\
Let $L$ be any vector space in $\R_n$. The maximal number of linearly independent vectors is $\le n$. Let $p$ be this maximal number of independent vectors in $L$. We call $p$ the {\bf dimension} of $L$. Thus,
$$0 \le p \le n.$$

We call any system of $p$ linearly independent vectors of $L$, such as $\a_1, \a_2, ..., \a_p$ a {\bf basis} of $L$.
\end{definition}

\begin{theorem}
Every vector of $L$ can be represented in exactly one way as a linear combination of the basis vectors $\a_1, \a_2, ..., \a_p$.\label{Th2_4}
\end{theorem}

\begin{theorem}
The dimension of the vector space spanned by the vectors $\a_1, \a_2, ...,\a_p$ is equal to the maximal number of linearly independent vectors among $\a_1, \a_2, ..., \a_p$.\label{Th2_5}
\end{theorem}

\begin{theorem}
Any $k ~ (k\le p)$ linearly independent vectors $$\b_1, \b_2, ..., \b_k$$
 of $L$ can be extended to form a basis of $L$ by suitably adjoining to them $p-k$ other vectors $\a_{k+1}, \a_{k+2}, ..., \a_p$.\label{Th2_6}
\end{theorem}

Let $L'$ and $L''$ be any two vector spaces in $\R_n$. By the {\bf intersection} $D$ of $L'$ and $L''$ we mean the set of all those vectors belonging to both $L'$ and $L''$.\\

If we form the totality $S$ of all vectors of the form $\a' + \a''$, where $\a'$ belongs to $L'$ and $\a''$ belongs to $L''$. $S$ is called the {\bf sum} of $L'$ and $L''$.

\begin{theorem}
If the dimensions of the vector spaces $L', L'', D,$ and $S$ are respectively $p', p'', d$ and $s$, then $$p' + p'' = d + s.$$\label{Th2_7}
\end{theorem}
\section{Some Vector Theorems (3-dimensions)}
\begin{align*}
{\bf B} \times {\bf C} = 
\det\begin{bmatrix}
{\bf i} & {\bf j} & {\bf k}\\
B_x & B_y & B_z  \\
C_x& X_y & C_z \\
\end{bmatrix} = {\bf i}(B_y C_z - C_y B_z) + {\bf j}(B_z C_x - B_x C_z) + {\bf k}(B_xC_y-B_yC_x) 
\end{align*}

\begin{theorem}[BAC-CAB]
\[ {\bf a} \times ({\bf b} \times {\bf c}) = {\bf b}({\bf a}\cdot {\bf c}) - {\bf c}({\bf a}\cdot {\bf b}) \]
\[ ({\bf a} \times {\bf b}) \times {\bf c} = {\bf b}({\bf a}\cdot {\bf c}) - {\bf a}({\bf b}\cdot {\bf c}) \]
\end{theorem}

\begin{proof}
Given the expression for ${\bf B} \times {\bf C}$ we have 
%\begin{align*}
\begin{eqnarray*}
{\bf a} \times ({\bf b} \times {\bf c}) &=& 
\det\begin{bmatrix}
{\bf i} & {\bf j} & {\bf k}\\
a_x & a_y & a_z  \\
(b_yc_z - b_zc_y)&(b_zc_x - b_xc_z)  &  (b_xc_y-b_yc_x)\\
\end{bmatrix} \\ \\
&=& {\bf i} \big( a_y(b_xc_y-b_yc_x) - a_z(b_zc_x - b_xc_z) \big) + \\  
&\null & {\bf j} \big( a_z(b_yc_z - c_yb_z) - a_x(b_xc_y-b_yc_x) \big) + \\  
&\null & {\bf k} \big( a_x(b_zc_x - c_zb_x) - a_y(b_yc_z-b_zc_y) \big) \\  
\end{eqnarray*}
%\end{align*}
Prove component-by-component\cite{Spiegel}. The ${\bf i}^{th}$ component can be written as 
$$ b_x (a_yc_y + a_zc_z) - c_x(a_yb_y + a_zb_z)  = b_x (a_xc_x + a_yc_y + a_zc_z) - c_x(a_xb_x +a_yb_y + a_zb_z) = b_x( {\bf a}\cdot{\bf c}) - c_x({\bf a}\cdot{\bf b}).$$  
The ${\bf j}^{th}$ component can be written as 
$$ b_y (a_zc_z + a_xc_x) - c_y(a_zb_z + a_xb_x)  = b_y (a_xc_x + a_yc_y + a_zc_z) - c_y(a_xb_x +a_yb_y + a_zb_z) = b_y( {\bf a}\cdot{\bf c}) - c_y({\bf a}\cdot{\bf b}).$$  
The ${\bf k}^{th}$ component can be written as 
$$ b_z (a_xc_x + a_yc_y) - c_z(a_xb_x + a_yb_y)  = b_z (a_xc_x + a_yc_y + a_zc_z) - c_z(a_xb_x +a_yb_y + a_zb_z) = b_z( {\bf a}\cdot{\bf c}) - c_z({\bf a}\cdot{\bf b}).$$  
Putting back the respective unit vectors and adding we find the vector sum to be equal to
\[ {\bf b}({\bf a}\cdot {\bf c}) - {\bf c}({\bf a}\cdot {\bf b}).\] This demonstrates the proof that 
\[ {\bf a} \times ({\bf b} \times {\bf c}) = {\bf b}({\bf a}\cdot {\bf c}) - {\bf c}({\bf a}\cdot {\bf b}).\]
The second part of the theorem is proven in a similar manner:
\[ ({\bf a} \times {\bf b}) \times {\bf c}  = - {\bf c} \times ({\bf a} \times {\bf b}) = {\bf b}({\bf a}\cdot {\bf c}) - {\bf a}({\bf b}\cdot {\bf c}), \] using the first half of the proof along with the substitutions:
\[ {\bf a} \goes -{\bf c}\quad {\bf b} \goes {\bf a}\quad {\bf c} \goes {\bf b}\] viz.
\[ ({\bf a} \times {\bf b}) \times {\bf c}  = -{\bf c} \times ({\bf a} \times {\bf b})  \goes {\bf a} (-{\bf c}\cdot{\bf b}) - {\bf b} (-{\bf c}\cdot{\bf a}) = {\bf b}({\bf a}\cdot{\bf c}) -{\bf a}({\bf b}\cdot{\bf c}).\]\qed
\end{proof}

\begin{theorem}
\[ ({\bf a} \times {\bf b}) \cdot ({\bf c} \times {\bf d}) = ({\bf a}\cdot {\bf c})({\bf b}\cdot {\bf d}) - ({\bf a}\cdot {\bf d})({\bf b}\cdot {\bf c}) \]
\end{theorem}
\begin{proof}
\[ {\bf x}\cdot ({\bf c}\times{\bf d}) = ({\bf x}\times {\bf c})\cdot{\bf d}\] Let ${\bf x} = {\bf a}\times{\bf b}$. Then 
\begin{eqnarray*}
({\bf a} \times {\bf b}) \cdot ({\bf c} \times {\bf d}) &=&\{({\bf a} \times {\bf b})  \times {\bf c}\} \cdot {\bf d} \\
&=& \{{\bf b}({\bf a}\cdot {\bf c}) - {\bf a}({\bf b}\cdot {\bf c}) \}\cdot {\bf d} \\
&=& ({\bf a}\cdot {\bf c})({\bf b}\cdot {\bf d}) - ({\bf a}\cdot {\bf d})({\bf b}\cdot {\bf c}) \\
&=&\det\begin{bmatrix}
{\bf a}\cdot {\bf c} & {\bf a}\cdot {\bf d} \\
{\bf b}\cdot {\bf c} & {\bf b}\cdot {\bf d}
\end{bmatrix}
\end{eqnarray*} 
\end{proof}

\begin{theorem}
\[ ({\bf a} \times {\bf b}) \times ({\bf c} \times {\bf d}) = {\bf b}({\bf a}\cdot {\bf c}\times {\bf d}) - {\bf a}({\bf b}\cdot {\bf c}\times {\bf d}) =  {\bf c}({\bf a}\cdot {\bf b}\times {\bf d}) - {\bf d}({\bf a}\cdot {\bf b}\times {\bf c})\]
\end{theorem}
\begin{proof}
\[ {\bf x} \times ({\bf c} \times {\bf d}) = {\bf c}({\bf x}\cdot {\bf d}) - {\bf d}({\bf x}\cdot {\bf c}) \] Let ${\bf x} = {\bf a}\times {\bf b}$. Then 
\begin{eqnarray*}
{\bf I}: ~~~({\bf a} \times {\bf b}) \times ({\bf c} \times {\bf d}) &=& {\bf c}({\bf a}\times {\bf b}\cdot {\bf d}) - {\bf d}({\bf a}\times {\bf b}\cdot {\bf c}) \\ 
 &=& {\bf c}({\bf a}\cdot {\bf b}\times {\bf d}) - {\bf d}({\bf a}\cdot {\bf b}\times{\bf c})\qed
\end{eqnarray*}
Also 
\[ ({\bf a} \times {\bf b}) \times {\bf y}  = {\bf b}({\bf a}\cdot {\bf y}) - {\bf a}({\bf b}\cdot {\bf y}) \] Let ${\bf y} = {\bf c}\times {\bf d}$. Then 
\[ {\bf II}:~~~~ ({\bf a} \times {\bf b}) \times ({\bf c} \times {\bf d}) ={\bf b}({\bf a}\cdot {\bf c}\times{\bf d}) - {\bf a}({\bf b}\cdot {\bf c}\times{\bf d})~~~~\qed~~~\]
\end{proof}


\begin{theorem}[Compound Vector Scalar Product]
\[ ({\bf a} \times {\bf b})\cdot ({\bf b}\times {\bf c}) \times ({\bf c}\times {\bf a}) = ({\bf a}\cdot {\bf b} \times {\bf c})^2 \]
\end{theorem}

\begin{proof}
\[ {\bf x} \times ({\bf c}\times {\bf a}) = {\bf c}({\bf x}\cdot {\bf a}) - {\bf a}({\bf x}\cdot {\bf c}) \]
Let ${\bf x} = {\bf b} \times {\bf c}$, then 
\begin{eqnarray*}
({\bf b}\times {\bf c}) \times ({\bf c}\times {\bf a}) & =&  {\bf c}({\bf b}\times {\bf c} \cdot {\bf a}) - {\bf a}({\bf b}\times{\bf c}\cdot {\bf c}) \\
                                                                                   & =&  {\bf c}({\bf a}\cdot {\bf b} \times {\bf c}) - {\bf a}({\bf b}\cdot \times{\bf c}\times {\bf c}) \\
                                                                                   & =&  {\bf c}({\bf a}\cdot {\bf b} \times {\bf c}).
\end{eqnarray*}

Therefore 
\begin{eqnarray*}
({\bf a} \times {\bf b})\cdot ({\bf b}\times {\bf c}) \times ({\bf c}\times {\bf a}) & =&  ({\bf a} \times {\bf b})\cdot {\bf c}({\bf a}\cdot {\bf b} \times {\bf c})\\
                                                                                                                                & =&  ({\bf a} \times {\bf b}\cdot {\bf c})({\bf a}\cdot {\bf b} \times {\bf c})\\
                                                                                   & =&  ({\bf a}\cdot {\bf b} \times {\bf c})^2 \qed\\
\end{eqnarray*}
\end{proof}

\begin{definition}The operator $\nabla$ is given by
\[ \nabla = {\bf i}{\partial\over\partial x} + {\bf j}{\partial\over\partial y} + {\bf k}{\partial\over\partial z}.\]
\end{definition}

\begin{definition}
The Divergence of a vector field is given by
\[ \nabla\cdot{\bf a} = {\partial\over\partial x} a_x +{\partial\over\partial y} a_y +{\partial\over\partial z} a_z.\]
\end{definition}

\begin{definition}
The Laplacian (Divergence of the Gradient) operator is given by
\[ \nabla^2 = \nabla\cdot \nabla = {\partial^2\over\partial x^2} +{\partial^2\over\partial y^2} +{\partial^2\over\partial z^2}.\]
\end{definition}

\begin{definition}
The Curl of a vector field is given by
\begin{eqnarray*}
\nabla \times {\bf a} = 
\det\begin{bmatrix}
{\bf i} & {\bf j} & {\bf k}\\
\partial_x & \partial_y & \partial_z  \\
a_x& a_y & a_z \\
\end{bmatrix}
\end{eqnarray*}

\end{definition}

\begin{theorem}
\[ \nabla\cdot ({\bf a} \times {\bf b}) = {\bf b}\cdot (\nabla \times {\bf a}) - {\bf a} \cdot (\nabla \times {\bf b}). \]
\end{theorem}
\begin{proof}
\begin{eqnarray*}
\nabla\cdot({\bf a} \times {\bf b}) &=& 
\det\begin{bmatrix}
\partial_x & \partial_y & \partial_z\\
a_x & a_y & a_z  \\
b_x& b_y  & b_z  \\
\end{bmatrix}
\end{eqnarray*}
\begin{eqnarray*}
&=&[\partial_x (a_y b_z - a_z b_y)  - \partial_y(a_x b_z - a_z b_x) +  \partial_z(a_x b_y - a_y b_x) ]\\
&=&b_z \partial_x a_y - b_y \partial_x a_z  - b_z \partial_y a_x + b_x\partial_y a_z + b_y \partial_z a_x  - b_x\partial_z a_y  \\
&+&a_y \partial_x b_z - a_z\partial_x b_y  - a_x  \partial_y b_z + a_z \partial_y b_x + a_x \partial_z b_y  - a_y \partial_z b_x  \\ \\
&=& [b_x(\partial_y a_z  - \partial_z a_y) -  b_y (\partial_x a_z - \partial_z a_x ) +  b_z (\partial_x a_y  - \partial_y a_x)]    \\
&- & [a_x (\partial_y b_z - \partial_z b_y ) - a_y( \partial_x b_z - \partial_z b_x)  +  a_z(\partial_x b_y  - \partial_y b_x)]   \\
&=& {\bf b}\cdot (\nabla \times {\bf a}) - {\bf a}\cdot (\nabla \times {\bf b})\ \qed
\end{eqnarray*}
\end{proof}

\begin{theorem}
\[ \nabla\times( {\bf a}\times {\bf b}) = ({\bf b}\cdot \nabla){\bf a} - ({\bf a}\cdot \nabla){\bf b} + (\nabla\cdot{\bf b}) {\bf a} -  (\nabla\cdot{\bf a}) {\bf b}. \]  
\end{theorem}

\begin{theorem}
\[ \nabla\times( \nabla \times {\bf a}) = \nabla (\nabla \cdot {\bf a})  -  (\nabla\cdot \nabla) {\bf a} =  \nabla (\nabla \cdot {\bf a})  -  \nabla^2{\bf a}.\]  
\end{theorem}
\section{Reciprocal Vector Space (3-dimensions)}
Take three non-coplanar vectors ${\bf a}, {\bf b}, {\bf c}$ which form a right-handed system of vectors and form the vector scalar product whose magnitude is the volume, $V$, of the parallelapiped with edges given by ${\bf a}, {\bf b}, {\bf c}$ 
\[ V = {\bf a}\cdot{\bf b} \times {\bf c}. \] The set of vectors ${\bf a'}, {\bf b'}, {\bf c'}$ which are reciprocal to ${\bf a}, {\bf b}, {\bf c}$  is given by

\begin{eqnarray*}
{\bf a'}&=& { {\bf b} \times{\bf c}\over {\bf a}\cdot{\bf b} \times {\bf c}} \\   
{\bf b'}&=& { {\bf c} \times{\bf a}\over {\bf a}\cdot{\bf b} \times {\bf c}} \\
{\bf c'}&=& { {\bf a} \times{\bf b}\over {\bf a}\cdot{\bf b} \times {\bf c}} \\
\end{eqnarray*}

The reciprocal basis ${\bf a'}, {\bf b'}, {\bf c'}$ satisfy the following identities
\begin{eqnarray*}
{\bf a}\cdot {\bf a'}&=& 1, \quad {\bf b}\cdot {\bf a'} = 0, \quad {\bf c}\cdot {\bf a'} =0 \\   
{\bf b}\cdot {\bf b'}&=& 1, \quad {\bf a}\cdot {\bf b'} = 0, \quad {\bf c}\cdot {\bf b'} =0  \\
{\bf c}\cdot {\bf c'}&=& 1,  \quad {\bf a}\cdot {\bf c'} = 0, \quad {\bf b}\cdot {\bf c'} =0 \\
\end{eqnarray*}
 
\begin{theorem}[Reciprocal Basis has Reciprocal Volume = 1/V]
\[ {\bf a'}\cdot{\bf b'} \times {\bf c'} = {1\over V}  \]
\end{theorem}
\begin{proof}
\begin{eqnarray*}
{\bf a'}&=& { {\bf b} \times{\bf c}\over V} \\   
{\bf b'}&=& { {\bf c} \times{\bf a}\over V} \\
{\bf c'}&=& { {\bf a} \times{\bf b}\over V} \\
\end{eqnarray*}
\begin{eqnarray*}
{\bf a'}\cdot{\bf b'} \times {\bf c'}  &=& {({\bf b}\times {\bf c})\cdot({\bf c}\times{\bf a})\times({\bf a}\times {\bf b})\over V^3} = {({\bf a}\times {\bf b})\cdot ({\bf b}\times {\bf c})\times({\bf c}\times{\bf a}) \over V^3} \\
                                                       &=& {({\bf a}\cdot {\bf b} \times {\bf c})^2\over V^3} = {V^2\over V^3} = {1\over V}\qed
\end{eqnarray*}
\end{proof}
\section{Determinants and Their Properties}
Consider a set of vectors $V = \{ \a_1, \a_2, ...,\a_n\}$ in $\R^n$.
The Determinant $D(\a_1,\a_2,...,\a_n)$ has the following properties
\begin{itemize}
\item[i)] Invariance property: $D(\a_1,\a_2,...,\a_n)$ is to remain unchanged if some $\a_i$ is replaced by $\a_i+\a_j ~~(i \ne j)$;
\item[ii)] Homogeneity property: $D(\a_1,\a_2,...,\a_n)$ is to go over into $\lambda D(\a_1,\a_2,...,\a_n)$ if $\a_i$ is replaced by $\lambda \a_i$;
\item[iii)] the Normalization: $D(e_1, e_2,..., e_n) = 1$.
\end{itemize} The determinant is the unique function of vectors $\{ \a_1, \a_2, ...,\a_n\}$ that satisfy the above properties \cite{Schreier and Sperner}.\\

{\bf Conventions}: We adopt the Range and Summation Conventions:

\begin{definition}
{\bf Range Convention.} {\it When a small Latin suffix (superscript or subscript) occurs unrepeated in a term, it is understood to take all the values
$1,2,...,n$, where n is the number of dimensions of the space.} 
\end{definition} 

\begin{definition}
{\bf Summation Convention.} {\it When a small Latin suffix is repeated in a term, summation with respect to that suffix is understood, the range of summation
being $1,2,...,n$.}
\end{definition}

{\bf Kronecker delta}:
\[ \delta^s_r = \begin{cases} 1 & \mbox{if } s = r \\0 & \mbox{if } r \ne s\end{cases} \]
Also, the another form of the  Kronecker delta is $\delta_{rs}$ which satisfies the same cases as above. \\

{\bf Permutation Symbol}: Consider a set $\sigma$ of $\sigma_1, \sigma_2, ..., \sigma_n$ numbers taken from the set $\{1,2,...,n\}$. Let 
\[ \epsilon_{\sigma_1\sigma_2...\sigma_n}  = 
\begin{cases}
+1 & \mbox{if }\sigma  \mbox{ is an even permutation of $\{1, 2, ..., n\}$}\\ 
-1 & \mbox{if }\sigma  \mbox{ is an odd permutation of $\{1, 2, ..., n\}$}\\
0  & \mbox{if }\sigma  \mbox{ has any duplicates}
\end{cases} \]

Let the $S_n$ be the set of all permutation of the numbers $\{1,2,...,n\}$. In the case of $n$-by-$n$ matrix $A$ we obtain the complete 
development of the determinant of $A$ as follows: 
\begin{equation}
\det |A| = \sum_{\sigma \in S_{n}} \epsilon_{ \sigma_1 \sigma_2 ... \sigma_{n}} a_{1 \sigma_1} a_{2 \sigma_2} ... a_{n \sigma_n} \label{detA} = 
\begin{vmatrix}
a_{11} & a_{12}  & \hdots & a_{1n}  \\
a_{21} & a_{22}  & \hdots & a_{2n}  \\
\hdotsfor[2]{4}\\
a_{i1} & a_{i2}  & \hdots & a_{in}  \\
\hdotsfor[2]{4}\\
a_{n1} & a_{n2}  & \hdots & a_{nn}  \\
\end{vmatrix}
\end{equation}
If the rows are permuted by $\rho \in S_n$, we find
\begin{equation}
\epsilon_{ \rho_1 \rho_2 ... \rho_{n}}\det |A| = \sum_{\sigma \in S_{n}} \epsilon_{ \sigma_1 \sigma_2 ... \sigma_{n}} a_{\rho_1 \sigma_1} a_{\rho_2 \sigma_2} ... a_{\rho_n \sigma_n} \label{detA}
\end{equation}

{\bf Cofactor} of an element $a_{rs}$,

\begin{equation}
M _{rs}= \hbox{cofactor}(a_{rs}) =  \sum_{\sigma \in S_{n},~\sigma_{r} = s } \epsilon_{ \sigma_1 \sigma_2 ... \sigma_{n}} a_{1 \sigma_1}  a_{2 \sigma_2} ... a_{r-1 \sigma_{r-1}} a_{r+1 \sigma_{r+1}} ....a_{n \sigma_n} \label{detA}
\end{equation}

Relation between cofactors and minors obtained by taking the determinant obtained from $A$ by striking out row $r$ and column $s$ from $A$. We denote such minor determinants by $A_{rs}$ : 

$$M_{rs} = (-1)^{r+s} A_{rs}$$

For every $r$, $1\le r \le n$ we can write $\det |A |$ in terms of a cofactor expansion
\begin{equation}
\det |A| = \sum_{s=1}^n  a_{rs} \, M_{rs} = \sum_{s=1}^n\, (-1)^{r+s} \,a_{rs} \,A_{rs}  \label{cofactorExpansion}
\end{equation}

Also, if there is a mismatch between the row indices between the matrix element and the cofactor or minor, i.e., 
\begin{equation}
\sum_{s=1}^n  a_{ks} \, M_{rs} = \sum_{s=1}^n\, (-1)^{r+s} \,a_{ks} \,A_{rs}  = 0, \mbox{ for}\quad k\ne r  \label{cofactorExpansion}
\end{equation} The case $k\ne r$ corresponds to a determinant with two identical rows and hence vanishes because the rows are linearly dependent.  
Putting both of the above equations together we find
\begin{equation}
\sum_{s=1}^n  a_{ks} \, M_{rs} = \sum_{s=1}^n\, (-1)^{r+s} \,a_{ks} \,A_{rs} = \delta_{kr} \det |A|  \label{cofactorExpansion}
\end{equation}

{\bf Classical Adjoint and Inverse Matrix}
We define the Classical Adjoint, also known as the Adjugate, of the matrix $A$ as
\[ {\bf adj}(A)= M^\dagger, \] where $M^\dagger$ is the transpose of the cofactor matrix, $M$.
Using Eq.(\ref{cofactorExpansion}), the inverse of $A$ is given by
\[ A^{-1} = {1\over|A|}{\bf adj}(A).\label{inverseMatrix}\] Since $A^{-1} A = I$ we see that the determinant of $A^{-1}$ is $|A|^{-1}$. 
Therefore the determinant of the Classical Adjoint matrix is \[ |{\bf adj}(A)| = |A|^n | |A|^{-1} = |A|^{n-1},\] for an $n$-by-$n$ matrix $A$. \\

{\bf Left Inverse and Right Inverse}
Suppose we have a left inverse $X$ and a right inverse $Y$ of a matrix $A$. This means that 
\[ XA = AY = I.\] Therefore we have 
\begin{eqnarray*}
XAY &=& X, \quad\mbox{ and}\\
XAY &=& Y.
\end{eqnarray*}Therefore 
\[X = XAY = Y, \mbox{ or }  X=Y.\]  Therefore left and right inverses of a non-singular matrix must be equal.\\

{\bf The inverse of a Symmetric Matrix is Symmetric}
A symmetric matrix is defined by $A = A^\dagger$. The inverse of $A$ satisfies
\[ A^{-1} A = I \] Therefore, by taking the transpose of both sides we have
\[ A^T (A^{-1})^\dagger = A (A^{-1})^\dagger = I.\]Therefore 
\[ A^{-1} A (A^{-1})^\dagger = A^{-1} I = A^{-1},\] or 
\[(A^{-1})^\dagger = A^{-1}.\]Therefore the inverse of a symmetric matrix is also symmetric. \\

{\bf Special Cases -- Inverting a Triangular Matrix}
\begin{theorem}[Upper Triangular Matrices]
Let $U$ be an upper triangular matrix. If the inverse $U^{-1}$ exists, then it is upper triangular.
\end{theorem}
\begin{proof}
Recall the $(n-1)\times(n-1)$ cofactor matrix, $M_{rs}$ that results from omitting row $r$ and column $s$ of $U = (u_{ij})$. 
When it exists, \[ U^{-1} = {1\over|U|}{\bf adj}(U),\] so it is enough to prove that the $n \times n$ matrix $(|C_{rs}|)$ of cofactor determinants whose transpose
is $(|C_{rs}|)^\dagger$ is the adjugate, is lower triangular.\\

In the case $r < s$, every element below the diagonal of the matrix $C_{rs}$ is also below the diagonal of $U$, so must be equal to $0$.\\

Hence $C_{rs}$ is upper triangula, with the determinant equal to the product of its diagonal elements.\\

However, $s-r$ of these diagonal elements are $u_{i+1,j}$ for $i=r,\hdots,s-1$. These elements are from below the diagonal of $U$, and so are equal to $0$. \\

Hence $r <  s$ implies that $|C_{rs}| = 0$, so the $n\times n$ matrix $(|C_{rs}|)$ of cofactor determinants is indeed lower triangular. Hence the adjugate 
$(|C_{rs}|)^\dagger$ is upper triangular. Hence the inverse of and upper triangular matrix is upper triangular. By similar reasoning we can show that the 
inverse of a lower triangular matrix is lower triangula.\qed
\end{proof}

{\bf Reciprocal Vector Space in $n$-Dimensions}
Consider a set of $n$ vectors $\{ \a'_1, \a'_2, ...,\a'_n\}$, given by
\[ \a'_j = {1\over|A|}\{M_{j1},M_{j2},\hdots,M_{jn}\}.\] We will refer to this set of vectors as being reciprocal to the set of vectors forming the columns of $A$. 
From the definition of inverse of matrix $A$ we immediately have
\[ \a_k \cdot \a'_j = \delta_{kj}.\] 
Also the determinant of the $n$-by-$n$ matrix formed from the set of $\a'_j, j=1,2,...,n$ vectors arranged in columns
\begin{equation}
A' = 
\begin{bmatrix}
\a'_{1} & \a'_{2}  & \hdots & \a'_{n}  \\
\end{bmatrix}
\end{equation}had a determinant equal to 
\begin{equation}
\begin{vmatrix} 
A'
\end{vmatrix} = |A|^{-n}| \det(M) =  |A|^{-n}| |A|^{n-1} = |A|^{-1}.
\end{equation}Therefore the volume of the set of reciprocal vectors ${\rm volume}\{\a'_1, \a'_2, \hdots, \a'_n\} = \displaystyle{{1\over|A|} = {1\over V}}$ where $V =|A|$ is the volume of the set of the vectors
$\{\a_1, \a_2, \hdots, \a_n\}$. This is the $n$-dimensional Euclidean space version of Reciprocal Vectors in 3-dimensional Euclidean space.\\

{\bf Schur's complement}:
Suppose p, q are nonnegative integers, and suppose $A$, $B$, $C$, $D$ are respectively $p$-by-$p$, $p$-by-$q$, $q$-by-$p$, and $q$-by-$q$ matrices of complex numbers. Let
\begin{equation}
{\displaystyle M=\left[{\begin{matrix}A&B\\C&D\end{matrix}}\right]}
\end{equation} so that M is a ($p$ + $q$)-by-($p$ + $q$) matrix.\\

If $D$ is invertible, then the Schur complement of the block $D$ of the matrix $M$ is the $p$-by-$p$ matrix defined by
\begin{equation}
{\displaystyle M/D:=A - BD^{-1}C.}\label{M/D}
\end{equation}
and $$\det M = \det D \det (A - BD^{-1}C).$$

If $A$ is invertible, the Schur complement of the block $A$ of the matrix $M$ is the $q$-by-$q$ matrix defined by
\begin{equation}
{\displaystyle M/A:=D - CA^{-1}B.}\label{M/A}
\end{equation}
and $$\det M = \det A \det (D - CA^{-1}B).$$

In the case that $A$ or $D$ is singular, substituting a generalized inverse for the inverses on $M/A$ and $M/D$ yields the generalized Schur complement.\\

{\bf Laplace's General Expansion Theorem}:
Divide the set of integers ${1,2,...,n}$ into two complementary sets $\{\alpha_1, \alpha_2, ..., \alpha_p\}$ and $\{ \beta_1, \beta_2, ..., \beta_q\}$ where $n = p + q$ and
$$\alpha_1 < \alpha_2 < ... < \alpha_p \mbox{ and } \beta_1 < \beta_2 < ... < \beta_q.$$ These two sets remained fixed throughout the following:
\begin{equation}
\det A = \sum_{\substack{\rho_1, \rho_2, ...,\rho_p \\ \rho_1 < \rho_2 < ... < \rho_p}} \, (-1)^{(\alpha_1 + \alpha_2 + ... + \alpha_p + \rho_1 + \rho_2 ... + \rho_p)}
\begin{vmatrix}
a_{\alpha_1\rho_1} & a_{\alpha_1\rho_2}  & \hdots & a_{\alpha_1\rho_p}  \\
a_{\alpha_2\rho_1} & a_{\alpha_2\rho_2}  & \hdots & a_{\alpha_2\rho_p}  \\
\hdotsfor[2]{4}\\
a_{\alpha_i\rho_1} & a_{\alpha_i\rho_2}  & \hdots & a_{\alpha_i\rho_p}  \\
\hdotsfor[2]{4}\\
a_{\alpha_p\rho_1} & a_{\alpha_p\rho_2}  & \hdots & a_{\alpha_p\rho_p}  \\
\end{vmatrix}
\begin{vmatrix}
a_{\beta_1\sigma_1} & a_{\beta_1\sigma_2}  & \hdots & a_{\beta_1\sigma_q}  \\
a_{\beta_2\sigma_1} & a_{\beta_2\sigma_2}  & \hdots & a_{\beta_2\sigma_q}  \\
\hdotsfor[2]{4}\\
a_{\beta_i\sigma_1} & a_{\beta_i\sigma_2}  & \hdots & a_{\beta_i\sigma_q}  \\
\hdotsfor[2]{4}\\
a_{\beta_q\sigma_1} & a_{\beta_q\sigma_2}  & \hdots & a_{\beta_q\sigma_q}  \\
\end{vmatrix}
\end{equation}
 The sets $\{\alpha_1, \alpha_2, ..., \alpha_p\}$ and $\{ \beta_1, \beta_2, ..., \beta_q\}$  were defined to be fixed. 
 The summation is taken over all $p$-tuples $(\rho_1, \rho_2, ..., \rho_p)$ from among the numbers $1,2,...,n$ for which $\rho_1 < \rho_2 < ... < \rho_p$ and where 
 $$\sigma_1 < \sigma_2 < ... < \sigma_q$$ are the remaining $q$ integers.\\
 

\section{Cauchy-Binet Formula}

 Application 1: {\bf Example Partitioned Matrix and the Cauchy-Binet Formula}
 \begin{equation}
 M =
\left[
  \begin{array}{cccc|cccc}
  0 &  0 & \hdots & 0  & b_{11} & b_{12} & \hdots & b_{1n}\\
  0 & 0 & \hdots  & 0 & b_{21} & b_{22} & \hdots & b_{2n}\\
  \hdotsfor[2]{3} & \hdots &\hdotsfor[2]{3} & \hdots\\
  0 & 0 & \hdots  & 0 & b_{m1} & b_{m2} & \hdots & b_{mn}\\
  \hline
  c_{11} &  c_{12} &\hdots & c_{1m} & 1 & 0& \hdots & 0\\
  c_{21} &  c_{22} &\hdots & c_{2m} & 0 & 1& \hdots & 0\\
  \hdotsfor[2]{3} & \hdots & \hdotsfor[2]{3} & 0\\
  c_{n1} &  c_{n2} &\hdots & c_{nm} & 0 & 0& \hdots& 1 
\end{array} \right]
\end{equation}
In the above $M$ is an ($m+n$)-by-($m+n$) matrix which is partitioned into: $A$ an $m$-by-$m$ matrix of 0's, $B$ an $m$-by-$n$ matrix, $C$ an $n$-by-$m$ matrix and $D$ is the $n$-by-$n$ Identity matrix. Then using the determinant in the Schur's Complement form we have
$$\det M = \det D \det(A - BD^{-1}C) = \det (-BC) = (-1)^m \det(BC).$$\\

We can obtain a similar result by row operations. Denote the ($m+i)$ row by $\a_i$.  Adding a multiple of row $i$ to row $j$ does not change the value of $\det M$.
We successively subtract the $\sum_{i=1}^n b_{k\, i} \a_i$ from the $k$-th row of M, for $k=1, 2, ..., m$ changing M into a different matrix with the same determinant, viz.,
 \begin{equation}
 \tilde M=
\left[
  \begin{array}{cccc|cccc}
  -d_{11} &  -d_{12} & \hdots & -d_{1m}  & 0 & 0& \hdots & 0\\
  -d_{21} & -d_{22}  & \hdots  & -d_{2m}  & 0 & 0 & \hdots & 0\\
  \hdotsfor[2]{3} & \hdots &\hdotsfor[2]{3} & \hdots\\
  -d_{m1} & -d_{m2} & \hdots  & -d_{mm} & 0 & 0 & \hdots & 0\\
  \hline
  c_{11} &  c_{12} &\hdots & c_{1m} & 1 & 0& \hdots & 0\\
  c_{21} &  c_{22} &\hdots & c_{2m} & 0 & 1& \hdots & 0\\
  \hdotsfor[2]{3} & \hdots & \hdotsfor[2]{3} & 0\\
  c_{n1} &  c_{n2} &\hdots & c_{nm} & 0 & 0& \hdots& 1 
\end{array} \right],
\end{equation}where $\displaystyle{d_{kj} = \sum_{i=1}^n \, b_{ki} c_{ij}}, ~\mbox{ for } k,j = 1\,,...,\,m$. Since $\det M = \det \tilde M$ we have another demonstration of 
\begin{equation}
\det M = (-1)^m \det (BC)
\end{equation} where $B$ and $C$ are the matrices corresponding to $b_{kj}$ and $c_{ij}$ respectively.\\

Another representation of $\det M$ can be obtained by applying Laplace's Generalized Theorem successively. First swap the bottom $n$ rows with the top $m$ rows (by moving each of the bottom $n$ rows up $m$ positions and changing the sign of $\det M$ accordingly
\begin{equation}
\det M = (-1)^{m^2}
\det \left[
  \begin{array}{cccc|cccc}
  c_{11} &  c_{12} &\hdots & c_{1m} & 1 & 0& \hdots & 0\\
  c_{21} &  c_{22} &\hdots & c_{2m} & 0 & 1& \hdots & 0\\
  \hdotsfor[2]{3} & \hdots & \hdotsfor[2]{3} & 0\\
  c_{n1} &  c_{n2} &\hdots & c_{nm} & 0 & 0& \hdots& 1 \\
  \hline
  0 &  0 & \hdots & 0  & b_{11} & b_{12} & \hdots & b_{1n}\\
  0 & 0 & \hdots  & 0 & b_{21} & b_{22} & \hdots & b_{2n}\\
  \hdotsfor[2]{3} & \hdots &\hdotsfor[2]{3} & \hdots\\
  0 & 0 & \hdots  & 0 & b_{m1} & b_{m2} & \hdots & b_{mn}
\end{array} \right]
\end{equation}

Using Laplace's Generalized Theorem the first time across the bottom $m$ rows and taking advantage of the $m$-by-$m$ block of zeros in the lower left hand corner to drop some of the minors in the expansion we have only a sum over the $m$-by-$m$ minors in the $b_{ij}$ elements starting in columns $m+1$ through $m+n$ of $M$. After forming any one of these minors we have to select $m$ columns to remove to form the algebraic complement to use in the Laplace Theorem. The $n$-by-$n$ complementary minor consists of the $n$-by-$m$ $c_{ij}$ matrix elements with an additional $n-m$ columns which remain in from the former Identity matrix in the upper right hand side of the working matrix.\\

We now apply Laplace's Generalized Theorem a second time, now expanding over the first $m$ columns of the complementary $n$-by$n$ minor. We must again sum over the choice of $m$ rows from the first $n$ rows. The complementary minor to this second set of minors is obtained from the original $n$-by-$n$ Identity matrix, $I_n$, with $m$ columns removed by forming the first set of minors in the $b_{ij} $ and removing the $m$ rows to form the second set of minors. We are left with an $n-m$-by-$n-m$ matrix from $I_{n}$. If the rows that were struck out do not match exactly the columns that were struck out then there will appear in the the reduced matrix rows and columns with all zero entries and the corresponding determinant vanishes. So in the second application of Laplace's Generalized theorem there is only one term surviving in the sum, for any given choice of $m$ numbers 
$1 \le \rho_1 \le \rho_2 \le  ...\le \rho_m \le n$, the only term that survive is 
\begin{equation}
\begin{vmatrix}
c_{1\rho_1} & c_{1\rho_2}  & \hdots & c_{1\rho_m}  \\
c_{2\rho_1} & c_{2\rho_2}  & \hdots & c_{2\rho_m}  \\
\hdotsfor[2]{4}\\
c_{m\rho_1} & c_{m\rho_2}  & \hdots & c_{m\rho_m}  \\
\end{vmatrix}
\begin{vmatrix}
b_{1\rho_1} & b_{1\rho_2}  & \hdots & b_{1\rho_m}  \\
b_{2\rho_1} & b_{2\rho_2}  & \hdots & b_{2\rho_m}  \\
\hdotsfor[2]{4}\\
b_{m\rho_1} & b_{m\rho_2}  & \hdots & b_{m\rho_m}  \\
\end{vmatrix}
\end{equation}

Putting both results together and using the fact that both methods give an expression for $\det M$ we obtain the Cauchy-Binet Formula for matrices $B$ ($n$-by-$m$) and $C$ ($m$-by-$n$) with $m\le n$:
\begin{equation}
\det(BC) =\sum_{\substack{\rho_1, \rho_2, ...,\rho_m =1 \\ \rho_1 < \rho_2 < ... < \rho_m}}^n \,
\begin{vmatrix}
c_{1\rho_1} & c_{1\rho_2}  & \hdots & c_{1\rho_m}  \\
c_{2\rho_1} & c_{2\rho_2}  & \hdots & c_{2\rho_m}  \\
\hdotsfor[2]{4}\\
c_{m\rho_1} & c_{m\rho_2}  & \hdots & c_{m\rho_m}  \\
\end{vmatrix}
\begin{vmatrix}
b_{1\rho_1} & b_{1\rho_2}  & \hdots & b_{1\rho_m}  \\
b_{2\rho_1} & b_{2\rho_2}  & \hdots & b_{2\rho_m}  \\
\hdotsfor[2]{4}\\
b_{m\rho_1} & b_{m\rho_2}  & \hdots & b_{m\rho_m}  \\
\end{vmatrix}
\end{equation}\\
 
Application 2: {\bf Alternate Derivation of the Cauchy-Binet Formula}\\
 Let $A$ be an $m$-by-$n$ matrix and let $B$ be an $n$-by-$m$ matrix. Let $1\le j_1,j_2,...,j_m\le n$.
 Let $A_{j_1j_2...j_m}$ denote the $m$-by-$m$ matrix consisting of the $m$ columns $j_1,j_2,...,j_m$ of $A$.
 Let $B_{j_1j_2...j_m}$ denote the $m$-by-$m$ matrix consisting of the $m$ rows $j_1,j_2,...,j_m$ of $B$.
 Let $\{k_1,k_2,...,k_m\}$ be an ordered $m$-tuple of integers and let $\{j_1,j_2,...,j_m\}$ be the same set of integers but arranged into non-decreasing order:
 $$\j_1\le j_2 \le ... \le j_m.$$
 
 Then we have the relationships that 
 \begin{eqnarray*}
 \det \left(B_{k_1k_2...k_m}\right) &=& \epsilon_{k_1k_2...k_m} \det \left(B_{j_1j_2...j_m}\right),\\
 \det \left(B_{j_1j_2...j_m}  \right) &=& \epsilon_{k_1k_2...k_m}  \det \left(B_{k_1k_2...k_m}\right)\mbox{ no implied sum on } k_i's.
\end{eqnarray*}\\
 Now from the definition of determinant for the $m$-by-$m$ matrix $AB$ we have 
 \begin{eqnarray*}
 \det (AB) &=& \sum_{1\le l_1,...l_m \le m} \, \epsilon_{l_1l_2...l_m} \left( \sum_{k=1}^n a_{1k} b_{kl_1} \right) ... \left( \sum_{k=1}^n a_{mk}b_{kl_m}\right)  \\
                &=& \sum_{1\le k_1,...k_m \le n} \, \left(a_{1k_1}...a_{mk_m}\right) \sum_{1\le l_1,...l_m \le m} \, \epsilon_{l_1l_2...l_m} b_{k_1l_1}  ... b_{k_ml_m} \\
                &=& \sum_{1\le k_1,...k_m \le n} \, \left( a_{1k_1}...a_{mk_m} \right) \det \left(B_{k_1...k_m}\right)\\
                &=& \sum_{1\le k_1,...k_m \le n} \, \left(a_{1k_1}...a_{mk_m} \epsilon_{k_1...k_m}\right) \det \left(B_{j_1...j_m}\right)\\
                &=& \sum_{1\le k_1,...k_m \le n} \, \left(\epsilon_{k_1...k_m} \, a_{1k_1}...a_{mk_m}\right) \det \left(B_{j_1...j_m}\right)\\
                &=& \sum_{1\le j_1 \le j_2 \le ...\le j_m \le n} \det \left(A_{j_1...j_m}\right)\,\det \left(B_{j_1...j_m}\right) \\
 \end{eqnarray*}
 
 If any two $j$'s are equal then $\det  \left(A_{j_1...j_m}\right) = 0$\\
 
 In the case of $A$ and $B = A^T$ we immediately have
$$ \det (AA^T) = \sum_{1\le j_1 \le j_2 \le ...\le j_m \le n} \left[ \det \left(A_{j_1...j_m}\right)\right]^2 $$
 

%\[ 
%M(z)= 
%\begin{array}{c@{}c}
%\left[
% \begin{BMAT}[5pt]{c|c}{c}
%    \begin{BMAT}[5pt]{c:c:c}{ccccccc}
%     & & \\
%      & & \\
%      & & \\
%      v_1 & \dots & v_m \\
%      & & \\
%      & & \\
%      & &
%    \end{BMAT}
%    &
%    \begin{BMAT}{c}{c|c}
%      \begin{BMAT}[5pt]{c:c:c}{ccc}
%        & & \\
%        w_1(a) & \dots & w_n(a) \\
%        & &
%      \end{BMAT}
%      \\
%      \begin{BMAT}[10pt]{c}{c}
%        B(z)
%      \end{BMAT}
%    \end{BMAT}
%  \end{BMAT} 
%\right] 
%& 
%\begin{array}{l}
%  \\[-8mm] \rdelim\}{4}{6mm}[$J$] \\ \\ \\[4mm]  \rdelim\}{3}{6mm}[$H$] \\ \\
%\end{array} \\[-1ex]
%\hexbrace{2.7cm}{m}\hexbrace{3.8cm}{n}
`%\end{array}
%\]

\section{Vandermonde Determinant}

The Vandermonde Determinant for n variables $x = [x_1, x_2, \hdots, x_n]$ is given by 

$$V_n(x) =  
\begin{bmatrix}
1 & x_1 & x_1^2 &  x_1^3 & \hdots & x_1^n \\
1 & x_2 & x_2^2 &  x_2^3 &\hdots &x_2^n \\
1 &  x_3 & x_3^2 & x_3^3 &\hdots & x_2^n \\
\vdots \\
1 & x_n & x_n^2 & x_n^3 &\hdots & x_n^n \\
\end{bmatrix}
$$

Except for the first column, subtract $x_1$ times the previous column to the rest of the columns. Since a multiple of any column added to any other column does not change the determinant, we can write
$$\det{V_n(x)} =  
\det{\begin{bmatrix}
1 & 0 & 0 & 0 & \hdots & 0 \\
1 & x_2 - x_1 & x_2(x_2 - x_1) & x_2^2(x_2 - x_1) & \hdots & x_2^{n-1}(x_2-x_1) \\
1 &  x_3 -x_1 & x_3(x_3 - x_1) & x_3^2(x_3 - x_1) & \hdots & x_3^{n-1}(x_3-x_1) \\
\vdots \\
1 & x_n-x_1 & x_n(x_n - x_1) & x_n^2(x_n - x_1) & \hdots & x_n^{n-1}(x_n-x_1) \\
\end{bmatrix}}
$$

Expanding the determinant using the elements in the first row we have

$$\det{V_n(x)} =  
\det{\begin{bmatrix}
x_2 - x_1 & x_2(x_2 - x_1) & x_2^2(x_2 - x_1) & \hdots & x_2^{n-1}(x_2-x_1) \\
x_3 -x_1 & x_3(x_3 - x_1) & x_3^2(x_3 - x_1) & \hdots & x_3^{n-1}(x_3-x_1) \\
\vdots \\
x_n-x_1 & x_n(x_n - x_1) & x_n^2(x_n - x_1) & \hdots & x_n^{n-1}(x_n-x_1) \\
\end{bmatrix}}
$$

Factoring out $x_2-x_1$ from the first row, $x_3-x_1$ from the second row, and so on until the last row we have

$$\det{V_n(x_1,x_2,\hdots,x_n)} =  
[\prod_{i = 2}^n (x_i - x_1)] 
\det{\begin{bmatrix}
1 & x_2  & x_2^2 & \hdots & x_2^{n-1} \\
1 & x_3  & x_3^2 & \hdots & x_3^{n-1} \\
\vdots \\
1 & x_n & x_n^2 & \hdots & x_n^{n-1} \\
\end{bmatrix} }=
\prod_{i = 2}^n (x_i - x_1) \, \det{V_{n-1}(x_2,x_3,\hdots,x_n)}
$$
 
 Proceeding in a similar way by subtracting $x_2$ times the previous column except for the first column we can continue
$$\det{V_n(x_1,x_2,\hdots,x_n)} =  
[\prod_{i = 2}^n (x_i - x_1)] \,
[\prod_{i = 3}^n (x_i - x_2)] \, \det{V_{n-2}(x_3,x_4,\hdots,x_n)}
$$
To arrive at 
$$\det{V_n(x_1,x_2,\hdots,x_n)} =  \prod_{\substack{1 < j \\ i < j}}^n (x_j - x_i) =  \prod_{1 \le i < j \le  n } (x_j - x_i)
$$

\section{Cayley-Hamilton Theorem}

The characteristic equation for eigenvalues $0 = \det|{A - \lambda I |}$ can be expanded as a polynomial in $\lambda$ 
\[ \phi(\lambda) = \det{|A - \lambda I|} = a_n\lambda^n + a_{n-1}\lambda^{n-1} +\hdots + a_1 \lambda + a_0, \quad a_n = (-1)^n.\]
And the characteristic equation can be written in the form $\phi(\lambda) = 0$.\\

If in any polynomial $f(\lambda) = \sum c_i \lambda^i$ we substitute the matrix $A$ for $\lambda$ and multiply the constant term
$c_0$ by $I$, we obtain a matrix denoted by $f(A)$.

\begin{theorem} (Cayley-Hamilton) Any square matrix satisfies its characteristic equation. \end{theorem}

\begin{proof}
Let $\phi(\lambda)$ be the characteristic determinant of $A$. We want to show that $\phi(A) = 0$.  Since the elements of $A - \lambda I$ are 
linear functions of $\lambda$, and the elements of its adjoint [the transpose of the matrix of signed minors of $A-\lambda I$ -- denoted by Adj\_$(A-\lambda I)$] are (n-1)-rowed  determinants,
they are polynomials in $\lambda$ of degree $\le n-1$. Let $C$ stand for the Adj-$(A-\lambda I)$.
If the element in the i-th row and j-th column of $C$ is $\sum_k c_{ijk}\lambda^k$, then 

\[ C = \sum_{k=0}^{n-1} C_k \lambda^k, \quad C_k = (c_{ijk})\quad (i,j = 1, 2, \hdots, n) \]
 
 Using the relationship between the adjoint and the determinant, $\phi(\lambda) = \det{|A - \lambda I|}$, we can write
 
 \[ (A - \lambda I) C = \phi(\lambda) I. \] This expression can be expanded as 
 \[ A \sum_{k = 0} ^{n-1} C_k \lambda^k  - \lambda \sum_{k=0}^{n-1} C_k \lambda^k = \sum_{k=0}^n a_k \lambda^k  I \]
 
 Equating the terms free of $\lambda$ and the coefficients of $\lambda, \lambda^1, ..., \lambda^{n-1}, \lambda^n$ we get
 
 \begin{eqnarray*}
 AC_0                      &=& a_0 I,\\
 AC_1 - C_0            &=& a_1 I,\\
 AC_2 - C_1            &=& a_2 I,\\
 \hdots                     &~& \hdots \\
 AC_{n-1} - C_{n-2} &=& a_{n-1} I,\\
-C_{n-1}                  &=& a_n I.
\end{eqnarray*}

Multiply these equations on the left by $I, A, A^2, ..., A^{n-1}, A^n$ respectively and add; we get

\[ 0 = a_0 I + a_1 A + a_2 A^2 + \hdots + a_{n-1}A^{n-1} + a_n A^n = \phi(A), \] and we see that the matrix $A$ satisfies its own
characteristic equation.  \qed
\end{proof}

\section{Sylvesters Criterion for Positive Definite Hermitian Matrices}

\begin{theorem} (Sylvester's Criterion)  An $n$-x-$n$ Hermitian matrix M is positive-definite if and only if all the leading principal minors are positive. 
\end{theorem}

\begin{proof}
Let $M$ be an n x n Hermitian positive definite matrix. Then $0 \le x^\dagger M x$ if $x$ is not the zero vector. Let $M_k$ be the $k$-th leading principal minor 
(the first $k$-rows and $k$-columns in the upper left hand corner of $M$). Select a non-zero vector $x$ as follows 

\[ x = 
\begin{bmatrix}
x_1 \\
\vdots \\
x_k\\
0\\
\vdots\\
0
\end{bmatrix}
= 
\begin{bmatrix}
\vec{x}\\
0\\
\vdots\\
0
\end{bmatrix}
\]

Then we have that $0 \le x^\dagger M x = \vec{x}^\dagger M_k \vec{x}$. Therefore, since $\vec{x} \ne \vec{0}$ is otherwise arbitrary, then all the eigenvalues of $M_k$ must be
positive and hence  $\det|M_k| > 0 $

To prove the reverse implication, we use induction. The general form of an $(n+1)$-x-$(n+1)$ Hermitian matrix is

\[ M_{n+1} = 
\begin{bmatrix}
M_n & \vec{v}\\
\vec{v}^\dagger & d
\end{bmatrix}
\]

Denote the $(n+1)$ dimensional vector $x$ as 

\[ x = 
\begin{bmatrix}
\vec{x} \\
x_{n+1}
\end{bmatrix}
\]

Then 

\[ x^\dagger M_{n+1} x = \vec{x}^\dagger M_n \vec{x} + x_{n+1} \vec{x}^\dagger \vec{v} + \overline{x}_{n+1} \vec{v}^\dagger \vec{x} + d |x_{n+1}|^2 \]

By completing the square, this last expression is equal to 
\[ (\vec{x}^\dagger + \vec{v}^\dagger M_n^{-1} \overline{x}_{n+1}) M_n (\vec{x} +  x_{n+1}M_n^{-1} \vec{v} ) - |x_{n+1}|^2  \vec{v}^\dagger M_n^{-1} \vec{v} + d|x_{n+1}|^2 \]
\[ = (\vec{x} + \vec{c})^\dagger M_n (\vec{x} + \vec{c}) + |x_{n+1}|^2 (d - \vec{v}^\dagger M_n^{-1} \vec{v}), \] where $\vec{c} = x_{n+1} M_n^{-1}\vec{v}$. $M_n^{-1}$ exists because
the eigenvalues of $M_n$ are all positive. The first term is positive by the induction hypothesis.  Using this useful determinant expansion

\[ \det{ \begin{bmatrix}
A & B  \\
C & D \\
\end{bmatrix}} = \det{|A|} \det{ | D - C A^{-1}B | } \]

We have that 
\[ \det{|M_{n+1}|} = \det{|M_n|}(d - \vec{v}^\dagger M_n^{-1} \vec{v} )> 0,\]
which implies $(d - \vec{v}^\dagger M_n^{-1} \vec{v}) > 0$. 
Hence $x^\dagger M_{n+1} x > 0$ (is also positive definite). 
\qed
\end{proof}

\section{Linear Spaces}

A linear space $\L_p$ of dimension $p$ is the totality of points of $\R_n$ into which a fixed point $P$ is carried by the vectors of a $p$-dimensional vector space $L$ of $\R_n$. Suppose there are given $p+1$ points 
$$P_k = (x^1_k, x^2_k,...,x^n_k), ~k = 0, 1, ..., p$$ which do not lie an any linear space of dimension $<p$. Let us consider the vectors $\a_k = \overrightarrow{P_0 P_k}, k = 1, 2, ...,p$. These $p$ vectors are linearly independent.
For if they were linearly dependent, then the vector space spanned by them would have dimension $<p$. By applying these vectors to this vector space to $P_0$, we obtain a linear space which on the one hand has dimension $<p$, and on
the other contains the points $P_0, P_1, ..., P_p$, which contradicts our assumption.\\

All points of a linear space are equivalent to one another because any point of it can be used to obtain the entire linear space. 

\section{Linear Equations}

Let a system of $m$ linear equations in $n$ unknowns $x^1, x^2, ..., x^n$ be given in the following form:

%\begin{array*}
%  x+y+z = 1\\,
%  x+y+z = \frac{5}{2},\\
%  x+y+z = 5\\
%\end{array*}

\begin{equation}
\begin{array}{cccc}
a_{11}x^1  &+ ~a_{12}x^2   & + \hdots  &+  a_{1n}x^n   = b_1 \\
a_{21}x^1  &+ ~ a_{22}x^2   &+ \hdots  &+  a_{2n}x^n   = b_2 \\
 \cdots       &\cdots            &\cdots      &\cdots                \cdots \\
a_{i1}x^1   &+  ~a_{i2}x^2    & + \hdots  &+  a_{in}x^n    = b_i \\
 \cdots       &\cdots            &\cdots      &\cdots                 \cdots\\
a_{m1}x^1 &+  ~a_{m2}x^2 & + \hdots  & +  a_{mn}x^n = b_m \\
\end{array}
\label{LinearSystem}
\end{equation}

which can be written in the following form based on the rules of matrix multiplication

\begin{equation}
\begin{bmatrix}
a_{11} & a_{12}  & \hdots & a_{1n}  \\
a_{21} & a_{22}  & \hdots & a_{2n}  \\
\hdotsfor[2]{4}\\
a_{i1} & a_{i2}  & \hdots & a_{in}  \\
\hdotsfor[2]{4}\\
a_{m1} & a_{m2}  & \hdots & a_{mn}  \\
\end{bmatrix}
\begin{bmatrix}
x^1\\
x^2\\
.\\
x^i\\
.\\
x^n\\
\end{bmatrix}
=
\begin{bmatrix}
b_1\\
b_2\\
.\\
b_i\\
.\\
b_m\\
\end{bmatrix}
\label{MatrixSystem}
\end{equation}

To every column of the matrix of coefficients we can associate a vector
$$\a_k = \{a_{1k}, a_{2k},...,a_{mk}\}, ~k = 1,2,...,n.$$
The original system of equations can be expressed as a single vector equation
\begin{equation}\a_1 x^2 + \a_2 x^2 + ... + \a_n x^n = \b,\label{VectorSystem}\end{equation} where $\b = \{b_1, b_2, ..., b_m\}$. \\

Consider the vector space $\L$ spanned by the vectors $\a_k, k = 1,...,n$ in $\R_m$. If Eq.(\ref{VectorSystem}) is solvable for $x^1, x^2, ..., x^n$, then $\b$ must be a linear combination of the $\a_k$ and therefore belongs to $\L$. Conversely if $\b$ belongs to $\L$ then Eq.(\ref{VectorSystem}) is solvable. \\

Consider the vector space $\L'$ spanned by the vectors $\a_1, \a_1, ..., \a_n, \b$. A vector $\c$ in $\L'$ is of the form 
$$\c = \lambda_1 \a_1 + \lambda_2 \a_2 + ... + \lambda_n \a_n + \lambda \b.$$ If $\b$ belongs to $\L$, then Eq. (\ref{VectorSystem}) is satisfied by certain numbers $x^1, x^2, ..., x^n$ which can be substituted for 
$\b$ and so $\c$ is a linear combination of the $\a_k, k = 1,..., n$ and therefore belongs to $\L$. In this case $\L' \subseteq \L$ and also $\L \subseteq \L'$ and therefore $\L'$ and $\L$ are identical. 

\begin{theorem}
The equation (\ref{VectorSystem}) is solvable for the $x^i$, if and only if, the maximal number of linearly independent vectors among $\a_1, \a_2, ..., \a_n$ is the same as the maximal number of linearly independent 
vectors among $\a_1, \a_2, ..., \a_n, \b$.
\end{theorem}

\begin{definition}
The {\bf rank} of a matrix is equal to the maximal number of independent column vectors in the matrix. \\ 
\end{definition}

Consider the following matrix

\[
\begin{bmatrix}
a_{11} & a_{12}  & \hdots & a_{1n}  & b_1\\
a_{21} & a_{22}  & \hdots & a_{2n}  & b_2\\
\hdotsfor[2]{5}\\
a_{i1} & a_{i2}  & \hdots & a_{in}  & b_i \\
\hdotsfor[2]{5}\\
a_{m1} & a_{m2}  & \hdots & a_{mn}  & b_m\\
\end{bmatrix}
\]

obtained by appending the vector $\{b_1, b_2, ..., b_m\}$ to the final column. This matrix is called the {\bf augmented} matrix of the system of equations (\ref{LinearSystem}).

\begin{theorem}
Eqs. (\ref{LinearSystem}) are solvable for the $x^i$ if and only if the rank of the coefficient matrix of the system (\ref{LinearSystem}) is equal to the rank of the augmented matrix
\end{theorem}

The following system of equations is called a {\bf homogenous} {\elevenit in the $x^i$}
\begin{equation}
\begin{array}{cccc}
a_{11}x^1  &+ ~a_{12}x^2   & + \hdots  &+  a_{1n}x^n   = 0 \\
a_{21}x^1  &+ ~ a_{22}x^2   &+ \hdots  &+  a_{2n}x^n   = 0 \\
 \cdots       &\cdots            &\cdots      &\cdots                \cdots \\
a_{i1}x^1   &+  ~a_{i2}x^2    & + \hdots  &+  a_{in}x^n    = 0 \\
 \cdots       &\cdots            &\cdots      &\cdots                 \cdots\\
a_{m1}x^1 &+  ~a_{m2}x^2 & + \hdots  & +  a_{mn}x^n = 0 \\
\end{array}
\label{HomogenousSystem}
\end{equation}

Any set of numbers $x^1, x^2, ..., x^n$ which satisfy Eq. (\ref{HomogenousSystem}) are now taken to be the components of an $n$-dimensional vector $\mathfrak{r} = \{x^1, x^2, ..., x^n\}$. 
We call such a vector a {\elevenit vector solution} of the system (\ref{HomogenousSystem}). The totality of vector solutions of the system of equations (\ref{HomogenousSystem}) forms a
vector space.

\begin{theorem}
If the coefficient matrix of the system of equations (\ref{HomogenousSystem}) has rank $r$, then the set of vector solutions of this system is an $(n-r)$-dimensional vector space. 
\end{theorem}

\begin{theorem}
The system of equations (\ref{HomogenousSystem}) has a non-trivial solution, i.e. a solution $x^1, x^2, ..., x^n$ such that not all the $x^i$ vanish, if and only if, the rank of the matrix of 
(\ref{HomogenousSystem}) is less than $n$.
\end{theorem}

\begin{theorem}
If the number of equations in the system of homogenous equations (\ref{HomogenousSystem}) is less than the number of unknowns, then the system must have non-trivial solutions. 
\end{theorem}

\begin{theorem}
All the solutions of a non-homogenous system of equations (\ref{LinearSystem}) are of the form $z = \mathfrak{r} + \eta$, where $\mathfrak{r}$ is a fixed solution of the non-homogenous system (\ref{LinearSystem})
and $\eta$ runs though all solutions of the corresponding homogenous system (\ref{HomogenousSystem}). 
\end{theorem}

Let $s$ be the maximal number of linearly independent rows of the matrix row vectors. Let us assume that the rows of the homogenous system are ordered so that the first $s$ rows are independent to 
start with. This involves no loss of generality since it does not affect that rank. The $i$-th equation of this system is, by definition of the scalar product of two vectors, equivalent to the vector equation
$$ \a_i \cdot \mathfrak{r} = 0,$$ where we set $\mathfrak{r} = \{x^1, x^2, ..., x^n\}$.

Every system of $n$ numbers $x^1, x^2, ..., x^n$ satisfying the first $s$ equation of (the re-ordered) (\ref{HomogenousSystem}), satisfies all of the equations of (\ref{HomogenousSystem}). Because
the first $s$-rows are linearly independent, all the rows satisfy a relation of the form
$$\a_k = \lambda^k_1 \a_1 + \lambda^k_2 \a_2 + ...  + \lambda^k_s \a_s, \hbox{~for~} 1\le k \le m.$$ If we apply the distributive law on the right, we obtain
$$\a_k \cdot \mathfrak{r} =   \lambda^k_1 (\a_1\cdot \mathfrak{r}) + \lambda^k_2 (\a_2 \cdot \mathfrak{r})+ ...  + \lambda^k_s (\a_s\cdot \mathfrak{r}).$$ Since $\a_i \cdot \mathfrak{r} = 0$ for $i = 1,2, ..., s$ the right hand side 
of this last equation becomes $0$, so that $$\a_k \cdot \mathfrak{r} = 0, \quad k = 1,2, ..., m,$$ and in particular for $k = s+1, s+2, ..., m$ as was to be proved.\\

The vector space of all the solutions of the first $s$ equations of (\ref{HomogenousSystem}) is thus identical with the space of solutions of the entire system (\ref{HomogenousSystem}). It dimension is this $n-r$
since $r$ is the rank of the matrix of coefficients in (\ref{MatrixSystem}). It thus follows that the rank of the matrix of the first $s$ equations, i.e of the matrix which consists only of the first $s$ rows of (\ref{HomogenousSystem})
is equal to $n - (n-r) = r$. But the column vectors of this matrix are vectors with $s$ components. Thus the maximal number of linearly independent column vectors of this matrix is $\le s$ since by previous theorem
any $s+1$ vectors of $s$-dimensional vector space are linearly dependent. \\

Therefore the maximal number of linearly independent column vectors of a matrix is at most as large as the maximal number of linearly independent row vectors. \\

Now consider the transpose of the matrix of coefficients of (\ref{HomogenousSystem}). The maximal number of independent column vectors of this matrix is $s$, and the maximal number of row vectors is $r$.
Applying the previous result we also obtain $s\le r$. Hence we have $r \le s$ and $s\le r$ and therefore s = r. Thus we have proved

\begin{theorem}
The maximal number of linearly independent column vectors of a matrix is equal to the maximal number of linearly independent row vectors of that matrix. 
\end{theorem}

\begin{theorem}
With any given vector space $L$ of $\R_n$, we can always associate a system of homogenous linear equations in $n$ unknowns such that all the vectors of $L$, and no others, are vector solutions of this system.
\end{theorem}\

Let $\a_1, \a_2, ..., \a_p$ be a basis of $L$, where we set $\a_i = \{\a_{i1}, \a_{i2}, ..., \a_{in}\}$. If in addition we set 
$$\mathfrak{r} = \{ x^1, x^2, ..., x^n\}$$ then the equations 

\begin{equation}\a_i \cdot \mathfrak{r} = 0, \quad i = 1, 2, ..., p,\label{10}\end{equation}

form a system of homogenous linear equations in the unknowns $x^1, x^2, ..., x^n$ whose matrix is

\[
\begin{bmatrix}
a_{11} & a_{12}  & \hdots & a_{1n} \\
a_{21} & a_{22}  & \hdots & a_{2n} \\
\hdotsfor[2]{4}\\
a_{i1} & a_{i2}  & \hdots & a_{in}  \\
\hdotsfor[2]{4}\\
a_{p1} & a_{p2}  & \hdots & a_{pn}\\
\end{bmatrix}
\]

Let $p$ be the dimension of $L$.The rank of this system must be $n-p$. The totality of vectors which are orthogonal to $L$ is a vector space $L'$ of dimension $n-p$. It consists precisely of all the vector solutions of (\ref{10}).\\

We now seek a those vectors which are othogonal to the vector space $L'$ just found, We chose a basis $\b_1, \b_2, ..., \b_{n-p}$ of $L'$.The $\b_i$ as vectors of $L'$ are orthogonal to $L$, and this in particular to 
$\a_1, \a_2, ... \a_p$. Thus, for $i=1, 2, ..., p$, we have

$$\b_1 \cdot \a_i = 0, ~~\b_2 \cdot \a_2 = 0, ~~..., ~~\b_{n-p} \cdot \a_i = 0.$$

Now just as for $L$, the totality of all vectors orthogonal to $L'$ consists of all the vector solutions of the equations
\begin{equation}\b_1 \cdot \mathfrak{r} = 0, ~~\b_2 \cdot \mathfrak{r} = 0, ~~..., ~~\b_{n-p} \cdot \mathfrak{r} = 0.\label{11}\end{equation}
 
 Thus the $\a_i, i =1,2, ..., p$ are orthogonal to $L'$. But the vectors which are orthogonal to $L'$ form a vector space $L''$ of dimension $n - (n-p) = p$.Since the vectors $\a_1, \a_2, ..., \a_p$ are in $L''$ and are 
 linearly independent, then form a basis of $L''$. Therefore $L''$ is identical with $L$. Thus the vectors of $L$ are precisely the vector solutions of equations (\ref{11}), i.e, of a certain system of $n-p$ linear homogenous equations.
 
 \begin{theorem}
 The point solutions of a solvable system of equations of the type (\ref{LinearSystem}) form a linear space. This space has dimensions $n-r$m where $r$ is the rank of the coefficient matrix of the system. Conversely, every linear
 space may be represented as the totality of all point solutions of some suitable system of linear equations. 
 \end{theorem}
\section{Equivalence Relations}
\begin{theorem}{The equivalence classes theorem}\\
Suppose in a set $S$ we have a relation $R$ defined between certain pairs of elements, $xRy$ meaning that $x$ and $y$ stand in the given relation $R$.\\
Suppose further that $R$ has the following 3 properties:
\begin{itemize}
\item[(1)] It is reflexive: i.e. $xRx$ for all $x\in S$.
\item[(2)] It is symmetric: i.e. $xRy \Rightarrow yRx$.
\item[(3)] It is transitive: i.e. $xRy  \hbox{ and } yRz  \Rightarrow xRz$.
\end{itemize}
Then $R$ is an equivalence relations: i.e. it divides $S$ into mutually exclusive subsets so that every element of $S$ is in one and only one subset,
and so that two elements are in the same subset if and only if they stand in the relation $R$ to one another.
\end{theorem}
The subsets are called equivalence classes defined by $R$.\\

Given any element $x$ in $S$ consider all the elements $y$ such that $xRy$. These form a subset of $S$: let us call it $A_x$. We show first that two elements are in the same subset
$A_x$ if and only if they stand in the relation $R$ to one another. Suppose $yRz$ and $y\in A_x$. Then $xRy$ and so $xRz$ since $R$ is transitive. Hence $z\in A_x$. Conversely, if $y$ and
$z$ are both in $A_x$ we have $xRy$ and $xRz$, i.e. $yRx$ by the symmetric property and $xRz$: thus $yRz$ by transitivity.\\

For each element $x$ we now have a subset $A_x$, but these will not all be distinct. We will show that two such are either mutually exclusive or else identical. Suppose $A_x$ and $A_y$ both have an 
element $z$, Then $xRz$ and $yRz$, snf do $zRy$ by the symmetric property; hence $xRy$ by the transitive property. Now take any element $w$ of $A_x$. Then $xRw$ and since $xRy$ we have $yRx$, giving
$yRw$. So $w$ is in $A_y$. Hence we have shown that if $A_x$ and $A_y$ have one element $z$ in common, any element $w$ of $A_x$ is in $A_y$, i.e, $A_x \subset A_y$. Similarly $A_y\subset A_x$, 
and so the subsets $A_x$ and $A_y$ are identical.\\
Thus we have mutually exclusive subsets $A_{x_1}, A_{x_2},....$. Finally by the reflexive property any element $t$ is in one of the subsets, viz. $A_t$.

\section{Subgroups and Cosets}
\begin{theorem}
A necessary and sufficient condition for a non-empty subset $H$ to be a\\ subgroup of a group $G$ is that $gh^{-1} \in H$ for all $g, h \in H$.
\end{theorem}

\begin{itemize}
\item[(1)] Necessary: If $H$ is a subgroup $h^{-1}\in H$ and so $gh^{-1} \in H$.
\item[(2)] Sufficient: If $g$ is any element of $H$ we have $gg^{-1} \in H$, i.e, $e\in H$. Hence $eg^{-1} = g^{-1}\in H$, i.e, $H$ contains inverses. Thus if $g,h \in H$, so is $h^{-1}$ and hence
$g(h^{-1})^{-1} \in H$, i.e. $gh \in H$.
\end{itemize}

Given a group $G$ and a subgroup $H$ it is possible to decompose the whole group into subsets `parallel' to $H$, two elements being in the same subset if their `difference' is in $H$. This concept is 
better adapted in the product notation where the difference between $g$ and $k$ is expressed as $k^{-1}g$. We therefore decompose group $G$ into subsets such that $g$ and $k$ are 
in the same subset if $k^{-1}g\in H$. These subsets are known as cosets and we have a true decomposition into mutually exclusive subsets because $k^{-1}g\in H$ sets up an equivalence
relation $gRk$ if $k^{-1}g\in H$.

\begin{theorem}{Equivalence Classes of $G$ relative to $H$}
The relation defined by $gRk$ above is an equivalence relation.
\end{theorem}

\begin{itemize}
\item[(1)] Reflexive: For any $g\in H$ we have $gRg$ because $g^{-1}g = e \in H$ since $H$ is a subgroup.
\item[(2)] Symmetric: If $gRk$ then $k^{-1}g \in H$. Hence its inverse is in H, i.e. $g^{-1}k\in H$ and so $kRq$.
\item[(3)] Transitive: If $gRk$ and $kRl$ then $k^{-1}g \in H$ and $l^{-1}k \in H$. Hence $(l^{-1}k)(k^{-1}g) = l^{-1}g \in H$ and so $gRl$.
\end{itemize}

These mutual exclusive classes are called left cosets of $G$ relative to $H$. So if $gRk$ then $kRg$ and $g^{-1}k \in H$ or $k = gh$ for some $h\in H$. Hence the coset containing g is
precisely the complex $gH$, the set $\{gh\}$ for all $h\in H$.

\begin{theorem}{Left Cosets}
The left cosets of $G$ relative to $H$ are the complexes $gH,~g\in G$. Any left coset may be expressed in this form for any $g$ in it. Any two cosets are either the same of have
no element in common.
\end{theorem}

We may similarly define \it  right cosets \rm of $G$ relative to $H$ as the equivalence classes under the relation given by $gk^{-1}\in H$. They are the complexes $Hg$.\\

We normally will use left cosets and therefore will just refer to them as cosets. 
\section{Some Results in the Theory of Numbers}

\begin{definition}
If $a$ and $b$ are two integers such that $a-b$ is divisible by $n$, we say that $a$ is congruent to $b$ modulo $n$, and write $a\equiv b\mod n$, or merely $a\equiv b(n)$.
\end{definition}


\begin{theorem}Lagrange's theorem.\\
The order of any subgroup of a finite group is a factor of the order of the group.\\
The order of a finite group $G$ is a multiple of the order of every one of its subgroups.
\end{theorem}

Suppose $H$ is a subgroup of $G$, where the order of $G$ is $n$ and that of $H$ is $m$.\\
The whole group $G$ may be decomposed into cosets $gH$ relative to $H$. The number of elements in each coset must be $m$, since the coset $gH$ is formed by
taking all the elements of $H$ and pre-multiplying them by $g$, and all the elements so formed are different by the Cancellation Law. But the cosets are completely disjoint.
Hence if there are $r$ cosets we must have that $n = mr$, and so $m$ is a factor of $n$.\\

Each element $a$ of $G$ generates a cyclic subgroup, whose order is simple the order of $a$. Therefore we have: 

\begin{corollary}{(Corollary 1):}
Every element of a finite group $G$ has as order a divisor of the order of $G$.
\end{corollary}

\begin{corollary}{(Corollary 2):}
Every group $G$ of prime order $p$ is cyclic.
\end{corollary}
\begin{proof}
For the cyclic subgroup $A$ generated by any element $a\ne e$ in such a group has an order $n>1$ dividing $p$. But this implies $n=p$, and so $G=A$ is cyclic.\qed
\end{proof}

\begin{corollary}{(Corollary 3):}
The only abstract groups of order four are the cyclic group of that order and the four-group (Vierergruppe).
\end{corollary}
\begin{proof}
If a group $G$ has order 4, contains and element of order 4, it is cyclic. Otherwise, by Corollary 1, all elements of $G$ except $e$ must have order 2. 
Call them $a, b, c$. By the cancellation law, $ab$ cannot be either $ac = a, eb = b,$ of $aa = e$; hence $ab=e$. By symmetry, $ac = ca = b, bc = cb = a, ba = c$. But these, together with $a^2 = b^2 =  c^2 = e$, and $ex = xe = x$ for all $x$, give the multiplication table of the four-group.\qed
\end{proof}

\begin{theorem}{(Fermat's Little Theorem):}
If $p$ is prime, $a^p\equiv a(p)$, and if $a$ is not a mulltiple of $p$, then $a^{p-1} \equiv 1 (p)$ 
\end{theorem}

\begin{proof}
(1) The numbers $a, 2a, 3a, ..., (p-1)a$ are all different modulo $p$ if $a$ is not a multiple of $p$, and so they must be $1,2,...,(p-1)$ modulo $p$ in some order. Thus
$$ a.2a.3a ... (p-1)a \equiv 1.2.3 ... (p-1).$$
Hence $a^{(p-1)} \equiv 1(p)$. Thus $a^p \equiv a(p)$ if $a$ is not a multiple of $p$, and this latter result is true also if $a$ is a multiple of $p$, since then both sides are 0. Hence it is true for all $a$.\qed
\end{proof}

\begin{proof}
(2) The multiplication group mod p (excluding zero) has $(p-1)$ elements. The order of any element $a$ of this group is then a divisor of $(p-1)$, by Corollary 1, so that $a^{(p-1)}\equiv 1(p)$ whenever $a\ne 0(p)$. If we multiply by $a$ or both sides, we obtain the desired congruence, except for the case $a\equiv 0(p)$, for which the conclusion is trivially true.\qed
\end{proof}

\begin{proof}
(3) For a fixed prime $p$, let $P(n)$ be the proposition that $n^p \equiv n(p)$.  Then $P(0)$ and $P(1)$ are obvious. In the binomial expansion for $(n+1)^p$, every coefficient except the first and the last is divisible by $p$, hence $(n+1)^p \equiv n^p + 1 (p)$, whence $P(n)$ implies $(n+1)^p \equiv n + 1(p)$, which is the proposition $P(n+1)$.\qed
\end{proof}

\begin{theorem}{(Wilson's Theorem):}
Let $p>2$ be a prime number. The field $\Z_p$ is the splitting field of $x^p - x$.
\end{theorem}

\begin{proof}
Since $\Z_p$ is the splitting field of $x^p - x$, we have $$x^p - x = x(x-1)(x-2)\hdots (x-(p-1)).$$Thus $$x^{(p-1)} - 1 = (x-1)(x-2)\hdots(x- (p-1)),$$
and substituting $x=0$ gives 
$$-1 = (-1)(-2) \hdots (-(p-1)).$$
There are an even number of factos on the right hand side, so the minus signs cancel out, leaving
$$(p-1)! \equiv -1 (\mod{p}),$$ which is Wilson's Theorem.
\end{proof}


%\begin{array}{l}
%0 \quad \hbox{if} \quad i\neq j \\
%E[e_j^2] \quad \hbox{if} \quad i=j \\
%\end{array}$$ 

%\begin{eqnarray*}
%E[\Delta B^2] &=& E[2\,B\, \Delta B + (\Delta B)^2]\\
%&=& 2B\, \Delta B + \sigma^2 \Delta t
%end{eqnarray*}

%\begin{eqnarray*}
%&\null& E[( {1\over2}B_t^2 - {1\over2}t - \sum_j B_j \Delta B_j)^2] \\
%&=&E[{1\over4} B_t^4 + {1\over4}t^2 + (\sum_j B_j \Delta B_j)^2 - B_t^2\cdot \sum_j B_j\Delta B_j - {1\over2}tB_t^2 + t\cdot \sum_j B_j\Delta B_j] \\
%&=& {3\over4}t^2 + {1\over4}t^2 + E[(\sum B_j \Delta B_j)^2] - E[B_t^2\cdot \sum B_j\Delta B_j] - {1\over2}t^2 + 0. 
%\end{eqnarray*}

%Also, $$E[(\sum B_j \Delta B_j)^2] = \sum_{i,j} E[B_iB_j\Delta B_i \Delta B_j] = \sum_iE[B_i^2(\Delta B_i)^2]  \sum_i t_i \Delta t_i$$ and
%\begin{eqnarray*}
%&\null&E[B_t^2\cdot \sum B_j \Delta B_j] = \sum_jE[B_t^2B_j\Delta B_j] = \sum_j E[\{(B_t - B_{j+1}) + \Delta B_j + B_j\}^2\cdot B_j\cdot \Delta B_j]\\
%&=&\sum_jE[(B_t - B_{j+1})^2 B_j\Delta B_j] + \sum_j E[B_j\cdot (\Delta B_j)^3] + \sum_jE[B_j^3\cdot \Delta B_j] \\
%&+& 2\sum E[(B_t - B_{j+1}) B_j\Delta B^2_j] + 2\cdot\sum_j E[(B_t-B_{j+1})B_j^2\Delta B_j + 2 \sum_j E[B_j^2\Delta B_j^2]\\
%&=& 0 + 0 + 0 + 0 + 0 + 2\sum_j t_j\cdot \Delta t_j
%\end{eqnarray*}

\section{Chinese Remainder Theorem}

Consider a sequence of congruence equations: 
\begin{align*}
x &\equiv a_1 \pmod{n_1}\\
&\vdots \\
x &\equiv a_k  \pmod{n_k},\\
\end{align*}where the $n_i$ are coprime in pairs. Let $N = n_1 n_2 \hdots n_k$ and define $N_i = N/n_i$ be the product of all moduli but one. As the $n_i$ are pairwise coprime, $N_i$ and $n_i$ are coprime. Therefore, by the Bezout identity, there exist integers $M_i$ and $m_i$ such that $$M_i N_i + m_i n_i = 1.$$ Therefore a solution to the system of congruences is $$x = \sum_{i=1}^k a_iM_iN_i.$$
\begin{definition}
We restate the result as: if the $n_i$ are pairwise coprime, the map 
$$ x \pmod{N} \rightarrow (x\pmod{n_1},\hdots,x\pmod{n_k})$$ defines a ring homomorphism
$$\Z /N \Z \cong \Z /n_1\Z \times \hdots \times \Z/n_k\Z$$between the ring of integers modulo $N$ and the direct product of rings of integers modulo the $n_i$. This means that for doing a sequence of arithmetic operations in $\Z/N\Z$, one may do the same computation independently in each $\Z/n_i\Z$ and then get the result by applying the isomorphism (from the right to the left). 
\end{definition}

$$R /(\cap A_i) \cong R /(A_1) \times \hdots \times R/(A_k)$$We have a surjective map and know the kernel is the intersection of all the $A_i$'s. If you take a map and quotient by the kernel, then that is isomorphic to the image -- which is the whole set (everything) because the map is surjective. Therefore the LHS is isomorphic to the RHS.
\section{Fields}

\begin{theorem}{(Fields have no non-trivial Ideals):}
If a field $F$ has an ideal $I$, then $I = (0)$ or $I = (1) = F$. 
\end{theorem}
\begin{proof}
Let $a \in I$ and  $a\ne 0$. Then $a a^{-1} = 1 \in I$ and so $f.1 = f \in I$ for any $f\in F$ Therefore $ F \subset I$. But $I \subset F$ by definition of Ideals. Therefore $ I = F$. The only other possibility is $a=0$ in which case we have the trivial Ideal $ I = (0) = 0$.
\end{proof}

If $F$ is any field, then the smallest subfield of $F$ that contains the identify element 1 is called the {\elevenit prime subfield}\/ of $F$. If $F$ is a finite field, then its prime subfield is isomorphic to $\Z_p$, where $p = \char(F)$ for some prime $p$. When we want to emphasize $\Z/p\Z$ with its field structure, we denote it by $\F_p$. The prime subfield of $\Z_p$ which is isomorphic to $\Z/p\Z$ is $\F_p$. The only prime fields are the fields of residues modulo $p$ and the field of rationals.\\

The prime subfield of a field $F$ is the intersection of all subfields of $F$. The intersection of any two subfields is a field and similarly for all the other subfields. Also the intersection of all subfields has no proper subfields, since any such would be subfields of R and hence must contain the intersection which is impossible. Hence the intersection is a prime subfield. $F$ cannot contain two distinct prime subfields, since their intersection would also be a subfield and hence must be identical with both. 

\section{Finite Fields}

\begin{theorem}{(Characteristic of a Field):}
In a field $F$, all non-zero elements have the same additive order. This is called the {\rm characteristic} of $F$ and is said to be zero if all non-zero elements have infinite order. 
\end{theorem}

\begin{proof}
Suppose that any given non-zero element $y$ has finite order $m$. Then if $x$ is any other non-zero element, $ (mx)y = m(xy) = x(my) = 0$. But $y\ne 0$ and so $mx = 0$. This is a step that could not be performed in the case of rings, which may have zero divisors. Also, suppose $rx = 0$ for some $r<m$. Then $x(ry) = (rx)y = 0$ as before and since $x\ne 0$, $ry = 0$, which is impossible since $r<m$ and $m$ is the order of $y$. Hence $x$ also has order $m$. 
\end{proof}

\begin{theorem}
If $F$ has a finite characteristic, this must be a prime.
\end{theorem}

\begin{proof}
Suppose $F$ has as characteristic a composite number $hk$. If $a$ is any non-zero element of $F$, $a^2$ is non-zero and so has order $hk$, so that $hka^2 = 0$, i.e., $(ha)(ka) = 0$ and so either $ha$ or $ka$ is 0, which is impossible, since both $h$ and $k$ are less than $hk$ and $a$ must have order $hk$.  
\end{proof}

\begin{theorem}
If $F$ has characteristic p, it contains a subfield isomorphic to the field of residues modulo p.
\end{theorem}

\begin{proof}
Consider the unity 1 and denote 1+1 by 2, 1+1+1 by 3, etc. Then 1 has order $p$, and so $p.1 = 0$ (i.e., the sum of the $p$ 1's is the zero element).  Now consider the subset of elements $0,1,2,\hdots,(p-1)$. The sum and difference of any two of these is in the set as are 0 and 1. Furthermore the product of any two is in the set, since the product is the ordinary product of residues modulo $p$, by the Distributive Law. For example, $$2.3 = (1+1)(1+1+1) = 1+1+1+1+1+1 = 6.$$ Since sum and product behave as they do for residues modulo $p$ the subset is isomorphic to the set of residues modulo $p$  under residue sum and product and so, since $p$ is prime, inverses exist in the subset and the subset is isomorphic to the field of residues modulo $p$.
\end{proof}

\begin{theorem}
Let $F$ be a field of characteristic 0, it contains a subfield isomorphic to the field of rational numbers.\label{IsoRationals} 
\end{theorem}

\begin{proof}
The unity 1 has infinite order and so the elements $0,1,2\hdots$ are all distinct. Consider the subset of all elements $m/n$ where $m$ and $n$ are co-prime, $n$ is non-zero and $m/n$ means $m.n^{-1}$. It is easily seen that this subset contains the sum, difference, product and quotient of any two of its elements (except dividing by 0, of course) and also contains 0 and 1, and so is a subfield, and is trivially isomorphic to the field of rationals. 
\end{proof}

\begin{definition}
A finite field is a field which has a finite number of elements. 
\end{definition}

\begin{theorem}
Any finite field has a characteristic $p$, for $p$ a finite prime (i.e., it cannot have characteristic 0).
\end{theorem}
For by Theorem \ref{IsoRationals}\ a field with characteristic 0 contains an infinite subfield and so must itself be infinite. 

\begin{theorem}
Let $F$ be a finite field of characteristic $p$. The $F$ has $p^n$ elements, for some positive integer $n$.\label{FiniteField}
\end{theorem}

\begin{proof}
Recall that if $F$ has characteristic $p$, then the ring homomorphism $\phi: \Z \rightarrow F$ defined by $\phi(n) = n.1$ for all $n\in \Z$ has kernel $p\Z$, and thus the image of $\phi$ is a subfield $K$ of $F$ isomorphic to $\Z_p$. Since $F$ is finite, it must certainly have finite dimension as a vector space over $K$, say $[F:K] = n$. If $v_1, v_2, \hdots, v_n$ is a basis for $F$ over $K$, then each element of $F$ has the form $a_1v_1 + a_2v_2 + \hdots + a_nv_n$ for elements $a_1, a_2, \hdots, a_n \in K$. Thus to define an elements of $F$ there are $n$ coefficients $a_i$, and for each coefficient there are $p$ choices, since $K$ has only $p$ elements. Therefore the total number of ways in which an element in $F$ can be defined is $p^n$. \qed
\end{proof}

\begin{theorem}
Let $F$ be a finite field with $p^n$ elements. Then $F$ is the splitting field of the polynomial $x^{p^n}-x$ over the prime subfield of $F$. 
\end{theorem}

\begin{proof}
Let $F$ be a finite field of characteristic $p$. Then as in Theorem \ref{FiniteField} the field $F$ is an extension of degree $n$ of its prime subfield $K$, which is isomorphic to $\Z_p$. Since $F$ has $p^n$ elements, the order of the multiplicative group $F^\times$ of non-zero elements of $F$ is $p^n - 1$. Therefor $x^{p^n-1} \equiv 1$ for all $0\ne x\in F$, and so $x^{p^n} = x$ for all $x\in F$. The polynomial $f(x) = x^{p^n} - x$ has at most $p^n$ roots, and so its roots must be precisely the elements of $F$. Thus $F$ is the splitting field of $f(x)$ over K. \qed
\end{proof}

\section{Descartes Rule of Signs}

\begin{theorem}{(Descartes Rule of Signs):}
The number of variatons of sign in the subsequent terms of a polynomial, $s$, minus the number of positive roots, $p$, is a non-negative even integer.\\
$$s-p = 2r, \quad r\ge 0$$ 
\end{theorem}


a) $s-p$ is an even integer.\\ \\
Let $f(x)$ be a polynomial with a positive leading coefficient . \[ f(x) = a_0 x^n + a_1 x^{n-1} + a_2 x^{n-2} + \hdots + a_{n-1} x + a_n, \quad \hbox{and~~~}  a_0 > 0. \]
Then since the sequence of signs starts and ends with a $+$ we must have an even number of variations $+-$ and $-+$ of sign in the sequence $$+\hdots+.$$ to get back eventually to find the final $+$ sign so $s$ is even.\\

Similarly $p$ is even because the function must start at $y=a_0$ and can only cross the $x$-axis an even number of times to eventually again be in the $y>0$ region. If there are multiple roots of the form $(x-\alpha)^k = 0$ , then these are counted with their respective multiplicity. If $k$ is even, then the function does not cross the $x$-axis at the multiple root. If $k$ is odd, then the function crosses the $x$ axis and goes into the $y<0$ region. In either case, $p$ is even. Hence if $a_0 > 0$, we have that $s-p$ is even. 

Similar arguments hold for the case when $a_0<0$. In that case both $s$ and $p$ are odd numbers and therefore $s-p$ is still even.  


b) $s-p$ is non-negative. 

Use induction on the degree of the polynomial.\\
$n=1$. $x + a_n = 0$. Zero variations if $a_n$ is positive and zero positive roots so $s-p = 0$ is even. If $a_n < 0$ then there is one variation in sign and one positive root, $s=1$ and $p=1$ and $s-p=0$.\\

Induct on $n = k$.  \[ f(x) = a_0 x^k + a_1 x^{k-1} + a_2 x^{k-2} + \hdots + a_{k-1} x + a_k. \] Take the derivative with respect to $x$,
\[ f'(x) = ka_0 x^{k-1} + (k-1)a_1 x^{k-2} + (k-2)a_2 x^{k-3} + \hdots + a_{k-1}. \] 

Now, since the  $k$  terms that multiply the various $x$ powers are all positive then $$ s= \hbox{sign variation} [a_0,a_1,a_2,\hdots,a_{k-1},a_k]$$ and
$$s' = \hbox{sign variation} [a_0,a_1,a_2,\hdots,a_{k-1}]$$ Therefore $$s\ge s'$$.\\

Since the roots $p'$ are interspersed between the roots of $f(x)$ by Rolle's theorem, the number of positive roots can be $p-1$ which occur between the roots of $f(x)$ plus possibly one more between $x=0$ and the first root of $\alpha_1$ of $f(x)$, so: $$p' \ge p -1.$$. Therefore we have 

\[ s \ge s' \ge p' \ge p - 1 \] or \[ s- p \ge -1 \]. Part a) shows that $s-p$ must be an even number in all cases, so $s-p\ge -1 \implies s-p \ge 0$. Hence $$s-p=2r, \quad r \ge 0 .$$



\section{Resultant of Two Polynomials}


\section{Resultant Expressed as a Sylvester Determinant}

Consider two polynomials 
\begin{align*}
f(x)  &= \sum_{i=0}^n f_i\,x^{n-i} = f_0 x^n + f_1 x^{n-1} + f_2 x^{n-2} + \hdots + f_{n-1} x + f_n\\
g(x) &=  \sum_{j=0}^m g_j\,x^{m-j} = g_0 x^m + g_1 x^{m-1} + g_2 x^{m-2} + \hdots + g_{m-1} x + g_m\\
\end{align*}
Let use assume that $f(x)$ and $g(x)$ have a common factor $(x-\alpha)$. Then 
\begin{align*} f(x) &= (x-\alpha) f_1(x) \quad \hbox{and} \\
                     g(x) & = (x-\alpha) g_1(x).\end{align*}
Then $$f(x) g_1(x) = g(x) f_1(x) = {f(x) g(x) \over x-\alpha}.$$
 
 Assuming that 
\begin{align*}
f_1(x)  &= -\sum_{i=0}^{n-1} z_i\,x^{n-1-i} = -(z_0 x^{n-1}  + z_1 x^{n-2}  + z_2 x^{n-3}  + \hdots  +z _{n-2} x + z_{n-1} ) \\
g_1(x) &= \sum_{j=0}^{m-1} y_j\,x^{m-1-i} =  y_0 x^{m-1} + y_1 x^{m-2} + y_2 x^{m-3} + \hdots  + y_{m-2} x + y_{m-1} \\
\end{align*}
By equating like powers of $x$, we can write out a linear system corresponding to the equation $$f(x) g_1(x) - g(x) f_1(x) = 0$$ we find 
\begin{align*}
-g(x) f_1(x)  = ~ &g_0 z_0 x^{n+m-1}  + g_0 z_1 x^{n+m-2}  + g_0 z_2 x^{n+m-3}  + \hdots  + g_0 z_{n-2} x^{m+1} + g_0  z_{n-1}x^m + \\
                      &g_1 z_0  x^{n+m-2}  + g_1 z_1 x^{n+m-3}  + \hdots  + g_1 z_{n-3} x^{m+1} + g_1 z_{n-2} x^{m} + g_1 z_{n-1} x^{m-1}  +\\
                      &g_2 z_0  x^{n+m-3}  + g_2 z_1 x^{n+m-4}  + \hdots  + g_2 z_{n-3} x^{m} + g_2 z_{n-2} x^{m-1} + g_2 z_{n-1} x^{m-2}  +\\
                      &\vdots  ~~~+ \\
                      &\hdots  + (g_{m-2} z_{n-1} + g_{m-1} z_{n-2} + g_{m} z_{n-3} ) x^{2} + (g_{m-1} z_{n-1}  +  g_m z_{n-2} )x + g_m z_{n-1}\\
\end{align*}
\begin{align*}
g(x) g_1(x)  = ~ &f_0 y_0 x^{n+m-1}  + f_0 y_1 x^{n+m-2}  + f_0 y_2 x^{n+m-3}  + \hdots  + f_0 y_{m-2} x^{n+1} + f_0  y_{n-1}x^n + \\
                      &f_1 y_0  x^{n+m-2}  + f_1 y_1 x^{n+m-3}  + \hdots  + f_1 y_{n-3} x^{n+1} + f_1 y_{m-2} x^{n} + f_1 y_{n-1} x^{n-1}  +\\
                      &f_2 y_0  x^{n+m-3}  + f_2 y_1 x^{n+m-4}  + \hdots  + f_2 y_{n-3} x^{n} + f_2 y_{m-2} x^{n-1} + f_2 y_{n-1} x^{n-2}  +\\
                      &\vdots  ~~~+ \\
                      &\hdots  + (f_{n-2} y_{m-1} + f_{n-1} y_{m-2} + f_{n} y_{m-3} ) x^{2} + (f_{n-1} y_{m-1}  +  f_n y_{m-2} )x + f_n y_{m-1}\\
\end{align*}

In order to satisfy $f(x) g_1(x) - g(x) f_1(x) = 0$, all the terms multiplying like powers of $x$ each must vanish. This can be arranged as a linear system in the vector $[ z_0,z_1,..., z_{n-1},y_0,y_2,...y_{m-1}]$. The coefficient matrix of this linear system is equal to the transpose of the Sylvester Resultant matrix which is defined as
\begin{align*}
R_{m,n}(g,f)=
\underbrace{
%\rdelim\}{4}{6mm}[$J$] \\ \\ \\[4mm]  \rdelim\}{3}{6mm}[$H$] \\ \\
\begin{bmatrix}
g_0 & \cdots & g_m & & \\
&\ddots & \cdots & \ddots & \\
& & g_0 & \cdots & g_m \\
f_0 & \cdots & ~~~~~~f_n & & \\
&\ddots & \cdots & \ddots & \\
& &f_0 & \cdots &f_n \\
\end{bmatrix}
}_{m+n}
\begin{array}{l}
  \\[-8mm] \rdelim\}{3.7}{6mm}[$n$ lines] \\ \\ \\[4mm]  \rdelim\}{3}{6mm}[$m$ lines] \\ \\
\end{array} \\[-1ex]
\end{align*}

The corresponding linear system that sets all the coefficients of the same powers of $x$ to $0$ in $f(x) g_1(x) - g(x) f_1(x) = 0$ is 
\[
\begin{bmatrix}
g_0 & \cdots & g_m & & \\
&\ddots & \cdots & \ddots & \\
& & g_0 & \cdots & g_m \\
f_0 & \cdots &~~~~~~f_n & & \\
&\ddots & \cdots & \ddots & \\
& &f_0 & \cdots &f_n \\
\end{bmatrix}^T 
\begin{bmatrix}
z_0\\
\vdots\\
z_{n-1}\\
y_0\\
\vdots \\
y_{m-1}\\
\end{bmatrix} = 
\begin{bmatrix}
0\\
\vdots\\
0\\
0\\
\vdots\\
0\\
\end{bmatrix}
\]

The criterion for a non-trivial solution to the above linear system is 
\[
\det \begin{bmatrix}
g_0 & \cdots & g_m & & \\
&\ddots & \cdots & \ddots & \\
& & g_0 & \cdots & g_m \\
f_0 & \cdots & ~~~~~~f_n & & \\
&\ddots & \cdots & \ddots & \\
& &f_0 & \cdots &f_n \\
\end{bmatrix} = 0\]
The rank of the matrix $R_{m,n}(g,f)$ also gives the multiplicity of the common root of $f(x)$ and $g(x)$.
The leading term of this determinant is $g_0^n f_n^m$. The terms of in the expansion an $(n + m) \times (n+m)$ matrix $M$  of the form
$\sum_{\sigma}\epsilon_{r_1,r_2,r_3,\hdots r_{n+m}} m_{1r_1} m_{2r_2} m_{3r_3}  \hdots m_{(n+m) r_{n+m}} $ where the sum is over all permutations $\sigma$ of the numbers $[1,2,\hdots,n+m]$ and 
$sgn(\sigma) = \epsilon_{r_1,r_2,r_3,\hdots r_{n+m}} $ is the sign of the permutation. In the case of the Sylvester Resultant the terms take the form 
\[sgn(\sigma) g_0^{\nu_0}g_1^{\nu_1}\hdots g_m^{\nu_m} f_0^{\mu_0} f_1^{\mu_1} \hdots f_n^{\mu_n},\]
where 
\begin{align*}
n &=  \nu_0 + \nu_1   +   \nu_2 + \hdots  + \nu_m \\ 
m & =  \mu_0 + \mu_1  +  \mu_2 + \hdots  + \mu_n \\
\end{align*}
Factoring out $g_0^n f_0^m$ the generic term in the expansion of the determinant becomes
\[sgn(\sigma) g_0^n f_0^m  ( {g_1\over g_0})^{\nu_1} ({g_2\over g_0})^{\nu_2} \hdots ({g_m \over g_0})^{\nu_m}  ({f_1\over f_0})^{\mu_1} ({f_2\over f_0})^{\mu_2} \hdots ({f_n\over f_0})^{\mu_n} \]
Hence the Sylvester Resultant has the form 
\[ R_{m,n}(g,f) = g_0^n f_0^m \phi(\alpha,\beta) \] where $\phi$ is an integral function of the roots $\alpha_i$ and $\beta_i$.
As the determinant vanishes for any common roots $\alpha_i = \beta_k$, the determinant is divisible by $(\alpha_i - \beta_k)$ and the determinant is divisible by 
\[ g_0^n f_0^m \prod_{i,k} (\alpha_i - \beta_k) \]

\begin{theorem} $R_{m,n}(g,f)$ is a homogenous polynomial with integer coefficients $g_i,f_j$.
\begin{itemize}
\item[]{(i)} $R_{m,n}(g,f)$ is homogenous of degree $n$ in $g_m,...,g_0$ and degree $m$ in $f_n,...,f_0$.
\item[]{(ii)}If $g_i$ and $f_i$ are regarded as having degree $i$. then $R_{m,n}(g,f)$ is homogenous of degree $mn$.
\end{itemize}
\end{theorem}
\begin{proof}
It is obvious that $R_{m,n}(g,f)$ is a homogenous polynomial with integer coefficients in the coefficients $g_i, f_j$, of total degree $m+n$.
Moreover, to replace $g_i$ by $tg_i$ and $f_j$ by $uf_j$ in the Sylvester matrix means that we multiply each of the first $m$ rows by $t$
and each of the last $n$ rows by $u$, and thus the determinant $R_{m,n}(g,f)$ by $t^mu^n$, which shows part ({\it{i}}). It is here best to 
treat $g_i, f_j, t$ and $u$ as different inderminants a  do the calculations in $F(g_0,...,g_m, f_0,...,f_n, t,u)$.\\
Similarly, ({\it ii}) follows because to replace each $g_i$ by $t^ig_i$ and each $f_j$ by $t^jf_j$ in $R_{m,n}(g,f)$ yields the same result as 
multiplying the $i$th row by $t^{m+i}$ for $i=1,2,...,m$ and by $t^i$ for $i=m+1,...,m+n$, and the $j$th column by $t^{-j}$\/. The overall 
effect it to multiply the determiant $R_{m,n}(g,f)$ by $t^K$ where $K = mn+\sum_{i=1}^{m+n} i - \sum_{j=1}^{m+n} j = mn$.\qed 
\end{proof}

%\begin{bmatrix}
%z_0 \\
%z_1 \\
%\vdots \\
% z_{m-1} \\
%y_0 \\
%\vdots\\
%y_{n-1} \\
%\end{bmatrix}^T 
%= 
%\begin{bmatrix}
%0 \\
%0 \\
%\vdots \\
%0\\
%0 \\
%\vdots\\
%0 \\
%\end{bmatrix}

\section{Some Vector Theorems (3-dimensions)}
\begin{align*}
{\bf B} \times {\bf C} = 
\det\begin{bmatrix}
{\bf i} & {\bf j} & {\bf k}\\
B_x & B_y & B_z  \\
C_x& X_y & C_z \\
\end{bmatrix} = {\bf i}(B_y C_z - C_y B_z) + {\bf j}(B_z C_x - B_x C_z) + {\bf k}(B_xC_y-B_yC_x) 
\end{align*}

\begin{theorem}[BAC-CAB]
\[ {\bf a} \times ({\bf b} \times {\bf c}) = {\bf b}({\bf a}\cdot {\bf c}) - {\bf c}({\bf a}\cdot {\bf b}) \]
\[ ({\bf a} \times {\bf b}) \times {\bf c} = {\bf b}({\bf a}\cdot {\bf c}) - {\bf a}({\bf b}\cdot {\bf c}) \]
\end{theorem}

\begin{proof}
Given the expression for ${\bf B} \times {\bf C}$ we have 
%\begin{align*}
\begin{eqnarray*}
{\bf a} \times ({\bf b} \times {\bf c}) &=& 
\det\begin{bmatrix}
{\bf i} & {\bf j} & {\bf k}\\
a_x & a_y & a_z  \\
(b_yc_z - b_zc_y)&(b_zc_x - b_xc_z)  &  (b_xc_y-b_yc_x)\\
\end{bmatrix} \\ \\
&=& {\bf i} \big( a_y(b_xc_y-b_yc_x) - a_z(b_zc_x - b_xc_z) \big) + \\  
&\null & {\bf j} \big( a_z(b_yc_z - c_yb_z) - a_x(b_xc_y-b_yc_x) \big) + \\  
&\null & {\bf k} \big( a_x(b_zc_x - c_zb_x) - a_y(b_yc_z-b_zc_y) \big) \\  
\end{eqnarray*}
%\end{align*}
Prove component-by-component. The ${\bf i}^{th}$ component can be written as 
$$ b_x (a_yc_y + a_zc_z) - c_x(a_yb_y + a_zb_z)  = b_x (a_xc_x + a_yc_y + a_zc_z) - c_x(a_xb_x +a_yb_y + a_zb_z) = b_x( {\bf a}\cdot{\bf c}) - c_x({\bf a}\cdot{\bf b}).$$  
The ${\bf j}^{th}$ component can be written as 
$$ b_y (a_zc_z + a_xc_x) - c_y(a_zb_z + a_xb_x)  = b_y (a_xc_x + a_yc_y + a_zc_z) - c_y(a_xb_x +a_yb_y + a_zb_z) = b_y( {\bf a}\cdot{\bf c}) - c_y({\bf a}\cdot{\bf b}).$$  
The ${\bf k}^{th}$ component can be written as 
$$ b_z (a_xc_x + a_yc_y) - c_z(a_xb_x + a_yb_y)  = b_z (a_xc_x + a_yc_y + a_zc_z) - c_z(a_xb_x +a_yb_y + a_zb_z) = b_z( {\bf a}\cdot{\bf c}) - c_z({\bf a}\cdot{\bf b}).$$  
Putting back the respective unit vectors and adding we find the vector sum to be equal to
\[ {\bf b}({\bf a}\cdot {\bf c}) - {\bf c}({\bf a}\cdot {\bf b}).\] This demonstrates the proof that 
\[ {\bf a} \times ({\bf b} \times {\bf c}) = {\bf b}({\bf a}\cdot {\bf c}) - {\bf c}({\bf a}\cdot {\bf b}).\]
The second part of the theorem is proven in a similar manner:
\[ ({\bf a} \times {\bf b}) \times {\bf c}  = - {\bf c} \times ({\bf a} \times {\bf b}) = {\bf b}({\bf a}\cdot {\bf c}) - {\bf a}({\bf b}\cdot {\bf c}), \] using the first half of the proof along with the substitutions:
\[ {\bf a} \goes -{\bf c}\quad {\bf b} \goes {\bf a}\quad {\bf c} \goes {\bf b}\] viz.
\[ ({\bf a} \times {\bf b}) \times {\bf c}  = -{\bf c} \times ({\bf a} \times {\bf b})  \goes {\bf a} (-{\bf c}\cdot{\bf b}) - {\bf b} (-{\bf c}\cdot{\bf a}) = {\bf b}({\bf a}\cdot{\bf c}) -{\bf a}({\bf b}\cdot{\bf c}).\]\qed
\end{proof}

\begin{theorem}
\[ ({\bf a} \times {\bf b}) \cdot ({\bf c} \times {\bf b}) = ({\bf a}\cdot {\bf c})({\bf b}\cdot {\bf d}) - ({\bf a}\cdot {\bf d})({\bf b}\cdot {\bf c}) \]
\end{theorem}

\begin{proof}
\[ {\bf x}\cdot ({\bf c}\times{\bf d}) = ({\bf x}\times {\bf c})\cdot{\bf d}\] Let ${\bf x} = {\bf a}\times{\bf b}$. Then 

\begin{eqnarray*}
({\bf a} \times {\bf b}) \cdot ({\bf c} \times {\bf b}) &=&\{({\bf a} \times {\bf b})  \times {\bf c}\} \cdot {\bf d} \\
&=& \{{\bf b}({\bf a}\cdot {\bf c}) - {\bf a}({\bf b}\cdot {\bf c}) \}\cdot {\bf d} \\
&=& ({\bf a}\cdot {\bf c})({\bf b}\cdot {\bf d}) - ({\bf a}\cdot {\bf d})({\bf b}\cdot {\bf c}) 
\end{eqnarray*} 
\end{proof}

\begin{theorem}
\[ ({\bf a} \times {\bf b}) \times ({\bf c} \times {\bf d}) = {\bf b}({\bf a}\cdot {\bf c}\times {\bf d}) - {\bf a}({\bf b}\cdot {\bf c}\times {\bf d}) =  {\bf c}({\bf a}\cdot {\bf b}\times {\bf d}) - {\bf d}({\bf a}\cdot {\bf b}\times {\bf c})\]
\end{theorem}
\begin{proof}
\[ {\bf x} \times ({\bf c} \times {\bf d}) = {\bf c}({\bf x}\cdot {\bf d}) - {\bf d}({\bf x}\cdot {\bf c}) \] Let ${\bf x} = {\bf a}\times {\bf b}$. Then 

\begin{eqnarray*}
({\bf a} \times {\bf b}) \times ({\bf c} \times {\bf d}) &=& {\bf c}({\bf a}\times {\bf b}\cdot {\bf d}) - {\bf d}({\bf a}\times {\bf b}\cdot {\bf c}) \\ 
 &=& {\bf c}({\bf a}\cdot {\bf b}\times {\bf d}) - {\bf d}({\bf a}\cdot {\bf b}\times{\bf c})\qed
\end{eqnarray*}
Also 
\[ ({\bf a} \times {\bf b}) \times {\bf y}  = {\bf b}({\bf a}\cdot {\bf y}) - {\bf a}({\bf b}\cdot {\bf y}) \] Let ${\bf y} = {\bf c}\times {\bf d}$. Then 
\[({\bf a} \times {\bf b}) \cdot ({\bf c} \times {\bf b}) ={\bf b}({\bf a}\cdot {\bf c}\times{\bf d}) - {\bf a}({\bf b}\cdot {\bf c}\times{\bf d})\qed\]
\end{proof}


\begin{theorem}[Compound Vector Scalar Product]
\[ ({\bf a} \times {\bf b})\cdot ({\bf b}\times {\bf c}) \times ({\bf c}\times {\bf a}) = ({\bf a}\cdot {\bf b} \times {\bf c})^2 \]
\end{theorem}

\begin{proof}
\[ {\bf x} \times ({\bf c}\times {\bf a}) = {\bf c}({\bf x}\cdot {\bf a}) - {\bf a}({\bf x}\cdot {\bf c}) \]
Let ${\bf x} = {\bf b} \times {\bf c}$, then 
\begin{eqnarray*}
({\bf b}\times {\bf c}) \times ({\bf c}\times {\bf a}) & =&  {\bf c}({\bf b}\times {\bf c} \cdot {\bf a}) - {\bf a}({\bf b}\times{\bf c}\cdot {\bf c}) \\
                                                                                   & =&  {\bf c}({\bf a}\cdot {\bf b} \times {\bf c}) - {\bf a}({\bf b}\cdot \times{\bf c}\times {\bf c}) \\
                                                                                   & =&  {\bf c}({\bf a}\cdot {\bf b} \times {\bf c}).
\end{eqnarray*}

Therefore 
\begin{eqnarray*}
({\bf a} \times {\bf b})\cdot ({\bf b}\times {\bf c}) \times ({\bf c}\times {\bf a}) & =&  ({\bf a} \times {\bf b})\cdot {\bf c}({\bf a}\cdot {\bf b} \times {\bf c})\\
                                                                                                                                & =&  ({\bf a} \times {\bf b}\cdot {\bf c})({\bf a}\cdot {\bf b} \times {\bf c})\\
                                                                                   & =&  ({\bf a}\cdot {\bf b} \times {\bf c})^2 \qed\\
\end{eqnarray*}
\end{proof}


\section{Reciprocal Vector Space (3-dimensions)}
Take three non-coplanar vectors ${\bf a}, {\bf b}, {\bf c}$ which form a right-handed system of vectors and form the vector scalar product whose magnitude is the volume, $V$, of the parallelapiped with edges given by ${\bf a}, {\bf b}, {\bf c}$ 
\[ V = {\bf a}\cdot{\bf b} \times {\bf c}. \] The set of vectors ${\bf a'}, {\bf b'}, {\bf c'}$ which are reciprocal to ${\bf a}, {\bf b}, {\bf c}$  is given by

\begin{eqnarray*}
{\bf a'}&=& { {\bf b} \times{\bf c}\over {\bf a}\cdot{\bf b} \times {\bf c}} \\   
{\bf b'}&=& { {\bf c} \times{\bf a}\over {\bf a}\cdot{\bf b} \times {\bf c}} \\
{\bf c'}&=& { {\bf a} \times{\bf b}\over {\bf a}\cdot{\bf b} \times {\bf c}} \\
\end{eqnarray*}

The reciprocal basis ${\bf a'}, {\bf b'}, {\bf c'}$ satisfy the following identities
\begin{eqnarray*}
{\bf a}\cdot {\bf a'}&=& 1, \quad {\bf b}\cdot {\bf a'} = 0, \quad {\bf c}\cdot {\bf a'} =0 \\   
{\bf b}\cdot {\bf b'}&=& 1, \quad {\bf a}\cdot {\bf b'} = 0, \quad {\bf c}\cdot {\bf b'} =0  \\
{\bf c}\cdot {\bf c'}&=& 1,  \quad {\bf a}\cdot {\bf c'} = 0, \quad {\bf b}\cdot {\bf c'} =0 \\
\end{eqnarray*}
 
\begin{theorem}[Reciprocal Basis has Reciprocal Volume = 1/V]
\[ {\bf a'}\cdot{\bf b'} \times {\bf c'} = {1\over V}  \]
\end{theorem}
\begin{proof}
\begin{eqnarray*}
{\bf a'}&=& { {\bf b} \times{\bf c}\over V} \\   
{\bf b'}&=& { {\bf c} \times{\bf a}\over V} \\
{\bf c'}&=& { {\bf a} \times{\bf b}\over V} \\
\end{eqnarray*}
\begin{eqnarray*}
{\bf a'}\cdot{\bf b'} \times {\bf c'}  &=& {({\bf b}\times {\bf c})\cdot({\bf c}\times{\bf a})\times({\bf a}\times {\bf b})\over V^3} = {({\bf a}\times {\bf b})\cdot ({\bf b}\times {\bf c})\times({\bf c}\times{\bf a}) \over V^3} \\
                                                       &=& {({\bf a}\cdot {\bf b} \times {\bf c})^2\over V^3} = {V^2\over V^3} = {1\over V}\qed
\end{eqnarray*}
\end{proof}


 \section{Notes on Riemannian Hypersurfaces}
 

 Let us summarize some facts about Riemannian Geometry:\\
 
Tangent Vectors $\r_j= \partial\r/\partial x^j $, $\a = a^i\r_i$, second derivatives: $\r_{ij} = {\partial^2 \r\over \partial x^i\partial x^j}$

Metric: $\gij= \r_i\cdot\r_j = g_{ji}$ (symmetric).

Dot product: $\a\cdot\b= \gij a^ib^j$.\\

Consider the determinant 
\begin{equation}
g = \det \left| \gij \right| =
\begin{vmatrix}
g_{11} & g_{12}  & \hdots & g_{1n}  \\
g_{21} & g_{22}  & \hdots & g_{2n}  \\
\hdotsfor[2]{4}\\
g_{i1} & g_{i2}  & \hdots & g_{in}  \\
\hdotsfor[2]{4}\\
g_{n1} & g_{n2}  & \hdots & g_{nn}  \\
\end{vmatrix}
\end{equation}


%Consider an $n$-by-$n$ matrix $A$ with elements $a_{ij}$. Take the columns of $A$ to be $n$-dimensional vectors $$A = \left[ \a_1, \a_2, ...., \a_n \right].$$ Divide these 
%vectors into to systems. Let the first consist of the $p$ vectors $$\a_{\alpha_1}, \a_{\alpha_2},...,\a_{\alpha_p},  (\alpha_1 < \alpha_2 <...< \alpha_p)$$, the second of the remaining $q =n-p$
%vectors $$\a_{\beta_1},\a_{\beta_2},...,\a_{\beta_q}, (\beta_1 < \beta_2 < ... < \beta_q, \quad p+q = n).$$ 
%The Laplace's Expansion Theorem of a determinant in terms of Complementary Minors is given by



%Affine Connection: 
%\begin{eqnarray*}
%\Gamma_{ijm} &=& \r_{ij}\cdot \r_m = \half \left({\partial g_{im}\over \partial x^j} +{\partial g_{jm}\over \partial x^i} - {\partial g_{ij}\over \partial x^m}\right), \\
%\Gamma^k_{ij} &=& g^{km}\Gamma_{ijm}= \half g^{km}\left({\partial g_{im}\over \partial x^j} +{\partial g_{jm}\over \partial x^i} - {\partial g_{ij}\over \partial x^m}\right).
%\end{eqnarray*}

%\begin{eqnarray*}
%\Gamma^j_{ij} &=& \half g^{jm}\left({\partial g_{im}\over \partial x^j} +{\partial g_{jm}\over \partial x^i} - {\partial g_{ij}\over \partial x^m}\right), \\
%                       &=& \half g^{jm}{\partial g_{jm}\over \partial x^i},       \\
%                       &=& \half {1\over g}{\partial g\over \partial x^i},\qquad\qquad\hbox{see below,}\\
%                       &=& \half {\partial \log{g}\over \partial x^i},                 \\
%                       &=& {1\over\sqrt{g}}{\partial \sqrt{g}\over \partial x^i}.
%\end{eqnarray*}

We suppose, here and throughout, that $g$ is not zero. Let $\Delta^{ij}$ be the cofactor of $\gij$ in this determinant, so that 
$$g_{mr} \Delta^{ms} = g_{rm}\Delta^{sm} = \delta^s_r\, g,$$ which follows from the ordinary rules for developing a determinant. \\
 
Let us construct new quantities $\gijinv$. 
The values of the components of $\gijinv$ are equal to the cofactor of the $\gij$, divided by the full determinant $g$,
\begin{equation} g^{kl} = g^{-1} \Delta^{kl}.  \label{ginveq} \end{equation} Alternatively the cofactor of $\gij$ can be expressed in terms of $\gijinv$ as folllows
\begin{equation} \Delta^{ij} = g\,g^{ij}\label{cofactor}. \end{equation}
$\gijinv$ satisfies the following equations:
\begin{equation}
g_{ik}g^{kl}  = \delta^l_i \label{ginv}\hbox{~~inverse}
\end{equation}
Now $\gij$ is symmetric and we can also show that $\gijinv$ is symmetric. Multiply Eq. (\ref{ginv}) by $g_{ls}g^{ir}$. The left hand side becomes
\[ g_{ik}g^{kl} g_{ls}g^{ir} =g_{ki}g^{ir} g^{kl}g_{sl} = \delta^r_k g^{kl} g_{sl} = g_{sl}g^{rl}, \] while the right hand side becomes
\[ \delta^l_i g_{ls}g^{ir} = g_{is}g^{ir} = g_{si}g^{ir} = \delta^r_s.\] Therefore we obtain 
\begin{equation}
g_{sl}g^{rl} = \delta^r_s \hbox{~~inverse}\label{ginvtr}
\end{equation}

Comparing Eqs (\ref{ginv}) and (\ref{ginvtr}), we find that
\[ g^{kl} = g^{lk} \]

Since $\gij$ is symmetric, it is obvious that $\gijinv$ should be symmetric also. \\

Let us examine the implications of the above applied to surfaces embedded in $\R^n$. 
A {\elevenit curve} is defined as the totality of points given by the equations

\begin{equation}
x^r = f^r(u)\quad (r = 1,2,...,n).\label{curve}
\end{equation}
Here $u$ is a parameter and $f^r(u)$ are $n$ functions.\\

Next consider the totality of points given by 
\begin{equation} x^r = f^r(t^1, t^2, ..., t^m)\quad (r= 1,2,...,n),\label{subspace}\end{equation}
where the $t$'s are parameters and $m<n$. This set of points forms $V_m$, a subspace of $V_n$. 
According to the Implicit Function Theorem it is possible to eliminate the parameters from (\ref{subspace}). As long as the system is differentiable enough and the the Jacobian of the first $m$ equations does not vanish, we can 
find $m$ functions $g^i(x^1,...,x^m)$ such that  \begin{equation} t^i = g^i(x^1,...,x^m) \mbox{ for } i = 1,...,m\end{equation}which implies that the remaining functions $x^r, r = m+1,...,n$ can be written in terms of the first $m$ coordinates 
$x^1, ..., x^m$ as follows:
\begin{equation} x^r = f^r(x^1, ..., x^m) \mbox{ for } r = m+1, ..., n\end{equation} In the case of $m=n-1$ there is only one such relation and
so, elimination gives just one equation 

\begin{equation} \phi(x^1, x^2, ..., x^n) = x^n - f^n(x^1,...,x^m) = 0.\label{hypersurface}\end{equation}

In this case, $V_{n-1}$ divides portions of $V_n$ into two parts for which respectively $\phi$ is positive and negative. $V_{n-1}$ is commonly referred to as a {\elevenit hypersurface} in $V_n$.\\

Let $V_{n}$ be a region of $\R^n$, bounded by a smooth (i.e. sufficiently differentiable) surface $B_{n-1}$, which is an $n-1$ dimensional closed manifold. We take $B_{n-1}$ as an example of a hypersurface $V_{n-1}$.
We has suppose that $B_{n-1}$ is parameterized by $n-1$ independent parameters $t^\alpha (\alpha = 1,2,...,n-1)$. Let the hypersurface $B_{n-1}$ be defined implicitly by the equation (\ref{hypersurface}).\\

On the surface of $B_{n-1}$, $x^i = x^i(t^\alpha)$, and therefore
\begin{equation} {\partial \phi\over \partial x^j}{\partial x^j\over \partial t^\alpha} = 0. \label{surfaceTangents} \end{equation}

Equation (\ref{surfaceTangents}) may be regarded as $n-1$  linear conditions upon the $n$ quantities ${\partial \phi / \partial x^j}$. Since the matrix 
$$\left[
{\partial x^j \over \partial t^a} 
\right]$$

is assumed to have its full rank value $n-1$ on $B_{n-1}$, the mutual  ratios of the partials $\partial \phi / \partial x^j$ are fully determined by
(\ref{surfaceTangents}).\\

For future reference let us define a set of determinants with respect to the change of variables between the $x^i$ and the $t^\alpha$

\begin{equation}
 D_k = (-1)^{k-1} \det\left[ {\partial(x^1, ..., x^{k-1},x^{k+1},...,x^n)\over \partial (t^1, ..., t^{n-1})} \right], ~~k = 1,2,...,n \label{dsubk}
 \end{equation}
The $n$ determinants $D_k$ satisfy the $n-1$ linear conditions 
\begin{equation} D_k {\partial x^k\over \partial t^\alpha} = 0,\label{DkEq} \end{equation}
which can be established by noting that the left hand side can be written as an $n$-by-$n$ determinant which has two rows the same and therefore vanishes.
In view of (\ref{surfaceTangents}, conditions (\ref{DkEq}) imply that the $\partial \phi / \partial x^k$ and $D_k$ are proportional
\[ {\partial \phi\over \partial x^k} =  \alpha D_k,\] for some scalar factor of proportionality $\alpha$. Hence the $\partial \phi/ \partial x^k$ determine the unit 
normal to $B_{n-1}$, namely

\begin{eqnarray}
n_k &=& {1\over \sqrt{(\nabla \phi)^2}} {\partial \phi\over \partial x^k}\\
n_k &=& {D_k\over\sqrt{g^{ij} D_iD_j}} = {D_k\over D},\label{normB}
\end{eqnarray}
where \[ D^2 = g^{ij} D_iD_j = g^{ij} {\partial \phi\over \partial x^i} {\partial \phi\over \partial x^j } \alpha^{-2}.\]

%Covariant Derivative:
%$$V^i_{;j} = {\partial V^i\over \partial x^j} + V^k\Gamma^i_{kj}.$$
%$$V_{i;j}  = {\partial V_i\over \partial x^j} - V_k\Gamma^k_{ij}.$$
%
%Covariant Divergence of a Contravariant Vector:
%$$\eqalign{
%V^j_{;j} &= {\partial V^j\over \partial x^j} + V^i\Gamma^j_{ij},\crno
%&={1\over\sqrt{g}}\left({\partial (\sqrt{g}V^i)\over\partial x^i}\right).\cr}$$
%
%Variation of $g$, Definitions:\hfil\break
%$g=\cofac(\gab)\gab=\cofac(\gabinv)\gabinv$,\hfil\break
%$\gabinv=\cofac(\gab)/g$,\hfil\break
%$\gab =\cofac(\gabinv)/g$.
%$$\eqalign{\delta g=&\cofac(\gabinv)\delta\gabinv\crno
%=&g\gab\delta\gabinv.\cr}$$
%$$\eqalign{\delta\sqrt{-g}=&-\half\delta g/\sqrt{-g}\crno
%=&\sqrt{-g}(-\half\gab\delta\gabinv).\cr}$$
%$$\delta(\sqrt{-g}\gmunuinv)= 
%\sqrt{-g}(\delta\gmunuinv - \half\gab\gmunuinv\delta\gabinv).$$
%Reduction of unity: Consider a set of n mutually orthogonal unit vectors
%$$e^j_{(1)},e^j_{(2)},...,e^j_{(n)}$$
%Then $$\gij e^i_{(l)}e^j_{(m)} = \delta_{lm}.$$
%If we set $$e_{(m)i} = \gij e^j_{(m)},$$
%this becomes $$e^i_{(l)}e_{(m)i} = \delta_{lm}.$$ The basis vectors can
%be arranged into a matrix which is orthogonal. The rows of an orthogonal
%matrix also form an orthogonal system. This implies 
%$$\sum^n_{m=1} e^i_{(m)}e_{(m)j} = \delta^i_j.$$ Multiplying this by 
%$g^{jk}$ and summing over $j$, we get
%$$\sum^n_{m=1}e^i_{(m)}g^{jk} e_{(m)j} = \delta^i_jg^{jk} = g^{ik}.$$
%From the above definitions we have:
%$$g^{jk}e_{(m)j} = g^{jk}g_{jr}e^r_{(m)} = \delta^k_re^r_{(m)} = e^k_{(m)}.$$
%So we get the well known expression for reduction of unity:
%$$g^{ik}= \sum^n_{m=1}e^i_{(m)}e^k_{(m)}.$$
%
%Curvature Tensor and derivatives:
%$$R^i_{ jkl} = 
%{\partial \Gamma^i_{jl}\over\partial u^k} -
%{\partial \Gamma^i_{jk}\over\partial u^l} +
%\Gamma^p_{jl}\Gamma^i_{pk} - \Gamma^p_{jk}\Gamma^i_{pl}.$$
%$$R_{\mu\nu} = R^\rho_{\mu\nu\rho}.$$
%$$R_{\mu\nu}\delta(\sqrt{-g}\gmunuinv)= 
%\sqrt{-g}(R_{\mu\nu} - \half\gmunu R)\delta\gmunuinv.$$
%Let $G_{\mu\nu} = R_{\mu\nu} - \half\gmunu R$, then
%$$R_{\mu\nu}\delta(\sqrt{-g}\gmunuinv)= 
%\sqrt{-g}G_{\mu\nu}\delta\gmunuinv.$$
%
\section{Hypersurfaces and Concepts of Area}
Consider an $n-1$ dimensional hypersurface, $B_{n-1}$, embedded in a Riemannian manifold $M_n$ (which we assume to be sufficiently differentiable). Let $x^1,...,x^n$ be local coordinates in $M_n$ and suppose that
$B_{n-1}$ is represented locally in the form 
\[ x^j = x^j(t^1, ..., t^{n-1}),  \quad j = 1,...,n \] 

The metric tensor of $M_n$ is denoted by $\gij$. and the metric tensor on $B_{n-1}$ by $\overline{g}_{\alpha\beta}$. These two tensors are related by
\begin{equation}
 \gstar= \gij {\partial x^i\over \partial t^\alpha} {\partial x^j\over \partial t^\beta} = {\partial x^i\over \partial t^\alpha} \gij {\partial x^j\over \partial t^\beta} \label{gstar}
\end{equation}

The element of area (or volume) in $M_n$ is 
\begin{equation}
dV = \sqrt{g}\,dx^1...dx^n,\label{areag}
\end{equation}
where $g$ is the determinant of the metric tensor in $M_n$. Similarly the element of area (volume) in the hypersurface $B_{n-1}$ is given by 
\begin{equation}
dS = \sqrt{\overline{g}}\,dt^1...dt^{n-1}.\label{areagstar}
\end{equation} 
We will use (\ref{gstar}) to define the element of area in $B_{n-1}$ based on expansion of the determinant of $\overline{g}$ in terms of products of the various minors in a manner similar to the
Cauchy-Binet formula or the Laplace expansion of a determinant. Let $S_n$ stand for the set of permutations of the numbers $\{1,2....,n \}$ and let $P$ represent one of the members of $S_n$.\\

$\gstar$ is an ($n$-1)-by-($n$-1) matrix and we can consider its determinant. 
\begin{equation}
\overline{g} = \det |\overline{g}| = \sum_{P \in S_{n-1}} \epsilon_{ p_1 p_2 ... p_{n-1}} \overline{g}_{1 p_1}  \overline{g}_{2 p_2} ... \overline{g}_{n-1 p_{n-1}} \label{detgstar}
\end{equation}

We can think of the right hand side of (\ref{gstar}) as representing the matrix product of three matrices. 
\begin{eqnarray*}
J &=&  \left[ {\partial x^i\over \partial t^\alpha} \right] \hbox{~is $n$-by-$(n-1)$},\\ 
G &=&\bigg[ g \bigg] \hbox{~is $n$-by-$n$}
\end{eqnarray*}
 and (\ref{gstar}) can be written in the form:
\begin{eqnarray*}
 \overline{g}_{\alpha\beta} &=& \sum_{i,j=1}^n \left[ {\partial x^i\over \partial t^\alpha} \right]^T_{\alpha i} \left[g\right]_{ij} \left[ {\partial x^i\over \partial t^\beta} \right]_{j\beta} = {\partial x^i\over \partial t^\alpha} \gij {\partial x^j\over \partial t^\beta} \hbox{~summation convention on $i$ and $j$}\\
\bigg[ \gstar\bigg] &=& J^T G J \\
\end{eqnarray*}
Inserting this computation into (\ref{detgstar}) we have
$\gstar$ is an ($n$-1)-by-($n$-1) matrix and we can consider its determinant.  Let $\overline{g} =\det |\overline{g}_{\alpha\beta} |$.

\begin{eqnarray*}
\overline{g} &=& \sum_{P \in S_{n-1}} \epsilon_{ p_1 p_2 ... p_{n-1}} \sum_{\substack{i1,..,i_{n-1} = 1 \\ k_1,...,k_{n-1}=1}}^n  
(J^T_{1i_1}G_{i_1k_1}J_{k_1p_1})
(J^T_{2i_2}G_{i_2k_2}J_{k_2p_2})...
(J^T_{n-1i_{n-1}}G_{i_{n-1}k_{n-1}}J_{k_{n-1}p_{n-1}})\\
&=& \sum_{P \in S_{n-1}} \epsilon_{ p_1 p_2 ... p_{n-1}} \sum_{\substack{i1,..,i_{n-1} = 1 \\ k_1,...,k_{n-1}=1}}^n  
(J^T_{1i_1}...J^T_{n-1i_{n-1}} )
(G_{i_1k_1}...G_{i_{n-1}k_{n-1}} )
(J_{k1p_1}...J_{k_{n-1}p_{n-1}} )\\
&=& \sum_{\substack{i1,..,i_{n-1} = 1 \\ k_1,...,k_{n-1}=1}}^n  
(J^T_{1i_1}...J^T_{n-1i_{n-1}} )
(G_{i_1k_1}...G_{i_{n-1}k_{n-1}} )
\sum_{P \in S_{n-1}} \epsilon_{ p_1 p_2 ... p_{n-1}} (J_{k1p_1}...J_{k_{n-1}p_{n-1}} )\\
\end{eqnarray*}
Examining the last term in the sum on the right hand side we find a sum over the $n-1$ set $\{p1,p2,...,p_{n-1}\}$ for a particular choice of index values: $\{k1,...k_{n-1}\}$ 
\begin{equation}
\sum_{P \in S_{n-1} }\epsilon_{ p_1 p_2 ... p_{n-1}} (J_{k1p_1}...J_{k_{n-1}p_{n-1}}) = \epsilon_{ k_1 k_2 ... k_{n-1}} | J_k | = \epsilon_{ k_1 k_2 ... k_{n-1}} (-1)^{1+k} D_k
\end{equation}
where the index $k$ is not included in $\{k1,...k_{n-1}\}$ and final result involves the respective determinant $D_k$.\\

At this point notice that the sum over the $n-1$ variables $k1,k2,...,k_{n-1}$ involves summing over all values from $1$ to $n$ for each $k_i$, but that these values must be distinct. If $k_i = k_j$ for some pair $i,j$, then the sum is
zero because of the presence of the permutation symbol $\epsilon_{ k_1 k_2 ... k_{n-1}}$. This means that there are $n-1$ distinct $k$ values chosen from the full set of $n$ values, therefore one of the $k_i$ values is to be omitted in this sum.  Let $k$ stand for this missing value and $J_k$ stand for the matrix $J$ with the $k$-th row deleted. The set of $k_i$ values in any set is also propagated on to the $G_{ik}$ matrices and the corresponding $k$-th column of $G$ will not contribute this particular set of choices. There will be $n$ such cases and therefore we insert a sum over $k=1,2,...,n$ to account for each case in the following. The sum over the $i_1,...,i_{n-1}$ indices follows that same way.

\begin{eqnarray*}
\overline{g} &=& \sum_{k=1}^n \sum_{\substack{i1,..,i_{n-1} = 1 \\ k_1,...,k_{n-1}=1, \ne k}}^n  
(J^T_{1i_1}...J^T_{n-1i_{n-1}} )
(G_{i_1k_1}...G_{i_{n-1}k_{n-1}} )
\epsilon_{ k_1 k_2 ... k_{n-1}} (-1)^{1+k} D_{k}\\
&=& \sum_{i, k =1}^n \sum_{i1,..,i_{n-1} = 1, \ne i }^n  
(J^T_{1i_1}...J^T_{n-1i_{n-1}} )
\epsilon_{ i_1 i_2 ... i_{n-1}} (-1)^{i + k} \hbox{cofactor}(g_{ik})
 (-1)^{1+k} D_{k}\\
&=& \sum_{i, k =1}^n \sum_{~~~i1,..,i_{n-1} = 1,\ne i}^n  
\epsilon_{ i_1 i_2 ... i_{n-1}} (J^T_{1i_1}...J^T_{n-1i_{n-1}} )
(-1)^{i + k} \Delta^{ik}
(-1)^{1+k} D_{k}\\
&=& \sum_{i,k =1}^n  
(-1)^{1+i} D_{i}
(-1)^{i + k} \Delta^{ik}
(-1)^{1+k} D_{k}\\
&=& \sum_{i,k =1}^n  D_{i} \,g \,g^{ik} \,D_{k}= g \sum_{i= 1}^n  D^{k} \,D_{k} = g\, D^2,\\ 
\label{expdetgstar}
\end{eqnarray*}

where $g = |\gij|$ as before and the sum is over the $n$ values of $i$ and $k$ which correspond to the rows or columns which are crossed out to form the $n$ minors each of dimension ($n$-1)-by-($n$-1). Hence we find that
$$\overline{g} = g\,D^2.$$

We finally derive the relationship between the element of surface are on the hyperplane $H$ and the metric of the $n$-dimensional Riemann $M_n$ space that $H$ is embedded inside. 

\begin{equation}
dS = \sqrt{\overline{g}}\,dt^1...dt^{n-1} =  \sqrt{g\,D^2} \,dt^1...dt^{n-1}  \label{relArea}
\end{equation}

If the $n$-1 dimensional hypersurface $H$ which is embedded in Cartesian $\R^n$ is given in parametric form as follows:
$$\mathfrak{r}(x) = \{ x, f(x) \} $$ 
Then $\gij  = I_{n}$ ($n$-by-$n$ identity matrix) and $J$ is an $n$-by-$n-1$ matrix:

\begin{equation}
J  =
\begin{bmatrix}
1 & 0 & \hdots  & 0 \\
0 & 1  & \hdots & 0 \\
\hdotsfor[2]{4}\\
0 & 0  & \hdots & 1 \\
{\partial f\over \partial x^1} & {\partial f\over \partial x^2}   & \hdots & {\partial f\over \partial x^{n-1}} \\
\end{bmatrix}
\end{equation}

\begin{equation}
J^T  =
\begin{bmatrix}
1 & 0 & \hdots  & & 0 & {\partial f\over \partial x^1}  \\
0 & 1  & \hdots & & 0 &{\partial f\over \partial x^2}  \\
\hdotsfor[2]{5}\\
0 & 0  & \hdots & & 1 &{\partial f\over \partial x^n} \\
\end{bmatrix}
\end{equation}

Working out the minors, we find 
$D_n = 1$ (cross out the last row of $J$), and crossing out any of the other rows $i<n$ inserts a $0$ in the $i$-th diagonal and shifts all the other rows $i+1, ..., n$ up one row. The resulting determinant is equal to  
$D_k = (-1)^{1+k}{\partial f\over \partial x_k}$
 
Therefore we obtain 

\begin{equation}
\overline{g} = \sum_{k=1}^n D_k = ({\partial f\over \partial x^1})^2 + ({\partial f\over \partial x^2})^2 + ... +  ({\partial f\over \partial x^n})^2 + 1  
\end{equation} 
 
And the resulting magnitude, given that $g_{ij} = I_n$ is 
$\sqrt{D^2} = \sqrt{1 + |\nabla f|^2}$ and

\begin{equation}
dS = \sqrt{1 + | \nabla f |^2} ~dt^1...dt^{n-1}.\label{Stephen}
\end{equation}

\section{Application to the Green Theorem}

In Riemannian Geometry the divergence of a vector $\b$ is given by 

$$ \nabla \cdot \b = {1\over \sqrt{g} }  {\partial\over \partial x^i} \big(\sqrt{g}\, b^i\big) \hbox{~summation convention}$$

Let us form the integral of the divergence of $\b$ over a region $R$ in $V_n$ bounded by the closed surface $B_{n-1}$

\begin{equation}
\int_{R} \, \nabla \cdot \b\, dV = \int_R \, {1\over \sqrt{g} }  {\partial\over \partial x^i} \,\big(\sqrt{g}\, b^i\big)\,  dV =\int_R\, {1\over \sqrt{g} }  {\partial\over \partial x^i} \big(\sqrt{g}\, b^i\big)\,\sqrt{g}\, dx^1...dx^n 
\end{equation}

Now let us integrate the differentiated terms over the range of the appropriate variable for each one. We obtain an integral over the boundary surface $B_{n-1}$: for the $k$ term this is ($k$ not summed)
\begin{equation}
\int_B \, \sqrt{g}\, b^k dx^1...dx^{k-1}dx^{k+1}...dx^n = \int_B \, \sqrt{g} \, b^k D_k\, dt^1...dt^{n-1},
\end{equation}

where $D_k$ is given as in (\ref{dsubk}) represents the transformation of the integration variables from the $x^i$ to the coordinates as given on $B_{n-1}$.\\

From the above set of equations we see that 
\begin{eqnarray*}
\int_{R_n}\, \nabla \cdot \b \, dV_n &=& \int_{B_{n-1}} \, b^k D_k \sqrt{g}\, dt^1...dt^{n-1}\\
&=& \int_{B_{n-1}}\, b^k n_k \sqrt{g}\, D\ dt^1...dt^{n-1}\\
&=& \int_{B_{n-1}} \b \cdot \mathfrak{n} \, dS,
\end{eqnarray*}
where $dS$ is given by (\ref{relArea}).

\begin{theorem}[Divergence Theorem]
\[\iiint_V \, A^k_{,k} \, dV = \iint_S \, A^k\, \nu_k \, dS \]
\end{theorem}

\section{Extension and Stokes' Theorem}

In a space of $n$ dimensions, the {\bf Generalized Kronecker delta} $\delta^{k_1\,k_2\,...\,k_m}_{s_1\,s_2\,...\,s_m}$ for any positive integer $m$ is defined as 
\begin{itemize}
\item[]$\delta^{k_1\,k_2\,...k_m}_{s_1\,s_2\,...s_m} = +1$ if $k_1\,k_2\,...k_m$ are distinct integers selected from the range $1, 2, ..., n$, \\
\null~~~~~~~~~~~~~~~~~~~and if $s_1, s_2, ...,s_m$ is an {\elevenit even} permutation of  $k_1\,k_2\,...k_m$.
\item[]$\delta^{k_1\,k_2\,...k_m}_{s_1\,s_2\,...s_m} = -1$ if $k_1\,k_2\,...k_m$ are distinct integers selected from the range $1, 2, ..., n$, \\
\null~~~~~~~~~~~~~~~~~~~and if $s_1, s_2, ...,s_m$ is an {\elevenit odd} permutation of  $k_1\,k_2\,...k_m$.
\item[]$\delta^{k_1\,k_2\,...k_m}_{s_1\,s_2\,...s_m} = 0$ if any two $k_1\,k_2\,...k_m$ are, equal, or if any two $s_1, s_2, ...,s_m$ are equal, or if the set \\
\null~~~~~~~~~~~~~~~~~of numbers $\{k_1, k_2, ..., k_m\}$ differs, apart from order, from the set $\{s_1, s_2, ..., s_m\}$.
\end{itemize} 

In a space of $n$ dimensions, in which no metric is assigned, consider a subspace $V_m$ of $m$ dimensions $m\le n$ defined by the parametric equations
$$x^k = x^k(y^1, y^2, ..., y^m),~ k = 1, 2, ..., n.$$ In this $V_m$ consider a certain region of $R_m$. Let us divide $R_m$ into cells by $m$ families of surfaces
$$f^\alpha(y) = c^\alpha,~ \alpha = 1, 2, ..., m.$$ where $c^\alpha$ are constants, each taking on a number of discrete values and so forming a family of surfaces.\\

It is easy to attach a meaning to the statement that a point $P$ of $V_m$ lies in a certain cell, because $P$ may be said to lie between two surfaces of the same 
family, say family $\alpha$, if the expression $f^\alpha(y) - c^\alpha$ changes sign when we change $c^\alpha$ from the value belonging to one of these surfaces to 
the value belonging to the other. Here $y$ refers to the parameter values at the point $P$.\\

Now pick one of the cells and let $A$ be the point corresponding to one of the corners in the cell. Through $A$ there passes one surface of each family. Let the 
corresponding constants be $c^1, c^2, ..., c^m$. Passing along the edge for which $c^1$ alone changes, we arrive at another corner $B_1$ for which the constants
have values $c^1 + \Delta c^1, c^2, ..., c^m$. In passing from $A$ to $B_1$ the parameters change from $y^\alpha$ at $A$ to $y^\alpha + \Delta_1 y^\alpha$ at $B_1$
and the coordinates from $x^k$ to $x^k + \Delta_1 x^k$. Similarly the parameter values and the coordinates of all corners $B_1, B_2, ..., B_m$ of the cell reached by
traveling from $A$ along an edge of the cell can be written as 
\begin{eqnarray*}
&& y^\alpha + \Delta_\beta y^\alpha,\\
&&x^\alpha + \Delta_\beta x^k.
\end{eqnarray*} and we form the determinants 

\begin{equation}
\Delta^{k_1\,...\,k_m}  =  
\begin{vmatrix}
\Delta_1 x^{k_1} & \Delta_1 x^{k_2} & \hdots & \Delta_1 x^{k_m} \\
\Delta_2 x^{k_1} & \Delta_2 x^{k_2} & \hdots & \Delta_2 x^{k_m} \\
\hdotsfor[2]{4}\\
\hdotsfor[2]{4}\\
\Delta_m x^{k_1} & \Delta_m x^{k_2} & \hdots & \Delta_m x^{k_m} \\
\end{vmatrix}\label{detDeltak}
\end{equation}
Introducing the generalized Kronecker delta $\delta^{k_1\,k_2\,...\,k_m}_{s_1\,s_2\,...\,s_m}$, these determinants can be written compactly using the Summation Convention
\begin{equation}
\Delta^{k_1\,...\,k_m}  =  \delta^{k_1\,k_2\,...\,k_m}_{s_1\,s_2\,...\,s_m}\, \Delta_1 x^{s_1}\, \Delta_2 x^{s_2} \hdots \Delta_m x^{s_m}\label{Deltaktensor}
\end{equation}
Now if the edges had been taken in a different order, then the various determinants might have opposite signs, but apart from this the determinants are independent of the order taken. 
Also if we have used a different corner to start from the determinants formed similarly would have different values. But since we are going to be working with infinitesimal, the differences 
will be of higher order than the determinants in (\ref{Deltaktensor}). Using differential displacements, $d_\beta x^i$ (where $\beta$ indexes the $\beta$-th edge of the cell) we have
\begin{equation}
d\tau_m^{k_1\,...\,k_m}  =  \delta^{k_1\,k_2\,...\,k_m}_{s_1\,s_2\,...\,s_m}\, d_1 x^{s_1}\, d_2 x^{s_2} \hdots d_m x^{s_m}\label{tauTensor}
\end{equation}\\
The set of quantities $d\tau_m^{k_1\,...\,k_m}$ is called the {\elevenit extension} of the infinitesimal $m$-cell. No metrical concepts have entered into the definition of extension. Extension is the 
nearest concept closest to the idea of elementary volume one can define in a non-metrical space. \\

The edges of the infinitesimal $m$-cell can be written as
\begin{equation}
d_\beta x^k =  {\partial x^k \over \partial y^\alpha }\,d_\beta y^\alpha
\end{equation}


\begin{theorem}[Stoke's Theorem]
\[ \oint_C \, A_p\, {dx^p\over ds} ds = - \iint_S \, \epsilon^{pqr}\,A_{q,r}\,\nu_p\,dS, \] 
where $\displaystyle{dx^p\over ds}$ is the unit tangent vector to the closed curve $C$ and
$\nu_p$ is the positive unit normal to the surface $S$ which has $C$ as its boundary.
\end{theorem}


%+1 & \mbox{if } k1,...,km \mbox{ are distinct integers selected from the range }  1,2,..,n \mbox{ and if  } s_1,...,s_m \mbox{ is an even permutation of}      \\ 
%-1 & \mbox{if } \mbox{ is an odd permutation}\\
%0  & \mbox{if } \mbox{ has any duplicates}
%+1 &  \mbox{if $k1,...,km$ are distinct integers \\ %selected from the range $1,2,..,n$ and if $s_1,...,s_m$ is an {\it even} permutation of $k_1,...,km $  \\
%-1 & \mbox{if } r \ne s   \\ 
%0 if  
%\end{cases} \]



%\begin{equation}
%\det |\overline{g}| = \sum_{P \in S_{n-1}} \epsilon_{ p_1 p_2 ... p_{n-1}}  \left(\sum_{kl} T
%\end{equation}





%In order to deal with the determinant of the matrix products involved in $\det \gstar$ we define the following sets $J$ and $K$:
%Let $J$, $K$ be a set of $n-1$ numbers chosen from $1,...,n$. 
%$$J, K\subset \{1,...,n\} \hbox{~each of cardinality~} n-1.$$
%Using this notation we can expand the determinant of $\gstar$ as follows

%\begin{equation}
%\det \left[ \gstar \right] = \sum \epsilon_{ j_1 j_2 ... j_{n-1}} 
%\end{equation}

%\begin{equation}
%\det \left[ \gstar \right] = \sum_{J,K}  \det\left[ {\partial x\over \partial t} \right]^T_{\star J} \det\left[ g \right]_{JK} \det\left[ {\partial x\over \partial t} \right]_{K\star} \label{sumdets}
%\end{equation}

\section{Applications In Mechanics}

\begin{definition}
Lie Algebras. Here $F = \R$ or $\C$. A Lie Algebra of $F$ is a pair $(\mathfrak{g}, [\cdot,\cdot])$, where $\mathfrak{g}$ is a vector space over $F$ and
$$[\cdot,\cdot]: \mathfrak{g} \times \mathfrak{g} \rightarrow \mathfrak{g}$$ is an F-blinear map satisfying the following properties 
 \begin{align*}
 &X, Y] = - [Y,X]\\
 &[X,[Y,Z]]] +  [Z,[X,Y]]] +  [Y,[Z,X]]] = 0,\
 \end{align*}The later is the Jacobi identity.
\end{definition}

\begin{definition}
Structure Constants: Given the elements $X_a, X_b$ and $X_c$ in a Lie Algebra we define the Structure Constraints $c_{ab}^c$ to be the coefficients in the expansion of the Lie Bracket: 
$$[X_a,X_b] = \sum_c c_{ab}^c\, X_c.$$
% example of Display Style:  $[X_a,X_b] = \displaystyle{\sum_c} c_{ab}^c\, X_c$.
\end{definition}


%\input{FourOutbounds.tex}
\clearpage

%\input{Appendices.tex}c`
%\input{Acknowledgements.tex}
%\clearpage

\begin{thebibliography}{9}
\bibitem{Schreier and Sperner} {O. Schreier and F. Sperner,``Introduction to Modern Algebra and Matrix Theory'', Second Edition, Chelsea Publishing, (1959).}
% \bibitem{Protter}{M.H. Protter and C.B. Morrey, ``A First Course in Real Analysis'', Springer-Verlag, Inc., New York (1977).}
% \bibitem{Hall}{F. M. Hall, ``An Introduction to Abstract Algebra'', Volume 2, Cambridge University Press, (1969).}
% \bibitem{Stensby}{John Stensby, ``EE603 Class Notes'', \url{http://www.ece.uah.edu/courses/ee385/}, University of Alabama, Huntsville (2016).}
% \bibitem{Baudoin}{Fabrice Baudoin,``Stochastic Processes and Brownian Motion'',\url{http://www.math.purdue.edu/~fbaudoin/MA539.pdf}, Purdue University, Indiana (2010).}
% \bibitem{vanZanten}{Harry van Zanten,``An Introduction to Stochastic Processes in Continuous Time'', \url{http://www.math.vu.nl/sto/onderwijs/sp/sp_2007.pdf}, University of Leiden (2007).}
% \bibitem{Weiner1}{Howard Weiner, UC Davis Mathematics Department, private communication (2007-2017).}
% \bibitem{Mahalanobis}{Mahalanobis, Prasanta Chandra (1936).``On the generalised distance in statistics'', Proceedings of the National Institute of Sciences of India. 2 (1): 49�55.}
% \bibitem{Einstein}{Albert Einstein, ``Investigations on the Theory of the Brownian Movement'', Dover Publications, Inc.(1956).}
% \bibitem {Ross1} {Sheldon Ross, ``A First Course in Probability'', First Edition, Macmillan Publishers Co, Inc., New York, NY (1976).}
% \bibitem {Ross2} {Sheldon Ross, ``Introduction to Probability Models'', Sixth Edition, Academic Press, San Diego, CA  (1997).}
% \bibitem{Parzen}{Emanuel Parzen, ``Stochastic Processes'', Holden-Day, Inc., San Francisco, CA (1962).}
% \bibitem{Medhi}{Jyotiprasad Medhi, ``Stochastic Processes'', Wiley Eastern Limited, New Delhi (1983).} 
% \bibitem{Jazwinski}{Andrew H. Jazwinski,``Stochastic Processes and Filtering Theory'', Dover Publications, Inc., Mineola, NY (2007).}
% \bibitem{Wiener1}{Norbert Wiener, ``Cybernetics: or Control and Communication in the Animal and the Machine'', The MIT Press, Cambridge, MA (1948).}
% \bibitem{Wiener2}{Norbert Wiener,``Extrapolation, Interpolation, and Smoothing of Stationary Time Series with Engineering Applications'', First Paperback Edition, The MIT Press, Cambridge, MA (1964). }  
% \bibitem{Wiener3}{Norbert Wiener, ``Generalized Harmonic Analysis'', Acta Mathematica, V. 55, p. 117 (1930).}  
% \bibitem{Oskendal1}{Bernt  {\O}skendal, ``Stochastic Differential Equations, An Introduction with Applications'', Third Edition, Springer-Verlag, Berlin (1992).}
% \bibitem{Oskendal2}{Bernt  {\O}skendal, ``Stochastic Differential Equations, An Introduction with Applications'', Sixth Edition, Springer-Verlag, Berlin (2005).}
% \bibitem{Doob}{J. L. Doob, ``Stochastic Processes'', John Wiley \& Sons, Inc., New York, (1953).}
% \bibitem{Sjogren}{Lennart Sjogren. ``Brownian Motion: Langevin Equation'', Chapter 6, \url{http://physics.gu.se/~frtbm/joomla/media/mydocs/LennartSjogren/kap6.pdf}}
% \bibitem{CalTech}{``Statistical Mechanics'', Physics 127b \url{http://www.cmp.caltech.edu/~mcc/Ph127/b/Lecture16.pdf}}
\end{thebibliography}

%\begin{description}
%\elevenitem[\elevenit Reflexive.] Since $x = e^{-1}xe$, x is conjugate to itself.
%\elevenitem[\elevenit Symmetric.] If $y=g^{-1}xg$ we have $x=gyg^{-1} = g^{-1})^{-1}y g^{-1}$.
%\elevenitem[\elevenit Transitive.] If $y=g^{-1}xg$ and $z=h^{-1}yh$ we have $z=h^{-1}g^{-1}xgh=(gh)^{-1}x(gh).$
%\end{description}

%\begin{eqnarray*}
%   P_1 & = & (E_1,0,0,-E_1),      \\
%   P_2 & = & (E_2,E_2\sin\theta\cos\phi,E_2\sin\theta\sin\phi,
%               -E_2\cos\theta),   \\
%   P_3 & = & (E_3,0,0,\beta E_3),      \\
%   q   & = & (E_1 - E_2,-E_2\sin\theta\cos\phi,-E_2\sin\theta\sin\phi,
%                    E_2\cos\theta - E_1).
%  \end{eqnarray*}

%\begin{figure*}[hb]
%\epsfysize=6cm %%%% whatever vertical size you want in cm or inches
%\centerline{\epsfbox{user$1:[smith.tex.eps]feyn.eps}}
%\caption[fig1]{Definition of Kinematic Variables.}
%\label{fig:feyn}
%\end{figure*}

%  \begin{eqnarray*}
%   y\     & \approx & 1 - {E_2\over E_1} {(1+\cos\theta)\over2}, \\
%    \qsq \ & \approx & {E_2\over E_1}{(1-\cos\theta)\over2}, \\
%    x      & \approx & {Q^2\over sy}.
%  \end{eqnarray*}

%\[ \begin{array}{lll} 
%  {\cal M}^{++}=({\cal M}^{22} + {\cal M}^{11})/2, 
%& {\cal M}^{+0}=({\cal M}^{23} -i{\cal M}^{13})/\sqrt{2},
%& {\cal M}^{+-}=({\cal M}^{22} - {\cal M}^{11})/2, \\ [8pt]
%  {\cal M}^{0+}=({\cal M}^{32} +i{\cal M}^{31})/\sqrt{2}, 
%& {\cal M}^{00}= {\cal M}^{33}, 
%& {\cal M}^{0-}=({\cal M}^{32} -i{\cal M}^{31})/\sqrt{2}, \\ [8pt]
%  {\cal M}^{-+}=({\cal M}^{22} - {\cal M}^{11})/2, 
%& {\cal M}^{-0}=({\cal M}^{23} +i{\cal M}^{13})/\sqrt{2}, 
%& {\cal M}^{--}=({\cal M}^{22} + {\cal M}^{11})/2. \\ [8pt]
%\end{array} \] 

%\[ \begin{array}{lll} 
%  {\cal M}^{++}=({\cal M}^{22} + {\cal M}^{11})/2, 
%& {\cal M}^{+0}= {\cal M}^{23}/\sqrt{2},
%& {\cal M}^{+-}=({\cal M}^{22} - {\cal M}^{11})/2, \\ [8pt]
%  {\cal M}^{0+}= {\cal M}^{32}/\sqrt{2}, 
%& {\cal M}^{00}= {\cal M}^{33}, 
%& {\cal M}^{0-}= {\cal M}^{32}/\sqrt{2}, \\ [8pt]
%  {\cal M}^{-+}=({\cal M}^{22} - {\cal M}^{11})/2, 
%& {\cal M}^{-0}= {\cal M}^{23}/\sqrt{2}, 
%& {\cal M}^{--}=({\cal M}^{22} + {\cal M}^{11})/2. 
%\end{array} \]
%
%\begin{thebibliography}{99}
%\bibitem{BCDMS} J.J.Aubert et al., EMC Collaboration,
% Nucl.Phys. B259(1985)189;\\
% A.C.Benvenuti et al., BCDMS Collaboration, Phys.Lett.B223(1989)485
%\bibitem{kaiserf}     F. Eisele, {\em First Results from the H1
%Experiment at HERA}, and
%                      F. Brasse, {\em The H1 Detector at HERA},
%                      Invited talks, Proceedings of the
%            26th International Conference on High Energy Physics,
%                   Dallas (1992) and DESY preprint 92-140 (1992)
%\bibitem{H1-249} I.\, Abt, J.R.\,Smith, {\em MC Upgrades to Study
%Untagged Events}, H1 Internal Note H1-249, (1992).
%H1-249 considered only the diagonal elements of the Photon Flux
%for photons coupled to Spin-1/2 particles.
%\end{thebibliography}
 %
 \end{document}
