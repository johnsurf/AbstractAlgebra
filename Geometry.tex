 \section{Notes on Riemannian Hypersurfaces}
 

 Let us summarize some facts about Riemannian Geometry:\\
 
Tangent Vectors $\r_j= \partial\r/\partial x^j $, $\a = a^i\r_i$, second derivatives: $\r_{ij} = {\partial^2 \r\over \partial x^i\partial x^j}$

Metric: $\gij= \r_i\cdot\r_j = g_{ji}$ (symmetric).

Dot product: $\a\cdot\b= \gij a^ib^j$.\\

Consider the determinant 
\begin{equation}
g = \det \left| \gij \right| =
\begin{vmatrix}
g_{11} & g_{12}  & \hdots & g_{1n}  \\
g_{21} & g_{22}  & \hdots & g_{2n}  \\
\hdotsfor[2]{4}\\
g_{i1} & g_{i2}  & \hdots & g_{in}  \\
\hdotsfor[2]{4}\\
g_{n1} & g_{n2}  & \hdots & g_{nn}  \\
\end{vmatrix}
\end{equation}


%Consider an $n$-by-$n$ matrix $A$ with elements $a_{ij}$. Take the columns of $A$ to be $n$-dimensional vectors $$A = \left[ \a_1, \a_2, ...., \a_n \right].$$ Divide these 
%vectors into to systems. Let the first consist of the $p$ vectors $$\a_{\alpha_1}, \a_{\alpha_2},...,\a_{\alpha_p},  (\alpha_1 < \alpha_2 <...< \alpha_p)$$, the second of the remaining $q =n-p$
%vectors $$\a_{\beta_1},\a_{\beta_2},...,\a_{\beta_q}, (\beta_1 < \beta_2 < ... < \beta_q, \quad p+q = n).$$ 
%The Laplace's Expansion Theorem of a determinant in terms of Complementary Minors is given by



%Affine Connection: 
%\begin{eqnarray*}
%\Gamma_{ijm} &=& \r_{ij}\cdot \r_m = \half \left({\partial g_{im}\over \partial x^j} +{\partial g_{jm}\over \partial x^i} - {\partial g_{ij}\over \partial x^m}\right), \\
%\Gamma^k_{ij} &=& g^{km}\Gamma_{ijm}= \half g^{km}\left({\partial g_{im}\over \partial x^j} +{\partial g_{jm}\over \partial x^i} - {\partial g_{ij}\over \partial x^m}\right).
%\end{eqnarray*}

%\begin{eqnarray*}
%\Gamma^j_{ij} &=& \half g^{jm}\left({\partial g_{im}\over \partial x^j} +{\partial g_{jm}\over \partial x^i} - {\partial g_{ij}\over \partial x^m}\right), \\
%                       &=& \half g^{jm}{\partial g_{jm}\over \partial x^i},       \\
%                       &=& \half {1\over g}{\partial g\over \partial x^i},\qquad\qquad\hbox{see below,}\\
%                       &=& \half {\partial \log{g}\over \partial x^i},                 \\
%                       &=& {1\over\sqrt{g}}{\partial \sqrt{g}\over \partial x^i}.
%\end{eqnarray*}

We suppose, here and throughout, that $g$ is not zero. Let $\Delta^{ij}$ be the cofactor of $\gij$ in this determinant, so that 
$$g_{mr} \Delta^{ms} = g_{rm}\Delta^{sm} = \delta^s_r\, g,$$ which follows from the ordinary rules for developing a determinant. \\
 
Let us construct new quantities $\gijinv$\cite{Bergmann}. 
The values of the components of $\gijinv$ are equal to the cofactor of the $\gij$, divided by the full determinant $g$,
\begin{equation} g^{kl} = g^{-1} \Delta^{kl}.  \label{ginveq} \end{equation} Alternatively the cofactor of $\gij$ can be expressed in terms of $\gijinv$ as folllows
\begin{equation} \Delta^{ij} = g\,g^{ij}\label{cofactor}. \end{equation}
$\gijinv$ satisfies the following equations:
\begin{equation}
g_{ik}g^{kl}  = \delta^l_i \label{ginv}\hbox{~~inverse}
\end{equation}
Now $\gij$ is symmetric and we can also show that $\gijinv$ is symmetric. Multiply Eq. (\ref{ginv}) by $g_{ls}g^{ir}$. The left hand side becomes
\[ g_{ik}g^{kl} g_{ls}g^{ir} =g_{ki}g^{ir} g^{kl}g_{sl} = \delta^r_k g^{kl} g_{sl} = g_{sl}g^{rl}, \] while the right hand side becomes
\[ \delta^l_i g_{ls}g^{ir} = g_{is}g^{ir} = g_{si}g^{ir} = \delta^r_s.\] Therefore we obtain 
\begin{equation}
g_{sl}g^{rl} = \delta^r_s \hbox{~~inverse}\label{ginvtr}
\end{equation}

Comparing Eqs (\ref{ginv}) and (\ref{ginvtr}), we find that
\[ g^{kl} = g^{lk} \]

Since $\gij$ is symmetric, it is obvious that $\gijinv$ should be symmetric also. \\

Let us examine the implications of the above applied to surfaces embedded in $\R^n$. 
A {\elevenit curve} is defined as the totality of points given by the equations

\begin{equation}
x^r = f^r(u)\quad (r = 1,2,...,n).\label{curve}
\end{equation}
Here $u$ is a parameter and $f^r(u)$ are $n$ functions.\\

Next consider the totality of points given by 
\begin{equation} x^r = f^r(t^1, t^2, ..., t^m)\quad (r= 1,2,...,n),\label{subspace}\end{equation}
where the $t$'s are parameters and $m<n$. This set of points forms $V_m$, a subspace of $V_n$. 
According to the Implicit Function Theorem it is possible to eliminate the parameters from (\ref{subspace}). As long as the system is differentiable enough and the the Jacobian of the first $m$ equations does not vanish, we can 
find $m$ functions $g^i(x^1,...,x^m)$ such that  \begin{equation} t^i = g^i(x^1,...,x^m) \mbox{ for } i = 1,...,m\end{equation}which implies that the remaining functions $x^r, r = m+1,...,n$ can be written in terms of the first $m$ coordinates 
$x^1, ..., x^m$ as follows:
\begin{equation} x^r = f^r(x^1, ..., x^m) \mbox{ for } r = m+1, ..., n\end{equation} In the case of $m=n-1$ there is only one such relation and
so, elimination gives just one equation 

\begin{equation} \phi(x^1, x^2, ..., x^n) = x^n - f^n(x^1,...,x^m) = 0.\label{hypersurface}\end{equation}

In this case, $V_{n-1}$ divides portions of $V_n$ into two parts for which respectively $\phi$ is positive and negative. $V_{n-1}$ is commonly referred to as a {\elevenit hypersurface} in $V_n$.\\

Let $V_{n}$ be a region of $\R^n$, bounded by a smooth (i.e. sufficiently differentiable) surface $B_{n-1}$, which is an $n-1$ dimensional closed manifold. We take $B_{n-1}$ as an example of a hypersurface $V_{n-1}$.
We has suppose that $B_{n-1}$ is parameterized by $n-1$ independent parameters $t^\alpha (\alpha = 1,2,...,n-1)$. Let the hypersurface $B_{n-1}$ be defined implicitly by the equation (\ref{hypersurface}).\\

On the surface of $B_{n-1}$, $x^i = x^i(t^\alpha)$, and therefore
\begin{equation} {\partial \phi\over \partial x^j}{\partial x^j\over \partial t^\alpha} = 0. \label{surfaceTangents} \end{equation}

Equation (\ref{surfaceTangents}) may be regarded as $n-1$  linear conditions upon the $n$ quantities ${\partial \phi / \partial x^j}$. Since the matrix 
$$\left[
{\partial x^j \over \partial t^a} 
\right]$$

is assumed to have its full rank value $n-1$ on $B_{n-1}$, the mutual  ratios of the partials $\partial \phi / \partial x^j$ are fully determined by
(\ref{surfaceTangents}).\\

For future reference let us define a set of determinants with respect to the change of variables between the $x^i$ and the $t^\alpha$

\begin{equation}
 D_k = (-1)^{k-1} \det\left[ {\partial(x^1, ..., x^{k-1},x^{k+1},...,x^n)\over \partial (t^1, ..., t^{n-1})} \right], ~~k = 1,2,...,n \label{dsubk}
 \end{equation}
The $n$ determinants $D_k$ satisfy the $n-1$ linear conditions 
\begin{equation} D_k {\partial x^k\over \partial t^\alpha} = 0,\label{DkEq} \end{equation}
which can be established by noting that the left hand side can be written as an $n$-by-$n$ determinant which has two rows the same and therefore vanishes.
In view of (\ref{surfaceTangents}, conditions (\ref{DkEq}) imply that the $\partial \phi / \partial x^k$ and $D_k$ are proportional
\[ {\partial \phi\over \partial x^k} =  \alpha D_k,\] for some scalar factor of proportionality $\alpha$. Hence the $\partial \phi/ \partial x^k$ determine the unit 
normal to $B_{n-1}$, namely

\begin{eqnarray}
n_k &=& {1\over \sqrt{(\nabla \phi)^2}} {\partial \phi\over \partial x^k}\\
n_k &=& {D_k\over\sqrt{g^{ij} D_iD_j}} = {D_k\over D},\label{normB}
\end{eqnarray}
where \[ D^2 = g^{ij} D_iD_j = g^{ij} {\partial \phi\over \partial x^i} {\partial \phi\over \partial x^j } \alpha^{-2}.\]

%Covariant Derivative:
%$$V^i_{;j} = {\partial V^i\over \partial x^j} + V^k\Gamma^i_{kj}.$$
%$$V_{i;j}  = {\partial V_i\over \partial x^j} - V_k\Gamma^k_{ij}.$$
%
%Covariant Divergence of a Contravariant Vector:
%$$\eqalign{
%V^j_{;j} &= {\partial V^j\over \partial x^j} + V^i\Gamma^j_{ij},\crno
%&={1\over\sqrt{g}}\left({\partial (\sqrt{g}V^i)\over\partial x^i}\right).\cr}$$
%
%Variation of $g$, Definitions:\hfil\break
%$g=\cofac(\gab)\gab=\cofac(\gabinv)\gabinv$,\hfil\break
%$\gabinv=\cofac(\gab)/g$,\hfil\break
%$\gab =\cofac(\gabinv)/g$.
%$$\eqalign{\delta g=&\cofac(\gabinv)\delta\gabinv\crno
%=&g\gab\delta\gabinv.\cr}$$
%$$\eqalign{\delta\sqrt{-g}=&-\half\delta g/\sqrt{-g}\crno
%=&\sqrt{-g}(-\half\gab\delta\gabinv).\cr}$$
%$$\delta(\sqrt{-g}\gmunuinv)= 
%\sqrt{-g}(\delta\gmunuinv - \half\gab\gmunuinv\delta\gabinv).$$
%Reduction of unity: Consider a set of n mutually orthogonal unit vectors
%$$e^j_{(1)},e^j_{(2)},...,e^j_{(n)}$$
%Then $$\gij e^i_{(l)}e^j_{(m)} = \delta_{lm}.$$
%If we set $$e_{(m)i} = \gij e^j_{(m)},$$
%this becomes $$e^i_{(l)}e_{(m)i} = \delta_{lm}.$$ The basis vectors can
%be arranged into a matrix which is orthogonal. The rows of an orthogonal
%matrix also form an orthogonal system. This implies 
%$$\sum^n_{m=1} e^i_{(m)}e_{(m)j} = \delta^i_j.$$ Multiplying this by 
%$g^{jk}$ and summing over $j$, we get
%$$\sum^n_{m=1}e^i_{(m)}g^{jk} e_{(m)j} = \delta^i_jg^{jk} = g^{ik}.$$
%From the above definitions we have:
%$$g^{jk}e_{(m)j} = g^{jk}g_{jr}e^r_{(m)} = \delta^k_re^r_{(m)} = e^k_{(m)}.$$
%So we get the well known expression for reduction of unity:
%$$g^{ik}= \sum^n_{m=1}e^i_{(m)}e^k_{(m)}.$$
%
%Curvature Tensor and derivatives:
%$$R^i_{ jkl} = 
%{\partial \Gamma^i_{jl}\over\partial u^k} -
%{\partial \Gamma^i_{jk}\over\partial u^l} +
%\Gamma^p_{jl}\Gamma^i_{pk} - \Gamma^p_{jk}\Gamma^i_{pl}.$$
%$$R_{\mu\nu} = R^\rho_{\mu\nu\rho}.$$
%$$R_{\mu\nu}\delta(\sqrt{-g}\gmunuinv)= 
%\sqrt{-g}(R_{\mu\nu} - \half\gmunu R)\delta\gmunuinv.$$
%Let $G_{\mu\nu} = R_{\mu\nu} - \half\gmunu R$, then
%$$R_{\mu\nu}\delta(\sqrt{-g}\gmunuinv)= 
%\sqrt{-g}G_{\mu\nu}\delta\gmunuinv.$$
%
\section{Hypersurfaces and Concepts of Area}
Consider an $n-1$ dimensional hypersurface, $B_{n-1}$, embedded in a Riemannian manifold $M_n$ (which we assume to be sufficiently differentiable). Let $x^1,...,x^n$ be local coordinates in $M_n$ and suppose that
$B_{n-1}$ is represented locally in the form 
\[ x^j = x^j(t^1, ..., t^{n-1}),  \quad j = 1,...,n \] 

The metric tensor of $M_n$ is denoted by $\gij$. and the metric tensor on $B_{n-1}$ by $\overline{g}_{\alpha\beta}$. These two tensors are related by
\begin{equation}
 \gstar= \gij {\partial x^i\over \partial t^\alpha} {\partial x^j\over \partial t^\beta} = {\partial x^i\over \partial t^\alpha} \gij {\partial x^j\over \partial t^\beta} \label{gstar}
\end{equation}

The element of area (or volume) in $M_n$ is 
\begin{equation}
dV = \sqrt{g}\,dx^1...dx^n,\label{areag}
\end{equation}
where $g$ is the determinant of the metric tensor in $M_n$. Similarly the element of area (volume) in the hypersurface $B_{n-1}$ is given by 
\begin{equation}
dS = \sqrt{\overline{g}}\,dt^1...dt^{n-1}.\label{areagstar}
\end{equation} 
We will use (\ref{gstar}) to define the element of area in $B_{n-1}$ based on expansion of the determinant of $\overline{g}$ in terms of products of the various minors in a manner similar to the
Cauchy-Binet formula or the Laplace expansion of a determinant. Let $S_n$ stand for the set of permutations of the numbers $\{1,2....,n \}$ and let $P$ represent one of the members of $S_n$.\\

$\gstar$ is an ($n$-1)-by-($n$-1) matrix and we can consider its determinant. 
\begin{equation}
\overline{g} = \det |\overline{g}| = \sum_{P \in S_{n-1}} \epsilon_{ p_1 p_2 ... p_{n-1}} \overline{g}_{1 p_1}  \overline{g}_{2 p_2} ... \overline{g}_{n-1 p_{n-1}} \label{detgstar}
\end{equation}

We can think of the right hand side of (\ref{gstar}) as representing the matrix product of three matrices. 
\begin{eqnarray*}
J &=&  \left[ {\partial x^i\over \partial t^\alpha} \right] \hbox{~is $n$-by-$(n-1)$},\\ 
G &=&\bigg[ g \bigg] \hbox{~is $n$-by-$n$}
\end{eqnarray*}
 and (\ref{gstar}) can be written in the form:
\begin{eqnarray*}
 \overline{g}_{\alpha\beta} &=& \sum_{i,j=1}^n \left[ {\partial x^i\over \partial t^\alpha} \right]^T_{\alpha i} \left[g\right]_{ij} \left[ {\partial x^i\over \partial t^\beta} \right]_{j\beta} = {\partial x^i\over \partial t^\alpha} \gij {\partial x^j\over \partial t^\beta} \hbox{~summation convention on $i$ and $j$}\\
\bigg[ \gstar\bigg] &=& J^T G J \\
\end{eqnarray*}
Inserting this computation into (\ref{detgstar}) we have
$\gstar$ is an ($n$-1)-by-($n$-1) matrix and we can consider its determinant.  Let $\overline{g} =\det |\overline{g}_{\alpha\beta} |$.

\begin{eqnarray*}
\overline{g} &=& \sum_{P \in S_{n-1}} \epsilon_{ p_1 p_2 ... p_{n-1}} \sum_{\substack{i1,..,i_{n-1} = 1 \\ k_1,...,k_{n-1}=1}}^n  
(J^T_{1i_1}G_{i_1k_1}J_{k_1p_1})
(J^T_{2i_2}G_{i_2k_2}J_{k_2p_2})...
(J^T_{n-1i_{n-1}}G_{i_{n-1}k_{n-1}}J_{k_{n-1}p_{n-1}})\\
&=& \sum_{P \in S_{n-1}} \epsilon_{ p_1 p_2 ... p_{n-1}} \sum_{\substack{i1,..,i_{n-1} = 1 \\ k_1,...,k_{n-1}=1}}^n  
(J^T_{1i_1}...J^T_{n-1i_{n-1}} )
(G_{i_1k_1}...G_{i_{n-1}k_{n-1}} )
(J_{k1p_1}...J_{k_{n-1}p_{n-1}} )\\
&=& \sum_{\substack{i1,..,i_{n-1} = 1 \\ k_1,...,k_{n-1}=1}}^n  
(J^T_{1i_1}...J^T_{n-1i_{n-1}} )
(G_{i_1k_1}...G_{i_{n-1}k_{n-1}} )
\sum_{P \in S_{n-1}} \epsilon_{ p_1 p_2 ... p_{n-1}} (J_{k1p_1}...J_{k_{n-1}p_{n-1}} )\\
\end{eqnarray*}
Examining the last term in the sum on the right hand side we find a sum over the $n-1$ set $\{p1,p2,...,p_{n-1}\}$ for a particular choice of index values: $\{k1,...k_{n-1}\}$ 
\begin{equation}
\sum_{P \in S_{n-1} }\epsilon_{ p_1 p_2 ... p_{n-1}} (J_{k1p_1}...J_{k_{n-1}p_{n-1}}) = \epsilon_{ k_1 k_2 ... k_{n-1}} | J_k | = \epsilon_{ k_1 k_2 ... k_{n-1}} (-1)^{1+k} D_k
\end{equation}
where the index $k$ is not included in $\{k1,...k_{n-1}\}$ and final result involves the respective determinant $D_k$.\\

At this point notice that the sum over the $n-1$ variables $k1,k2,...,k_{n-1}$ involves summing over all values from $1$ to $n$ for each $k_i$, but that these values must be distinct. If $k_i = k_j$ for some pair $i,j$, then the sum is
zero because of the presence of the permutation symbol $\epsilon_{ k_1 k_2 ... k_{n-1}}$. This means that there are $n-1$ distinct $k$ values chosen from the full set of $n$ values, therefore one of the $k_i$ values is to be omitted in this sum.  Let $k$ stand for this missing value and $J_k$ stand for the matrix $J$ with the $k$-th row deleted. The set of $k_i$ values in any set is also propagated on to the $G_{ik}$ matrices and the corresponding $k$-th column of $G$ will not contribute this particular set of choices. There will be $n$ such cases and therefore we insert a sum over $k=1,2,...,n$ to account for each case in the following. The sum over the $i_1,...,i_{n-1}$ indices follows that same way.

\begin{eqnarray*}
\overline{g} &=& \sum_{k=1}^n \sum_{\substack{i1,..,i_{n-1} = 1 \\ k_1,...,k_{n-1}=1, \ne k}}^n  
(J^T_{1i_1}...J^T_{n-1i_{n-1}} )
(G_{i_1k_1}...G_{i_{n-1}k_{n-1}} )
\epsilon_{ k_1 k_2 ... k_{n-1}} (-1)^{1+k} D_{k}\\
&=& \sum_{i, k =1}^n \sum_{i1,..,i_{n-1} = 1, \ne i }^n  
(J^T_{1i_1}...J^T_{n-1i_{n-1}} )
\epsilon_{ i_1 i_2 ... i_{n-1}} (-1)^{i + k} \hbox{cofactor}(g_{ik})
 (-1)^{1+k} D_{k}\\
&=& \sum_{i, k =1}^n \sum_{~~~i1,..,i_{n-1} = 1,\ne i}^n  
\epsilon_{ i_1 i_2 ... i_{n-1}} (J^T_{1i_1}...J^T_{n-1i_{n-1}} )
(-1)^{i + k} \Delta^{ik}
(-1)^{1+k} D_{k}\\
&=& \sum_{i,k =1}^n  
(-1)^{1+i} D_{i}
(-1)^{i + k} \Delta^{ik}
(-1)^{1+k} D_{k}\\
&=& \sum_{i,k =1}^n  D_{i} \,g \,g^{ik} \,D_{k}= g \sum_{i= 1}^n  D^{k} \,D_{k} = g\, D^2,\\ 
\label{expdetgstar}
\end{eqnarray*}

where $g = |\gij|$ as before and the sum is over the $n$ values of $i$ and $k$ which correspond to the rows or columns which are crossed out to form the $n$ minors each of dimension ($n$-1)-by-($n$-1). Hence we find that
$$\overline{g} = g\,D^2.$$

We finally derive the relationship between the element of surface are on the hyperplane $H$ and the metric of the $n$-dimensional Riemann $M_n$ space that $H$ is embedded inside. 

\begin{equation}
dS = \sqrt{\overline{g}}\,dt^1...dt^{n-1} =  \sqrt{g\,D^2} \,dt^1...dt^{n-1}  \label{relArea}
\end{equation}

If the $n$-1 dimensional hypersurface $H$ which is embedded in Cartesian $\R^n$ is given in parametric form as follows:
$$\mathfrak{r}(x) = \{ x, f(x) \} $$ 
Then $\gij  = I_{n}$ ($n$-by-$n$ identity matrix) and $J$ is an $n$-by-$n-1$ matrix:

\begin{equation}
J  =
\begin{bmatrix}
1 & 0 & \hdots  & 0 \\
0 & 1  & \hdots & 0 \\
\hdotsfor[2]{4}\\
0 & 0  & \hdots & 1 \\
{\partial f\over \partial x^1} & {\partial f\over \partial x^2}   & \hdots & {\partial f\over \partial x^{n-1}} \\
\end{bmatrix}
\end{equation}

\begin{equation}
J^T  =
\begin{bmatrix}
1 & 0 & \hdots  & & 0 & {\partial f\over \partial x^1}  \\
0 & 1  & \hdots & & 0 &{\partial f\over \partial x^2}  \\
\hdotsfor[2]{5}\\
0 & 0  & \hdots & & 1 &{\partial f\over \partial x^n} \\
\end{bmatrix}
\end{equation}

Working out the minors, we find 
$D_n = 1$ (cross out the last row of $J$), and crossing out any of the other rows $i<n$ inserts a $0$ in the $i$-th diagonal and shifts all the other rows $i+1, ..., n$ up one row. The resulting determinant is equal to  
$D_k = (-1)^{1+k}{\partial f\over \partial x_k}$
 
Therefore we obtain 

\begin{equation}
\overline{g} = \sum_{k=1}^n D_k = ({\partial f\over \partial x^1})^2 + ({\partial f\over \partial x^2})^2 + ... +  ({\partial f\over \partial x^n})^2 + 1  
\end{equation} 
 
And the resulting magnitude, given that $g_{ij} = I_n$ is 
$\sqrt{D^2} = \sqrt{1 + |\nabla f|^2}$ and

\begin{equation}
dS = \sqrt{1 + | \nabla f |^2} ~dt^1...dt^{n-1}.\label{Stephen}
\end{equation}

\section{Application to the Green Theorem}

In Riemannian Geometry the divergence of a vector $\b$ is given by 

$$ \nabla \cdot \b = {1\over \sqrt{g} }  {\partial\over \partial x^i} \big(\sqrt{g}\, b^i\big) \hbox{~summation convention}$$

Let us form the integral of the divergence of $\b$ over a region $R$ in $V_n$ bounded by the closed surface $B_{n-1}$

\begin{equation}
\int_{R} \, \nabla \cdot \b\, dV = \int_R \, {1\over \sqrt{g} }  {\partial\over \partial x^i} \,\big(\sqrt{g}\, b^i\big)\,  dV =\int_R\, {1\over \sqrt{g} }  {\partial\over \partial x^i} \big(\sqrt{g}\, b^i\big)\,\sqrt{g}\, dx^1...dx^n 
\end{equation}

Now let us integrate the differentiated terms over the range of the appropriate variable for each one. We obtain an integral over the boundary surface $B_{n-1}$: for the $k$ term this is ($k$ not summed)
\begin{equation}
\int_B \, \sqrt{g}\, b^k dx^1...dx^{k-1}dx^{k+1}...dx^n = \int_B \, \sqrt{g} \, b^k D_k\, dt^1...dt^{n-1},
\end{equation}

where $D_k$ is given as in (\ref{dsubk}) represents the transformation of the integration variables from the $x^i$ to the coordinates as given on $B_{n-1}$.\\

From the above set of equations we see that 
\begin{eqnarray*}
\int_{R_n}\, \nabla \cdot \b \, dV_n &=& \int_{B_{n-1}} \, b^k D_k \sqrt{g}\, dt^1...dt^{n-1}\\
&=& \int_{B_{n-1}}\, b^k n_k \sqrt{g}\, D\ dt^1...dt^{n-1}\\
&=& \int_{B_{n-1}} \b \cdot \mathfrak{n} \, dS,
\end{eqnarray*}
where $dS$ is given by (\ref{relArea}).

\begin{theorem}[Divergence Theorem]
\[\iiint_V \, A^k_{,k} \, dV = \iint_S \, A^k\, \nu_k \, dS \]
\end{theorem}

\section{Extension and Stokes' Theorem}

In a space of $n$ dimensions, the {\bf Generalized Kronecker delta} $\delta^{k_1\,k_2\,...\,k_m}_{s_1\,s_2\,...\,s_m}$ for any positive integer $m$ is defined as 
\begin{itemize}
\item[]$\delta^{k_1\,k_2\,...k_m}_{s_1\,s_2\,...s_m} = +1$ if $k_1\,k_2\,...k_m$ are distinct integers selected from the range $1, 2, ..., n$, \\
\null~~~~~~~~~~~~~~~~~~~and if $s_1, s_2, ...,s_m$ is an {\elevenit even} permutation of  $k_1\,k_2\,...k_m$.
\item[]$\delta^{k_1\,k_2\,...k_m}_{s_1\,s_2\,...s_m} = -1$ if $k_1\,k_2\,...k_m$ are distinct integers selected from the range $1, 2, ..., n$, \\
\null~~~~~~~~~~~~~~~~~~~and if $s_1, s_2, ...,s_m$ is an {\elevenit odd} permutation of  $k_1\,k_2\,...k_m$.
\item[]$\delta^{k_1\,k_2\,...k_m}_{s_1\,s_2\,...s_m} = 0$ if any two $k_1\,k_2\,...k_m$ are, equal, or if any two $s_1, s_2, ...,s_m$ are equal, or if the set \\
\null~~~~~~~~~~~~~~~~~of numbers $\{k_1, k_2, ..., k_m\}$ differs, apart from order, from the set $\{s_1, s_2, ..., s_m\}$.
\end{itemize} 

In a space of $n$ dimensions, in which no metric is assigned, consider a subspace $V_m$ of $m$ dimensions $m\le n$ defined by the parametric equations
$$x^k = x^k(y^1, y^2, ..., y^m),~ k = 1, 2, ..., n.$$ In this $V_m$ consider a certain region of $R_m$. Let us divide $R_m$ into cells by $m$ families of surfaces
$$f^\alpha(y) = c^\alpha,~ \alpha = 1, 2, ..., m.$$ where $c^\alpha$ are constants, each taking on a number of discrete values and so forming a family of surfaces.\\

It is easy to attach a meaning to the statement that a point $P$ of $V_m$ lies in a certain cell, because $P$ may be said to lie between two surfaces of the same 
family, say family $\alpha$, if the expression $f^\alpha(y) - c^\alpha$ changes sign when we change $c^\alpha$ from the value belonging to one of these surfaces to 
the value belonging to the other. Here $y$ refers to the parameter values at the point $P$.\\

Now pick one of the cells and let $A$ be the point corresponding to one of the corners in the cell. Through $A$ there passes one surface of each family. Let the 
corresponding constants be $c^1, c^2, ..., c^m$. Passing along the edge for which $c^1$ alone changes, we arrive at another corner $B_1$ for which the constants
have values $c^1 + \Delta c^1, c^2, ..., c^m$. In passing from $A$ to $B_1$ the parameters change from $y^\alpha$ at $A$ to $y^\alpha + \Delta_1 y^\alpha$ at $B_1$
and the coordinates from $x^k$ to $x^k + \Delta_1 x^k$. Similarly the parameter values and the coordinates of all corners $B_1, B_2, ..., B_m$ of the cell reached by
traveling from $A$ along an edge of the cell can be written as 
\begin{eqnarray*}
&& y^\alpha + \Delta_\beta y^\alpha,\\
&&x^\alpha + \Delta_\beta x^k.
\end{eqnarray*} and we form the determinants 

\begin{equation}
\Delta^{k_1\,...\,k_m}  =  
\begin{vmatrix}
\Delta_1 x^{k_1} & \Delta_1 x^{k_2} & \hdots & \Delta_1 x^{k_m} \\
\Delta_2 x^{k_1} & \Delta_2 x^{k_2} & \hdots & \Delta_2 x^{k_m} \\
\hdotsfor[2]{4}\\
\hdotsfor[2]{4}\\
\Delta_m x^{k_1} & \Delta_m x^{k_2} & \hdots & \Delta_m x^{k_m} \\
\end{vmatrix}\label{detDeltak}
\end{equation}
Introducing the generalized Kronecker delta $\delta^{k_1\,k_2\,...\,k_m}_{s_1\,s_2\,...\,s_m}$, these determinants can be written compactly using the Summation Convention
\begin{equation}
\Delta^{k_1\,...\,k_m}  =  \delta^{k_1\,k_2\,...\,k_m}_{s_1\,s_2\,...\,s_m}\, \Delta_1 x^{s_1}\, \Delta_2 x^{s_2} \hdots \Delta_m x^{s_m}\label{Deltaktensor}
\end{equation}
Now if the edges had been taken in a different order, then the various determinants might have opposite signs, but apart from this the determinants are independent of the order taken. 
Also if we have used a different corner to start from the determinants formed similarly would have different values. But since we are going to be working with infinitesimal, the differences 
will be of higher order than the determinants in (\ref{Deltaktensor}). Using differential displacements, $d_\beta x^i$ (where $\beta$ indexes the $\beta$-th edge of the cell) we have
\begin{equation}
d\tau_m^{k_1\,...\,k_m}  =  \delta^{k_1\,k_2\,...\,k_m}_{s_1\,s_2\,...\,s_m}\, d_1 x^{s_1}\, d_2 x^{s_2} \hdots d_m x^{s_m}\label{tauTensor}
\end{equation}\\
The set of quantities $d\tau_m^{k_1\,...\,k_m}$ is called the {\elevenit extension} of the infinitesimal $m$-cell. No metrical concepts have entered into the definition of extension. Extension is the 
nearest concept closest to the idea of elementary volume one can define in a non-metrical space. \\

The edges of the infinitesimal $m$-cell can be written as
\begin{equation}
d_\beta x^k =  {\partial x^k \over \partial y^\alpha }\,d_\beta y^\alpha
\end{equation}


\begin{theorem}[Stoke's Theorem]
\[ \oint_C \, A_p\, {dx^p\over ds} ds = - \iint_S \, \epsilon^{pqr}\,A_{q,r}\,\nu_p\,dS, \] 
where $\displaystyle{dx^p\over ds}$ is the unit tangent vector to the closed curve $C$ and
$\nu_p$ is the positive unit normal to the surface $S$ which has $C$ as its boundary.
\end{theorem}


%+1 & \mbox{if } k1,...,km \mbox{ are distinct integers selected from the range }  1,2,..,n \mbox{ and if  } s_1,...,s_m \mbox{ is an even permutation of}      \\ 
%-1 & \mbox{if } \mbox{ is an odd permutation}\\
%0  & \mbox{if } \mbox{ has any duplicates}
%+1 &  \mbox{if $k1,...,km$ are distinct integers \\ %selected from the range $1,2,..,n$ and if $s_1,...,s_m$ is an {\it even} permutation of $k_1,...,km $  \\
%-1 & \mbox{if } r \ne s   \\ 
%0 if  
%\end{cases} \]



%\begin{equation}
%\det |\overline{g}| = \sum_{P \in S_{n-1}} \epsilon_{ p_1 p_2 ... p_{n-1}}  \left(\sum_{kl} T
%\end{equation}





%In order to deal with the determinant of the matrix products involved in $\det \gstar$ we define the following sets $J$ and $K$:
%Let $J$, $K$ be a set of $n-1$ numbers chosen from $1,...,n$. 
%$$J, K\subset \{1,...,n\} \hbox{~each of cardinality~} n-1.$$
%Using this notation we can expand the determinant of $\gstar$ as follows

%\begin{equation}
%\det \left[ \gstar \right] = \sum \epsilon_{ j_1 j_2 ... j_{n-1}} 
%\end{equation}

%\begin{equation}
%\det \left[ \gstar \right] = \sum_{J,K}  \det\left[ {\partial x\over \partial t} \right]^T_{\star J} \det\left[ g \right]_{JK} \det\left[ {\partial x\over \partial t} \right]_{K\star} \label{sumdets}
%\end{equation}
